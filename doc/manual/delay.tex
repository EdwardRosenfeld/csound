\begin{comment}
\documentclass[10pt]{article}
\usepackage{fullpage, graphicx, url}
\setlength{\parskip}{1ex}
\setlength{\parindent}{0ex}
\title{delay}
\begin{document}


\begin{tabular}{ccc}
The Alternative Csound Reference Manual & & \\
Previous & &Next

\end{tabular}

%\hline 
\end{comment}
\section{delay}
delay�--� Delays an input signal by some time interval. \subsection*{Description}


  A signal can be read from or written into a delay path, or it can be automatically delayed by some time interval. 
\subsection*{Syntax}


 ar \textbf{delay}
 asig, idlt [, iskip]
\subsection*{Initialization}


 \emph{idlt}
 -- requested delay time in seconds. This can be as large as available memory will permit. The space required for n seconds of delay is 4n * \emph{sr}
 bytes. It is allocated at the time the instrument is first initialized, and returned to the pool at the end of a score section. 


 \emph{iskip}
 (optional, default=0) -- initial disposition of delay-loop data space (see \emph{reson}
). The default value is 0. 
\subsection*{Performance}


 \emph{asig}
 -- audio signal 


 \emph{delay}
 is a composite of \emph{delayr}
 and \emph{delayw}
, both reading from and writing into its own storage area. It can thus accomplish signal time-shift, although modified feedback is not possible. There is no minimum delay period. 
\subsection*{Examples}


  Here is an example of the delay opcode. It uses the files \emph{delay.orc}
 and \emph{delay.sco}
. 


 \textbf{Example 1. Example of the delay opcode.}

\begin{lstlisting}
/* delay.orc */
; Initialize the global variables.
sr = 44100
kr = 4410
ksmps = 10
nchnls = 2

; Instrument #1 -- Delayed beeps.
instr 1
  ; Make a basic sound.
  abeep vco 20000, 440, 1

  ; Delay the beep by .1 seconds.
  idlt = 0.1
  adel delay abeep, idlt

  ; Send the beep to the left speaker and
  ; the delayed beep to the right speaker.
  outs abeep, adel
endin
/* delay.orc */
        
\end{lstlisting}
\begin{lstlisting}
/* delay.sco */
; Table #1, a sine wave.
f 1 0 16384 10 1

; Keep the score running for 2 seconds.
f 0 2

; Play Instrument #1.
i 1 0.0 0.2
i 1 0.5 0.2
e
/* delay.sco */
        
\end{lstlisting}
\subsection*{See Also}


 \emph{delay1}
, \emph{delayr}
, \emph{delayw}

\subsection*{Credits}


 Example written by Kevin Conder.
%\hline 


\begin{comment}
\begin{tabular}{lcr}
Previous &Home &Next \\
dconv &Up &delay1

\end{tabular}


\end{document}
\end{comment}
