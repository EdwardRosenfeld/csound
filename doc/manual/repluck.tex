\begin{comment}
\documentclass[10pt]{article}
\usepackage{fullpage, graphicx, url}
\setlength{\parskip}{1ex}
\setlength{\parindent}{0ex}
\title{repluck}
\begin{document}


\begin{tabular}{ccc}
The Alternative Csound Reference Manual & & \\
Previous & &Next

\end{tabular}

%\hline 
\end{comment}
\section{repluck}
repluck�--� Physical model of the plucked string. \subsection*{Description}


 \emph{repluck}
 is an implementation of the physical model of the plucked string. A user can control the pluck point, the pickup point, the filter, and an additional audio signal, \emph{axcite}
. \emph{axcite}
 is used to excite the 'string'. Based on the Karplus-Strong algorithm. 
\subsection*{Syntax}


 ar \textbf{repluck}
 iplk, kamp, icps, kpick, krefl, axcite
\subsection*{Initialization}


 \emph{iplk}
 -- The point of pluck is \emph{iplk}
, which is a fraction of the way up the string (0 to 1). A pluck point of zero means no initial pluck. 


 \emph{icps}
 -- The string plays at \emph{icps}
 pitch. 
\subsection*{Performance}


 \emph{kamp}
 -- Amplitude of note. 


 \emph{kpick}
 -- Proportion of the way along the string to sample the output. 


 \emph{krefl}
 -- the coefficient of reflection, indicating the lossiness and the rate of decay. It must be strictly between 0 and 1 (it will complain about both 0 and 1). 
\subsection*{Performance}


 \emph{axcite}
 -- A signal which excites the string. 
\subsection*{Examples}


  Here is an example of the repluck opcode. It uses the files \emph{repluck.orc}
 and \emph{repluck.sco}
. 


 \textbf{Example 1. Example of the repluck opcode.}

\begin{lstlisting}
/* repluck.orc */
; Initialize the global variables.
sr = 44100
kr = 4410
ksmps = 10
nchnls = 1

; Instrument #1.
instr 1
  iplk = 0.75
  kamp = 30000
  icps = 220
  kpick = 0.75
  krefl = 0.5
  axcite oscil 1, 1, 1

  apluck repluck iplk, kamp, icps, kpick, krefl, axcite

  out apluck
endin
/* repluck.orc */
        
\end{lstlisting}
\begin{lstlisting}
/* repluck.sco */
; Table #1, a sine wave.
f 1 0 16384 10 1

; Play Instrument #1 for two seconds.
i 1 0 2
e
/* repluck.sco */
        
\end{lstlisting}
\subsection*{See Also}


 \emph{wgpluck2}

\subsection*{Credits}


 


 


\begin{tabular}{cccc}
Author: John ffitch &University of Bath/Codemist Ltd. &Bath, UK &1997

\end{tabular}



 
%\hline 


\begin{comment}
\begin{tabular}{lcr}
Previous &Home &Next \\
release &Up &reson

\end{tabular}


\end{document}
\end{comment}
