\begin{comment}
\documentclass[10pt]{article}
\usepackage{fullpage, graphicx, url}
\setlength{\parskip}{1ex}
\setlength{\parindent}{0ex}
\title{clfilt}
\begin{document}


\begin{tabular}{ccc}
The Alternative Csound Reference Manual & & \\
Previous & &Next

\end{tabular}

%\hline 
\end{comment}
\section{clfilt}
clfilt�--� Implements low-pass and high-pass filters of different styles. \subsection*{Description}


  Implements the classical standard analog filter types: low-pass and high-pass. They are implemented with the four classical kinds of filters: Butterworth, Chebyshev Type I, Chebyshev Type II, and Elliptical. The number of poles may be any even number from 2 to 80. 
\subsection*{Syntax}


 ar \textbf{clfilt}
 asig, kfreq, itype, inpol [, ikind] [, ipbr] [, isba] [, iskip]
\subsection*{Initialization}


 \emph{itype}
 -- 0 for low-pass, 1 for high-pass. 


 \emph{inpol}
 -- The number of poles in the filter. It must be an even number from 2 to 80. 


 \emph{ikind}
 (optional) -- 0 for Butterworth, 1 for Chebyshev Type I, 2 for Chebyshev Type II, 3 for Elliptical. Defaults to 0 (Butterworth) 


 \emph{ipbr}
 (optional) -- The pass-band ripple in dB. Must be greater than 0. It is ignored by Butterworth and Chebyshev Type II. The default is 1 dB. 


 \emph{isba}
 (optional) -- The stop-band attenuation in dB. Must be less than 0. It is ignored by Butterworth and Chebyshev Type I. The default is -60 dB. 


 \emph{iskip}
 (optional) -- 0 initializes all filter internal states to 0. 1 skips initialization. The default is 0. 
\subsection*{Performance}


 \emph{asig}
 -- The input audio signal. 


 \emph{kfreq}
 -- The corner frequency for low-pass or high-pass. 
\subsection*{Examples}


  Here is an example of the clfilt opcode as a low-pass filter. It uses the files \emph{clfilt\_lowpass.orc}
 and \emph{clfilt\_lowpass.sco}
. 


 \textbf{Example 1. Example of the clfilt opcode as a low-pass filter.}

\begin{lstlisting}
/* clfilt_lowpass.orc */
; Initialize the global variables.
sr = 22050
kr = 2205
ksmps = 10
nchnls = 1

; Instrument #1 - an unfiltered noise waveform.
instr 1
  ; White noise signal
  asig rand 22050

  out asig
endin


; Instrument #2 - a filtered noise waveform.
instr 2
  ; White noise signal
  asig rand 22050

  ; Lowpass filter signal asig with a 
  ; 10-pole Butterworth at 500 Hz.
  a1 clfilt asig, 500, 0, 10

  out a1
endin
/* clfilt_lowpass.orc */
        
\end{lstlisting}
\begin{lstlisting}
/* clfilt_lowpass.sco */
; Play Instrument #1 for two seconds.
i 1 0 2
; Play Instrument #2 for two seconds.
i 2 2 2
e
/* clfilt_lowpass.sco */
        
\end{lstlisting}


  Here is an example of the clfilt opcode as a high-pass filter. It uses the files \emph{clfilt\_highpass.orc}
 and \emph{clfilt\_highpass.sco}
. 


 \textbf{Example 2. Example of the clfilt opcode as a high-pass filter.}

\begin{lstlisting}
/* clfilt_highpass.orc */
; Initialize the global variables.
sr = 22050
kr = 2205
ksmps = 10
nchnls = 1

; Instrument #1 - an unfiltered noise waveform.
instr 1
  ; White noise signal
  asig rand 22050

  out asig
endin


; Instrument #2 - a filtered noise waveform.
instr 2
  ; White noise signal
  asig rand 22050

  ; Highpass filter signal asig with a 6-pole Chebyshev
  ; Type I at 20 Hz with 3 dB of passband ripple.
  a1 clfilt asig, 20, 1, 6, 1, 3

  out a1
endin
/* clfilt_highpass.orc */
        
\end{lstlisting}
\begin{lstlisting}
/* clfilt_highpass.sco */
; Play Instrument #1 for two seconds.
i 1 0 2
; Play Instrument #2 for two seconds.
i 2 2 2
e
/* clfilt_highpass.sco */
        
\end{lstlisting}
\subsection*{Credits}


 Author: Erik Spjut


 New in version 4.20
%\hline 


\begin{comment}
\begin{tabular}{lcr}
Previous &Home &Next \\
clear &Up &clip

\end{tabular}


\end{document}
\end{comment}
