\begin{comment}
\documentclass[10pt]{article}
\usepackage{fullpage, graphicx, url}
\setlength{\parskip}{1ex}
\setlength{\parindent}{0ex}
\title{gogobel}
\begin{document}


\begin{tabular}{ccc}
The Alternative Csound Reference Manual & & \\
Previous & &Next

\end{tabular}

%\hline 
\end{comment}
\section{gogobel}
gogobel�--� Audio output is a tone related to the striking of a cow bell or similar. \subsection*{Description}


  Audio output is a tone related to the striking of a cow bell or similar. The method is a physical model developed from Perry Cook, but re-coded for Csound. 
\subsection*{Syntax}


 ar \textbf{gogobel}
 kamp, kfreq, ihrd, ipos, imp, kvibf, kvamp, ivfn
\subsection*{Initialization}


 \emph{ihrd}
 -- the hardness of the stick used in the strike. A range of 0 to 1 is used. 0.5 is a suitable value. 


 \emph{ipos}
 -- where the block is hit, in the range 0 to 1. 


 \emph{imp}
 -- a table of the strike impulses. The file \emph{marmstk1.wav}
 is a suitable function from measurements and can be loaded with a \emph{GEN01}
 table. It is also available at \emph{\url{ftp://ftp.cs.bath.ac.uk/pub/dream/documentation/sounds/modelling/}}
. 


 \emph{ivfn}
 -- shape of vibrato, usually a sine table, created by a function. 
\subsection*{Performance}


  A note is played on a cowbell-like instrument, with the arguments as below. 


 \emph{kamp}
 -- Amplitude of note. 


 \emph{kfreq}
 -- Frequency of note played. 


 \emph{kvibf}
 -- frequency of vibrato in Hertz. Suggested range is 0 to 12 


 \emph{kvamp}
 -- amplitude of the vibrato 
\subsection*{Examples}


  Here is an example of the gogobel opcode. It uses the files \emph{gogobel.orc}
, \emph{gogobel.sco}
, and \emph{marmstk1.wav}
, 


 \textbf{Example 1. Example of the gogobel opcode.}

\begin{lstlisting}
/* gogobel.orc */
; Initialize the global variables.
sr = 22050
kr = 2205
ksmps = 10
nchnls = 1

; Instrument #1.
instr 1
  ; kamp = 31129.60
  ; kfreq = 440
  ; ihrd = 0.5
  ; ipos = 0.561
  ; imp = 1
  ; kvibf = 6.0
  ; kvamp = 0.3
  ; ivfn = 2

  a1 gogobel 31129.60, 440, 0.5, 0.561, 1, 6.0, 0.3, 2
  out a1
endin
/* gogobel.orc */
        
\end{lstlisting}
\begin{lstlisting}
/* gogobel.sco */
; Table #1, the "marmstk1.wav" audio file.
f 1 0 256 1 "marmstk1.wav" 0 0 0
; Table #2, a sine wave for the vibrato.
f 2 0 128 10 1

; Play Instrument #1 for one second.
i 1 0 1
e
/* gogobel.sco */
        
\end{lstlisting}
\subsection*{Credits}


 


 


\begin{tabular}{ccc}
Author: John ffitch (after Perry Cook) &University of Bath, Codemist Ltd. &Bath, UK

\end{tabular}



 


 New in Csound version 3.47
%\hline 


\begin{comment}
\begin{tabular}{lcr}
Previous &Home &Next \\
gbuzz &Up &goto

\end{tabular}


\end{document}
\end{comment}
