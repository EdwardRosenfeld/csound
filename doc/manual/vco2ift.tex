\begin{comment}
\documentclass[10pt]{article}
\usepackage{fullpage, graphicx, url}
\setlength{\parskip}{1ex}
\setlength{\parindent}{0ex}
\title{vco2ift}
\begin{document}


\begin{tabular}{ccc}
The Alternative Csound Reference Manual & & \\
Previous & &Next

\end{tabular}

%\hline 
\end{comment}
\section{vco2ift}
vco2ift�--� Returns a table number at i-time for a given oscillator frequency and wavform. \subsection*{Description}


 \emph{vco2ift}
 is the same as \emph{vco2ft}
, but works at i-time. It is suitable for use with opcodes that expect an i-rate table number (for example, \emph{oscili}
). 
\subsection*{Syntax}


 ifn \textbf{vco2ift}
 icps, iwave [, inyx]
\subsection*{Initialization}


 \emph{ifn}
 -- the ftable number. 


 \emph{icps}
 -- frequency in Hz. Zero and negative values are allowed. However, if the absolute value exceeds \emph{sr}
/2 (or sr*inyx), the selected table will contain silence. 


 \emph{iwave}
 -- the waveform for which table number is to be selected. Allowed values are: 


 
\begin{itemize}
\item 

 0: sawtooth

\item 

 1: 4 * x * (1 - x) (integrated sawtooth)

\item 

 2: pulse (not normalized)

\item 

 3: square wave

\item 

 4: triangle


\end{itemize}


  Additionally, negative \emph{iwave}
 values select user defined waveforms (see also \emph{vco2init}
). 


 \emph{inyx}
 (optional, default=0.5) -- bandwidth of the generated waveform, as percentage (0 to 1) of the sample rate. The expected range is 0 to 0.5 (i.e. up to \emph{sr}
/2), other values are limited to the allowed range. 


  Setting inyx to 0.25 (sr/4), or 0.3333 (sr/3) can produce a ``fatter'' sound in some cases, although it is more likely to reduce quality. 
\subsection*{See Also}


 \emph{vco2ft}
, \emph{vco2init}
, and \emph{vco2}
. 
\subsection*{Credits}


 Author: Istvan Varga


 New in version 4.22
%\hline 


\begin{comment}
\begin{tabular}{lcr}
Previous &Home &Next \\
vco2ft &Up &vco2init

\end{tabular}


\end{document}
\end{comment}
