\begin{comment}
\documentclass[10pt]{article}
\usepackage{fullpage, graphicx, url}
\setlength{\parskip}{1ex}
\setlength{\parindent}{0ex}
\title{zakinit}
\begin{document}


\begin{tabular}{ccc}
The Alternative Csound Reference Manual & & \\
Previous & &Next

\end{tabular}

%\hline 
\end{comment}
\section{zakinit}
zakinit�--� Establishes zak space. \subsection*{Description}


  Establishes zak space. Must be called only once. 
\subsection*{Syntax}


 \textbf{zakinit}
 isizea, isizek
\subsection*{Initialization}


 \emph{isizea}
 -- the number of audio rate locations for a-rate patching. Each location is actually an array which is ksmps long. 


 \emph{isizek}
 -- the number of locations to reserve for floats in the zk space. These can be written and read at i- and k-rates. 
\subsection*{Performance}


  At least one location each is always allocated for both za and zk spaces. There can be thousands or tens of thousands za and zk ranges, but most pieces probably only need a few dozen for patching signals. These patching locations are referred to by number in the other zak opcodes. 


  To run \emph{zakinit}
 only once, put it outside any instrument definition, in the orchestra file header, after \emph{sr}
, \emph{kr}
, \emph{ksmps}
, and \emph{nchnls}
. 
\subsection*{Examples}


  Here is an example of the zakinit opcode. It uses the files \emph{zakinit.orc}
 and \emph{zakinit.sco}
. 


 \textbf{Example 1. Example of the zakinit opcode.}

\begin{lstlisting}
/* zakinit.orc */
; Initialize the global variables.
sr = 44100
kr = 4410
ksmps = 10
nchnls = 1

; Initialize the ZAK space.
; Create 3 a-rate variables and 5 k-rate variables.
zakinit 3, 5

; Instrument #1 -- a simple waveform.
instr 1
  ; Generate a simple sine waveform.
  asin oscil 20000, 440, 1

  ; Send the sine waveform to za variable #1.
  zaw asin, 1
endin

; Instrument #2 -- generates audio output.
instr 2
  ; Read za variable #1.
  a1 zar 1

  ; Generate audio output.
  out a1

  ; Clear the za variables, get them ready for 
  ; another pass.
  zacl 0, 3
endin
/* zakinit.orc */
        
\end{lstlisting}
\begin{lstlisting}
/* zakinit.sco */
; Table #1, a sine wave.
f 1 0 16384 10 1

; Play Instrument #1 for one second.
i 1 0 1
; Play Instrument #2 for one second.
i 2 0 1
e
/* zakinit.sco */
        
\end{lstlisting}
\subsection*{Credits}


 


 


\begin{tabular}{ccc}
Author: Robin Whittle &Australia &May 1997

\end{tabular}



 


 Example written by Kevin Conder.
%\hline 


\begin{comment}
\begin{tabular}{lcr}
Previous &Home &Next \\
zacl &Up &zamod

\end{tabular}


\end{document}
\end{comment}
