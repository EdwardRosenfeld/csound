\begin{comment}
\documentclass[10pt]{article}
\usepackage{fullpage, graphicx, url}
\setlength{\parskip}{1ex}
\setlength{\parindent}{0ex}
\title{spsend}
\begin{document}


\begin{tabular}{ccc}
The Alternative Csound Reference Manual & & \\
Previous & &Next

\end{tabular}

%\hline 
\end{comment}
\section{spsend}
spsend�--� Generates output signals based on a previously defined \emph{space}
 opcode. \subsection*{Description}


 \emph{spsend}
 depends upon the existence of a previously defined \emph{space}
. The output signals from \emph{spsend}
 are derived from the values given for xy and reverb in the \emph{space}
 and are ready to be sent to local or global reverb units (see example below). 
\subsection*{Syntax}


 a1, a2, a3, a4 \textbf{spsend}

\subsection*{Performance}


  The configuration of the xy coordinates in space places the signal in the following way: 


 
\begin{itemize}
\item 

 a1 is -1, 1

\item 

 a2 is 1, 1

\item 

 a3 is -1, -1

\item 

 a4 is 1, -1


\end{itemize}


  This assumes a loudspeaker set up as a1 is left front, a2 is right front, a3 is left back, a4 is right back. Values greater than 1 will result in sounds being attenuated, as if in the distance. \emph{space}
 considers the speakers to be at a distance of 1; smaller values of xy can be used, but \emph{space}
 will not amplify the signal in this case. It will, however balance the signal so that it can sound as if it were within the 4 speaker \emph{space}
. x=0, y=1, will place the signal equally balanced between left and right front channels, x=y=0 will place the signal equally in all 4 channels, and so on. Although there must be 4 output signals from \emph{space}
, it can be used in a 2 channel orchestra. If the xy's are kept so that Y$>$=1, it should work well to do panning and fixed localization in a stereo field. 
\subsection*{Examples}


 


 
\begin{lstlisting}
\emph{instr}
 1
  asig    ;some audio signal
  ktime              \emph{line}
  0, p3, p10
  a1, a2, a3, a4     \emph{space}
 asig,1, ktime, .1
  ar1, ar2, ar3, ar4 \emph{spsend}
        
  
  ga1 = ga1+ar1
  ga2 = ga2+ar2
  ga3 = ga3+ar3
  ga4 = ga4+ar4
  
                     \emph{outq}
 a1, a2, a3, a4
\emph{endin}


\emph{instr}
 99 ; reverb instrument
          
  a1 \emph{reverb2}
 ga1, 2.5, .5
  a2 \emph{reverb2}
 ga2, 2.5, .5
  a3 \emph{reverb2}
 ga3, 2.5, .5
  a4 \emph{reverb2}
 ga4, 2.5, .5
  
     \emph{outq}
 a1, a2, a3, a4
  ga1=0
  ga2=0
  ga3=0
  ga4=0
        
\end{lstlisting}


 


  In the above example, the signal, \emph{asig}
, is moved according to the data in Function \#1 indexed by \emph{ktime}
. \emph{space}
 sends the appropriate amount of the signal internally to \emph{spsend}
. The outputs of the \emph{spsend}
 are added to global accumulators in a common Csound style and the global signals are used as inputs to the reverb units in a separate instrument. 


 \emph{space}
 can useful for quad and stereo panning as well as fixed placed of sounds anywhere between two loudspeakers. Below is an example of the fixed placement of sounds in a stereo field using xy values from the score instead of a function table. 


 
\begin{lstlisting}
\emph{instr}
 1
  ...
  a1, a2, a3, a4     \emph{space}
 asig, 0, 0, .1, p4, p5
  ar1, ar2, ar3, ar4 \emph{spsend
  }

  ga1=ga1+ar1
  ga2=ga2+ar2
                     \emph{outs}
  a1, a2
\emph{endin}


\emph{instr}
 99 ; reverb....
  ....
\emph{endin}

        
\end{lstlisting}


 


  A few notes: p4 and p5 are the X and Y values 


 
\begin{lstlisting}
  ;place the sound in the left speaker and near
    i1 0 1 -1 1
  ;place the sound in the right speaker and far
    i1 1 1 45 45
  ;place the sound equally between left and right and in the middle ground distance
    i1 2 1 0 12
e
        
\end{lstlisting}


 


  The next example shows a simple intuitive use of the distance values returned by \emph{spdist}
 to simulate Doppler shift. 


 
\begin{lstlisting}
  ktime              \emph{line}
   0, p3, 10
  kdist              \emph{spdist}
 1, ktime
  kfreq = (ifreq * 340) / (340 + kdist)
  asig               \emph{oscili}
 iamp, kfreq, 1
  
  a1, a2, a3, a4     \emph{space}
  asig, 1, ktime, .1
  ar1, ar2, ar3, ar4 \emph{spsend}

        
\end{lstlisting}


 


  The same function and time values are used for both \emph{spdist}
 and \emph{space}
. This insures that the distance values used internally in the \emph{space}
 unit will be the same as those returned by \emph{spdist}
 to give the impression of a Doppler shift! 
\subsection*{See Also}


 \emph{space}
, \emph{spdist}

\subsection*{Credits}


 


 


\begin{tabular}{ccc}
Author: Richard Karpen &Seattle, WA USA &1998

\end{tabular}



 


 New in Csound version 3.48
%\hline 


\begin{comment}
\begin{tabular}{lcr}
Previous &Home &Next \\
spectrum &Up &sqrt

\end{tabular}


\end{document}
\end{comment}
