\begin{comment}
\documentclass[10pt]{article}
\usepackage{fullpage, graphicx, url}
\setlength{\parskip}{1ex}
\setlength{\parindent}{0ex}
\title{sr}
\begin{document}


\begin{tabular}{ccc}
The Alternative Csound Reference Manual & & \\
Previous & &Next

\end{tabular}

%\hline 
\end{comment}
\section{sr}
sr�--� Sets the audio sampling rate. \subsection*{Description}


  These statements are global value \emph{assignments}
, made at the beginning of an orchestra, before any instrument block is defined. Their function is to set certain \emph{reserved symbol variables}
 that are required for performance. Once set, these reserved symbols can be used in expressions anywhere in the orchestra. 
\subsection*{Syntax}


 \textbf{sr}
 = iarg
\subsection*{Initialization}


 \emph{sr}
 = (optional) -- set sampling rate to \emph{iarg}
 samples per second per channel. The default value is 10000. 


  In addition, any \emph{global variable}
 can be initialized by an \emph{init-time assignment}
 anywhere before the first \emph{instr statement}
. All of the above assignments are run as instrument 0 (i-pass only) at the start of real performance. 


  Beginning with Csound version 3.46, \emph{sr}
 may be omitted. Csound will attempt to calculate the omitted value from the specified values, but it should evaluate to an integer. 
\subsection*{Examples}


 


 
\begin{lstlisting}
\textbf{sr}
 = 10000
\textbf{kr}
 = 500
\textbf{ksmps}
 = 20
gi1 \textbf{= }
sr/2.
ga \textbf{init }
0
itranspose \textbf{= }
octpch(.0l)
        
\end{lstlisting}


 
\subsection*{See Also}


 \emph{kr}
, \emph{ksmps}
, \emph{nchnls}

%\hline 


\begin{comment}
\begin{tabular}{lcr}
Previous &Home &Next \\
sqrt &Up &stix

\end{tabular}


\end{document}
\end{comment}
