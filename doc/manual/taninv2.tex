\begin{comment}
\documentclass[10pt]{article}
\usepackage{fullpage, graphicx, url}
\setlength{\parskip}{1ex}
\setlength{\parindent}{0ex}
\title{taninv2}
\begin{document}


\begin{tabular}{ccc}
The Alternative Csound Reference Manual & & \\
Previous & &Next

\end{tabular}

%\hline 
\end{comment}
\section{taninv2}
taninv2�--� Returns an arctangent. \subsection*{Description}


  Returns the arctangent of \emph{iy/ix}
, \emph{ky/kx}
, or \emph{ay/ax}
. 
\subsection*{Syntax}


 ar \textbf{taninv2}
 ay, ax


 ir \textbf{taninv2}
 iy, ix


 kr \textbf{taninv2}
 ky, kx


  Returns the arctangent of \emph{iy/ix}
, \emph{ky/kx}
, or \emph{ay/ax}
. If y is zero, \emph{taninv2}
 returns zero regardless of the value of x. If x is zero, the return value is: 


 
\begin{itemize}
\item 

 \emph{PI/2}
, if y is positive.

\item 

 \emph{-PI/2}
, if y is negative.

\item 

 \emph{0}
, if y is 0.


\end{itemize}
\subsection*{Initialization}


 \emph{iy, ix}
 -- values to be converted 
\subsection*{Performance}


 \emph{ky, kx}
 -- control rate signals to be converted 


 \emph{ay, ax}
 -- audio rate signals to be converted 
\subsection*{Examples}


  Here is an example of the taninv2 opcode. It uses the files \emph{taninv2.orc}
 and \emph{taninv2.sco}
. 


 \textbf{Example 1. Example of the taninv2 opcode.}

\begin{lstlisting}
/* taninv2.orc */
; Initialize the global variables.
sr = 44100
kr = 4410
ksmps = 10
nchnls = 1

; Instrument #1.
instr 1
  ; Returns the arctangent for 1/2.
  i1 taninv2 1, 2

  print i1
endin
/* taninv2.orc */
        
\end{lstlisting}
\begin{lstlisting}
/* taninv2.sco */
; Play Instrument #1 for one second.
i 1 0 1
e
/* taninv2.sco */
        
\end{lstlisting}
 Its output should include a line like this: \begin{lstlisting}
instr 1:  i1 = 0.464
      
\end{lstlisting}
\subsection*{See Also}


 \emph{taninv}

\subsection*{Credits}


 


 


\begin{tabular}{cccc}
Author: John ffitch &University of Bath/Codemist Ltd. &Bath, UK &April 1998

\end{tabular}



 


 Example written by Kevin Conder.


 New in Csound version 3.48


 Corrected on May 2002, thanks to Istvan Varga.
%\hline 


\begin{comment}
\begin{tabular}{lcr}
Previous &Home &Next \\
taninv &Up &tbvcf

\end{tabular}


\end{document}
\end{comment}
