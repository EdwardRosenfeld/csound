\begin{comment}
\documentclass[10pt]{article}
\usepackage{fullpage, graphicx, url}
\setlength{\parskip}{1ex}
\setlength{\parindent}{0ex}
\title{lpshold}
\begin{document}


\begin{tabular}{ccc}
The Alternative Csound Reference Manual & & \\
Previous & &Next

\end{tabular}

%\hline 
\end{comment}
\section{lpshold}
lpshold�--� Generate control signal consisting of held segments. \subsection*{Description}


  Generate control signal consisting of held segments delimited by two or more specified points. The entire envelope is looped at kfreq rate. Each parameter can be varied at k-rate. 
\subsection*{Syntax}


 ksig \textbf{lpshold}
 kfreq, ktrig, ktime0, kvalue0 [, ktime1] [, kvalue1] [, ktime2] [, kvalue2] [...]
\subsection*{Performance}


 \emph{ksig}
 -- Output signal 


 \emph{kfreq}
 -- Repeat rate in Hz or fraction of Hz 


 \emph{ktrig}
 -- If non-zero, retriggers the envelope from start (see \emph{trigger opcode}
), before the envelope cycle is completed. 


 \emph{ktime0...ktimeN}
 -- Times of points; expressed in fraction of a cycle 


 \emph{kvalue0...kvalueN}
 -- Values of points 


 \emph{lpshold}
 is similar to \emph{loopseg}
, but can generate only horizontal segments, i.e. holds values for each time interval placed between \emph{ktimeN}
 and \emph{ktimeN+1}
. It can be useful, among other things, for melodic control, like old analog sequencers. 
\subsection*{Examples}


  Here is an example of the lpshold opcode. It uses the files \emph{lpshold.orc}
 and \emph{lpshold.sco}
. 


 \textbf{Example 1. Example of the lpshold opcode.}

\begin{lstlisting}
/* lpshold.orc */
; Initialize the global variables.
sr = 44100
kr = 4410
ksmps = 10
nchnls = 1

; Instrument #1
instr 1
  kfreq line 1, p3, 20

  klp lpshold kfreq, 0, 0, 0, p3*0.25, 20000, p3*0.75, 0

  a1 oscil klp, 440, 1
  out a1
endin
/* lpshold.orc */
        
\end{lstlisting}
\begin{lstlisting}
/* lpshold.sco */
; Table #1, a sine wave.
f 1 0 16384 10 1

; Play Instrument #1 for five seconds.
i 1 0 5
e
/* lpshold.sco */
        
\end{lstlisting}
\subsection*{See Also}


 \emph{loopseg}

\subsection*{Credits}


 Author: Gabriel Maldonado


 New in Version 4.13
%\hline 


\begin{comment}
\begin{tabular}{lcr}
Previous &Home &Next \\
lpreson &Up &lpslot

\end{tabular}


\end{document}
\end{comment}
