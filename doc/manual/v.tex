\begin{comment}
\documentclass[10pt]{article}
\usepackage{fullpage, graphicx, url}
\setlength{\parskip}{1ex}
\setlength{\parindent}{0ex}
\title{v Statement}
\begin{document}


\begin{tabular}{ccc}
The Alternative Csound Reference Manual & & \\
Previous & &Next

\end{tabular}

%\hline 
\end{comment}
\section{v Statement}
v�--� Provides for locally variable time warping of score events. \subsection*{Description}


  The \emph{v statement}
 provides for locally variable time warping of score events. 
\subsection*{Syntax}


 \textbf{v}
 p1
\subsection*{Initialization}


 \textbf{p1}
 -- Time warp factor (must be positive). 
\subsection*{Performance}


  The \emph{v statement}
 takes effect with the following \emph{i statement}
, and remains in effect until the next \emph{v statement}
, \emph{s statement}
, or \emph{e statement}
. 
\subsection*{Examples}


  The value of p1 is used as a multiplier for the start times (p2) of subsequent \emph{i statements}
. 


 


 
\begin{lstlisting}
\emph{i}
1   0 1  ;note1
\emph{v}
2
\emph{i}
1   1 1  ;note2
        
\end{lstlisting}


 
 In this example, the second note occurs two beats after the first note, and is twice as long. 

  Although the \emph{v statement}
 is similar to the \emph{t statement}
, the \emph{v statement}
 is local in operation. That is, \emph{v}
 affects only the following notes, and its effect may be cancelled or changed by another \emph{v statement}
. 


  Carried values are unaffected by the \emph{v statement}
 (see \emph{Carry}
). 


 


 
\begin{lstlisting}
\emph{i}
1   0 1  ;note1
\emph{v}
2
\emph{i}
1   1 .  ;note2
\emph{i}
1   2 .  ;note3
\emph{v}
1
\emph{i}
1   3 .  ;note4
\emph{i}
1   4 .  ;note5
\emph{e}

        
\end{lstlisting}


 


  In this example, note2 and note4 occur simultaneously, while note3 actually occurs before note2, that is, at its original place. Durations are unaffected. 


 


 
\begin{lstlisting}
\emph{i}
1   0 1
\emph{v}
2
\emph{i}
.   + .
\emph{i}
.   . .
        
\end{lstlisting}


 
 In this example, the \emph{v statement}
 has no effect. %\hline 


\begin{comment}
\begin{tabular}{lcr}
Previous &Home &Next \\
t Statement (Tempo Statement) &Up &x Statement

\end{tabular}


\end{document}
\end{comment}
