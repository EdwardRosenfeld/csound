\begin{comment}
\documentclass[10pt]{article}
\usepackage{fullpage, graphicx, url}
\setlength{\parskip}{1ex}
\setlength{\parindent}{0ex}
\title{FLsetBox}
\begin{document}


\begin{tabular}{ccc}
The Alternative Csound Reference Manual & & \\
Previous & &Next

\end{tabular}

%\hline 
\end{comment}
\section{FLsetBox}
FLsetBox�--� Sets the appearance of a box surrounding a FLTK widget. \subsection*{Description}


 \emph{FLsetBox}
 sets the appearance of a box surrounding the target widget. 
\subsection*{Syntax}


 \textbf{FLsetBox}
 itype, ihandle
\subsection*{Initialization}


 \emph{itype}
 -- an integer number that modify the appearance of the target widget. 


  Legal values for the \emph{itype}
 argument are: 


 
\begin{itemize}
\item 

 1 - flat box

\item 

 2 - up box

\item 

 3 - down box

\item 

 4 - thin up box

\item 

 5 - thin down box

\item 

 6 - engraved box

\item 

 7 - embossed box

\item 

 8 - border box

\item 

 9 - shadow box

\item 

 10 - rounded box

\item 

 11 - rounded box with shadow

\item 

 12 - rounded flat box

\item 

 13 - rounded up box

\item 

 14 - rounded down box

\item 

 15 - diamond up box

\item 

 16 - diamond down box

\item 

 17 - oval box

\item 

 18 - oval shadow box

\item 

 19 - oval flat box


\end{itemize}


 \emph{ihandle}
 -- an integer number (used as unique identifier) taken from the output of a previously located widget opcode (which corresponds to the target widget). It is used to unequivocally identify the widget when modifying its appearance with this class of opcodes. The user must not set the \emph{ihandle}
 value directly, otherwise a Csound crash will occur. 
\subsection*{See Also}


 \emph{FLcolor}
, \emph{FLcolor2}
, \emph{FLhide}
, \emph{FLlabel}
, \emph{FLsetAlign}
, \emph{FLsetColor}
, \emph{FLsetColor2}
, \emph{FLsetFont}
, \emph{FLsetPosition}
, \emph{FLsetSize}
, \emph{FLsetText}
, \emph{FLsetTextColor}
, \emph{FLsetTextSize}
, \emph{FLsetTextType}
, \emph{FLsetVal\_i}
, \emph{FLsetVal}
, \emph{FLshow}

\subsection*{Credits}


 Author: Gabriel Maldonado


 New in version 4.22
%\hline 


\begin{comment}
\begin{tabular}{lcr}
Previous &Home &Next \\
FLsetAlign &Up &FLsetColor

\end{tabular}


\end{document}
\end{comment}
