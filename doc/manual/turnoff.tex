\begin{comment}
\documentclass[10pt]{article}
\usepackage{fullpage, graphicx, url}
\setlength{\parskip}{1ex}
\setlength{\parindent}{0ex}
\title{turnoff}
\begin{document}


\begin{tabular}{ccc}
The Alternative Csound Reference Manual & & \\
Previous & &Next

\end{tabular}

%\hline 
\end{comment}
\section{turnoff}
turnoff�--� Enables an instrument to turn itself off. \subsection*{Description}


  Enables an instrument to turn itself off. 
\subsection*{Syntax}


 \textbf{turnoff}

\subsection*{Performance}


 \emph{turnoff}
 -- this p-time statement enables an instrument to turn itself off. Whether of finite duration or ``held'', the note currently being performed by this instrument is immediately removed from the active note list. No other notes are affected. 
\subsection*{Examples}


  The following example uses the turnoff opcode. It will cause a note to terminate when a control signal passes a certain threshold (here the Nyquist frequency). It uses the files \emph{turnoff.orc}
 and \emph{turnoff.sco}
. 


 \textbf{Example 1. Example of the turnoff opcode.}

\begin{lstlisting}
/* turnoff.orc */
; Initialize the global variables.
sr = 44100
kr = 4410
ksmps = 10
nchnls = 1

; Instrument #1.
instr 1
  k1 expon 440, p3/10,880     ; begin gliss and continue
  if k1 < sr/2  kgoto contin  ; until Nyquist detected
    turnoff  ; then quit

contin:
  a1 oscil 10000, k1, 1
  out a1
endin
/* turnoff.orc */
        
\end{lstlisting}
\begin{lstlisting}
/* turnoff.sco */
; Table #1: an ordinary sine wave.
f 1 0 32768 10 1

; Play Instrument #1 for 4 seconds.
i 1 0 4
e
/* turnoff.sco */
        
\end{lstlisting}
\subsection*{See Also}


 \emph{ihold}

%\hline 


\begin{comment}
\begin{tabular}{lcr}
Previous &Home &Next \\
trirand &Up &turnon

\end{tabular}


\end{document}
\end{comment}
