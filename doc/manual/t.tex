\begin{comment}
\documentclass[10pt]{article}
\usepackage{fullpage, graphicx, url}
\setlength{\parskip}{1ex}
\setlength{\parindent}{0ex}
\title{t Statement (Tempo Statement)}
\begin{document}


\begin{tabular}{ccc}
The Alternative Csound Reference Manual & & \\
Previous & &Next

\end{tabular}

%\hline 
\end{comment}
\section{t Statement (Tempo Statement)}
t�--� Sets the tempo. \subsection*{Description}


  This statement sets the tempo and specifies the accelerations and decelerations for the current section. This is done by converting beats into seconds. 
\subsection*{Syntax}


 \textbf{t}
 p1 p2 p3 p4 ... (unlimited)
\subsection*{Initialization}


 \emph{p1}
 -- Must be zero. 


 \emph{p2}
 -- Initial tempo on beats per minute. 


 \emph{p3, p5, p7,...}
 -- Times in beats per minute (in non-decreasing order). 


 \emph{p4, p6, p8,...}
 -- Tempi for the referenced beat times. 
\subsection*{Performance}


  Time and Tempo-for-that-time are given as ordered couples that define points on a ``tempo vs. time'' graph. (The time-axis here is in beats so is not necessarily linear.) The beat-rate of a Section can be thought of as a movement from point to point on that graph: motion between two points of equal height signifies constant tempo, while motion between two points of unequal height will cause an accelarando or ritardando accordingly. The graph can contain discontinuities: two points given equal times but different tempi will cause an immediate tempo change. 


  Motion between different tempos over non-zero time is inverse linear. That is, an accelerando between two tempos M1 and M2 proceeds by linear interpolation of the single-beat durations from 60/M1 to 60/M2. 


  The first tempo given must be for beat 0. 


  A tempo, once assigned, will remain in effect from that time-point unless influenced by a succeeding tempo, i.e. the last specified tempo will be held to the end of the section. 


  A \emph{t statement}
 applies only to the score section in which it appears. Only one \emph{t statement}
 is meaningful in a section; it can be placed anywhere within that section. If a score section contains no \emph{t statement}
, then beats are interpreted as seconds (i.e. with an implicit \emph{t 0 60}
 statement). 


  N.B. If the CSound command includes a \emph{-t flag}
, the interpreted tempo of all score \emph{t statements}
 will be overridden by the command-line tempo. 
%\hline 


\begin{comment}
\begin{tabular}{lcr}
Previous &Home &Next \\
s Statement &Up &v Statement

\end{tabular}


\end{document}
\end{comment}
