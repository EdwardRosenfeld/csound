\begin{comment}
\documentclass[10pt]{article}
\usepackage{fullpage, graphicx, url}
\setlength{\parskip}{1ex}
\setlength{\parindent}{0ex}
\title{cross2}
\begin{document}


\begin{tabular}{ccc}
The Alternative Csound Reference Manual & & \\
Previous & &Next

\end{tabular}

%\hline 
\end{comment}
\section{cross2}
cross2�--� Cross synthesis using FFT's. \subsection*{Description}


  This is an implementation of cross synthesis using FFT's. 
\subsection*{Syntax}


 ar \textbf{cross2}
 ain1, ain2, isize, ioverlap, iwin, kbias
\subsection*{Initialization}


 \emph{isize}
 -- This is the size of the FFT to be performed. The larger the size the better the frequency response but a sloppy time response. 


 \emph{ioverlap}
 -- This is the overlap factor of the FFT's, must be a power of two. The best settings are 2 and 4. A big overlap takes a long time to compile. 


 \emph{iwin}
 -- This is the function table that contains the window to be used in the analysis. One can use the \emph{GEN20}
 routine to create this window. 
\subsection*{Performance}


 \emph{ain1}
 -- The stimulus sound. Must have high frequencies for best results. 


 \emph{ain2}
 -- The modulating sound. Must have a moving frequency response (like speech) for best results. 


 \emph{kbias}
 -- The amount of cross synthesis. 1 is the normal, 0 is no cross synthesis. 
\subsection*{Examples}


  Here is an example of the cross2 opcode. It uses the files \emph{cross2.orc}
, \emph{cross2.sco}
 and \emph{beats.wav}
. 


 \textbf{Example 1. Example of the cross2 opcode.}

\begin{lstlisting}
/* cross2.orc */
; Initialize the global variables.
sr = 44100
kr = 4410
ksmps = 10
nchnls = 1

; Instrument #1 - Play an audio file.
instr 1
  ; Use the "beats.wav" audio file.
  aout soundin "beats.wav"
  out aout
endin

; Instrument #2 - Cross-synthesize!
instr 2
  ; Use the "ahh" sound stored in Table #1.
  ain1 loscil 30000, 1, 1, 1
  ; Use the "beats.wav" audio file.
  ain2 soundin "beats.wav"

  isize = 4096
  ioverlap = 2
  iwin = 2
  kbias init 1

  aout cross2 ain1, ain2, isize, ioverlap, iwin, kbias

  out aout
endin
/* cross2.orc */
        
\end{lstlisting}
\begin{lstlisting}
/* cross2.sco */
; Table #1: An audio file.
f 1 0 128 1 "ahh.aiff" 0 4 0
; Table #2: A windowing function.
f 2 0 2048 20 2

; Play Instrument #1 for 2 seconds.
i 1 0 2
; Play Instrument #2 for 2 seconds.
i 2 2 2
e
/* cross2.sco */
        
\end{lstlisting}
\subsection*{Credits}


 


 


\begin{tabular}{ccc}
Author: Paris Smaragdis &MIT, Cambridge &1997

\end{tabular}



 


 Example written by Kevin Conder.
%\hline 


\begin{comment}
\begin{tabular}{lcr}
Previous &Home &Next \\
cpuprc &Up &crunch

\end{tabular}


\end{document}
\end{comment}
