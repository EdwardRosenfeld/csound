\begin{comment}
\documentclass[10pt]{article}
\usepackage{fullpage, graphicx, url}
\setlength{\parskip}{1ex}
\setlength{\parindent}{0ex}
\title{jspline}
\begin{document}


\begin{tabular}{ccc}
The Alternative Csound Reference Manual & & \\
Previous & &Next

\end{tabular}

%\hline 
\end{comment}
\section{jspline}
jspline�--� A jitter-spline generator. \subsection*{Description}


  A jitter-spline generator. 
\subsection*{Syntax}


 ar \textbf{jspline}
 xamp, kcpsMin, kcpsMax


 kr \textbf{jspline}
 kamp, kcpsMin, kcpsMax
\subsection*{Performance}


 \emph{kr, ar}
 -- Output signal 


 \emph{xamp}
 -- Amplitude factor 


 \emph{kcpsMin, kcpsMax}
 -- Range of point-generation rate. Min and max limits are expressed in cps. 


 \emph{jspline}
 (jitter-spline generator) generates a smooth curve based on random points generated at [cpsMin, cpsMax] rate. This opcode is similar to randomi or randi or jitter, but segments are not straight lines, but cubic spline curves. Output value range is approximately $>$ -xamp and $<$ xamp. Actually, real range could be a bit greater, because of interpolating curves beetween each pair of random-points. 


  At present time generated curves are quite smooth when cpsMin is not too different from cpsMax. When cpsMin-cpsMax interval is big, some little discontinuity could occurr, but it should not be a problem, in most cases. Maybe the algorithm will be improved in next versions. 


  These opcodes are often better than \emph{jitter}
 when user wants to ``naturalize'' or ``analogize'' digital sounds. They could be used also in algorithmic composition, to generate smooth random melodic lines when used together with \emph{samphold}
 opcode. 


  Note that the result is quite different from the one obtained by filtering white noise, and they allow the user to obtain a much more precise control. 
\subsection*{Credits}


 Author: Gabriel Maldonado


 New in Version 4.15
%\hline 


\begin{comment}
\begin{tabular}{lcr}
Previous &Home &Next \\
jitter2 &Up &kbetarand

\end{tabular}


\end{document}
\end{comment}
