\begin{comment}
\documentclass[10pt]{article}
\usepackage{fullpage, graphicx, url}
\setlength{\parskip}{1ex}
\setlength{\parindent}{0ex}
\title{ftgen}
\begin{document}


\begin{tabular}{ccc}
The Alternative Csound Reference Manual & & \\
Previous & &Next

\end{tabular}

%\hline 
\end{comment}
\section{ftgen}
ftgen�--� Generate a score function table from within the orchestra. \subsection*{Description}


 Generate a score function table from within the orchestra.
\subsection*{Syntax}


 gir \textbf{ftgen}
 ifn, itime, isize, igen, iarga [, iargb ] [...]
\subsection*{Initialization}


 \emph{gir}
 -- either a requested or automatically assigned table number above 100. 


 \emph{ifn}
 -- requested table number If \emph{ifn}
 is zero, the number is assigned automatically and the value placed in \emph{gir}
. Any other value is used as the table number 


 \emph{itime}
 -- is ignored, but otherwise corresponds to p2 in the score \emph{f statement}
. 


 \emph{isize}
 -- table size. Corresponds to p3 of the score \emph{f statement}
. 


 \emph{igen}
 -- function table \emph{GEN}
 routine. Corresponds to p4 of the score \emph{f statement}
. 


 \emph{iarga, iargb, ...}
 -- function table arguments. Correspond to p5 through p\emph{n}
 of the score \emph{f statement}
. 
\subsection*{Performance}


  This is equivalent to table generation in the score with the \emph{f statement}
. 


 


\begin{tabular}{cc}
Warning &\textbf{Warning}
 \\
� &

  Although Csound will not protest if ftgen is used inside instr-endin statements, this is not the intended or supported use, and must be handled with care as it has global effects. (In particular, a different size usually leads to relocation of the table, which may cause a crash or otherwise erratic behaviour. 


\end{tabular}

\subsection*{Examples}


  Here is an example of the ftgen opcode. It uses the files \emph{ftgen.orc}
 and \emph{ftgen.sco}
. 


 \textbf{Example 1. Example of the ftgen opcode.}

\begin{lstlisting}
/* ftgen.orc */
; Initialize the global variables.
sr = 44100
kr = 4410
ksmps = 10
nchnls = 1

; Table #1, a sine wave using the GEN10 routine.
gitemp ftgen 1, 0, 16384, 10, 1

; Instrument #1 - a basic oscillator.
instr 1
  kamp = 10000
  kcps = 440
  ; Use Table #1.
  ifn = 1

  a1 oscil kamp, kcps, ifn
  out a1
endin
/* ftgen.orc */
        
\end{lstlisting}
\begin{lstlisting}
/* ftgen.sco */
; Play Instrument #1 for 2 seconds.
i 1 0 2
e
/* ftgen.sco */
        
\end{lstlisting}
\subsection*{Credits}


 


 


\begin{tabular}{ccc}
Author: Barry L. Vercoe &M.I.T., Cambridge, Mass &1997

\end{tabular}



 


 Example written by Kevin Conder.


 Added warning April 2002 by Rasmus Ekman
%\hline 


\begin{comment}
\begin{tabular}{lcr}
Previous &Home &Next \\
ftchnls &Up &ftlen

\end{tabular}


\end{document}
\end{comment}
