\begin{comment}
\documentclass[10pt]{article}
\usepackage{fullpage, graphicx, url}
\setlength{\parskip}{1ex}
\setlength{\parindent}{0ex}
\title{reinit}
\begin{document}


\begin{tabular}{ccc}
The Alternative Csound Reference Manual & & \\
Previous & &Next

\end{tabular}

%\hline 
\end{comment}
\section{reinit}
reinit�--� Suspends a performance while a special initialization pass is executed. \subsection*{Description}


  Suspends a performance while a special initialization pass is executed. 


  Whenever this statement is encountered during a p-time pass, performance is temporarily suspended while a special Initialization pass, beginning at \emph{label}
 and continuing to \emph{rireturn}
 or \emph{endin}
, is executed. Performance will then be resumed from where it left off. 
\subsection*{Syntax}


 \textbf{reinit}
 label
\subsection*{Examples}


  The following statements will generate an exponential control signal whose value moves from 440 to 880 exactly ten times over the duration p3. They use the files \emph{reinit.orc}
 and \emph{reinit.sco}
. 


 \textbf{Example 1. Example of the reinit opcode.}

\begin{lstlisting}
/* reinit.orc */
sr = 44100
kr = 44100
ksmps = 1
nchnls = 1

instr 1

reset:
        timout 0, p3/10, contin
        reinit reset

contin:
        kLine expon 440, p3/10, 880
        aSig oscil 10000, kLine, 1
        out aSig
        rireturn

endin
/* reinit.orc */
        
\end{lstlisting}
\begin{lstlisting}
/* reinit.sco */
f1 0 4096 10 1

i1 0 10
e
/* reinit.sco */
        
\end{lstlisting}
\subsection*{See Also}


 \emph{rigoto}
, \emph{rireturn}

%\hline 


\begin{comment}
\begin{tabular}{lcr}
Previous &Home &Next \\
readk4 &Up &release

\end{tabular}


\end{document}
\end{comment}
