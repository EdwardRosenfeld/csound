\begin{comment}
\documentclass[10pt]{article}
\usepackage{fullpage, graphicx, url}
\setlength{\parskip}{1ex}
\setlength{\parindent}{0ex}
\title{oscils}
\begin{document}


\begin{tabular}{ccc}
The Alternative Csound Reference Manual & & \\
Previous & &Next

\end{tabular}

%\hline 
\end{comment}
\section{oscils}
oscils�--� A simple, fast sine oscillator \subsection*{Description}


  Simple, fast sine oscillator, that uses only one multiply, and two add operations to generate one sample of output, and does not require a function table. 
\subsection*{Syntax}


 ar \textbf{oscils}
 iamp, icps, iphs [, iflg]
\subsection*{Initialization}


 \emph{iamp}
 -- output amplitude. 


 \emph{icps}
 -- frequency in Hz (may be zero or negative, however the absolute value must be less than sr/2). 


 \emph{iphs}
 -- start phase between 0 and 1. 


 \emph{iflg}
 -- sum of the following values: 


 
\begin{itemize}
\item 

 \emph{2}
: use double precision even if Csound was compiled to use floats. This improves quality (especially in the case of long performance time), but may be up to twice as slow. 

\item 

 \emph{1}
: skip initialization. 


\end{itemize}
\subsection*{Performance}


 \emph{ar}
 -- audio output 
\subsection*{Examples}


  Here is an example of the oscils opcode. It uses the files \emph{oscils.orc}
 and \emph{oscils.sco}
. 


 \textbf{Example 1. Example of the oscils opcode.}

\begin{lstlisting}
/* oscils.orc */
; Initialize the global variables.
sr = 44100
kr = 4410
ksmps = 10
nchnls = 1

; Instrument #1 - a fast sine oscillator.
instr 1
  iamp = 10000
  icps = 440
  iphs = 0

  a1 oscils iamp, icps, iphs
  out a1
endin
/* oscils.orc */
        
\end{lstlisting}
\begin{lstlisting}
/* oscils.sco */
; Play Instrument #1 for 2 seconds.
i 1 0 2
e
/* oscils.sco */
        
\end{lstlisting}
\subsection*{Credits}


 


 


\begin{tabular}{cc}
Author: Istvan Varga &January 2002

\end{tabular}



 


 Example written by Kevin Conder.


 New in version 4.18
%\hline 


\begin{comment}
\begin{tabular}{lcr}
Previous &Home &Next \\
osciln &Up &oscilx

\end{tabular}


\end{document}
\end{comment}
