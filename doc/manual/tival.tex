\begin{comment}
\documentclass[10pt]{article}
\usepackage{fullpage, graphicx, url}
\setlength{\parskip}{1ex}
\setlength{\parindent}{0ex}
\title{tival}
\begin{document}


\begin{tabular}{ccc}
The Alternative Csound Reference Manual & & \\
Previous & &Next

\end{tabular}

%\hline 
\end{comment}
\section{tival}
tival�--� Puts the value of the instrument's internal ``tie-in'' flag into the named i-rate variable. \subsection*{Syntax}


 ir \textbf{tival}

\subsection*{Description}


  Puts the value of the instrument's internal ``tie-in'' flag into the named i-rate variable. 
\subsection*{Initialization}


  Puts the value of the instrument's internal ``tie-in'' flag into the named i-rate variable. Assigns 1 if this note has been ``tied'' onto a previously held note (see \emph{i statement}
); assigns 0 if no tie actually took place. (See also \emph{tigoto}
.) 
\subsection*{See Also}


 \emph{=}
, \emph{divz}
, \emph{init}

%\hline 


\begin{comment}
\begin{tabular}{lcr}
Previous &Home &Next \\
timout &Up &tlineto

\end{tabular}


\end{document}
\end{comment}
