\begin{comment}
\documentclass[10pt]{article}
\usepackage{fullpage, graphicx, url}
\setlength{\parskip}{1ex}
\setlength{\parindent}{0ex}
\title{sensekey}
\begin{document}


\begin{tabular}{ccc}
The Alternative Csound Reference Manual & & \\
Previous & &Next

\end{tabular}

%\hline 
\end{comment}
\section{sensekey}
sensekey�--� Returns the ASCII code of a key that has been pressed. \subsection*{Description}


  Returns the ASCII code of a key that has been pressed, or -1 if no key has been pressed. 
\subsection*{Syntax}


 kr \textbf{sensekey}

\subsection*{Performance}


  At release, this has not been properly verified, and seems not to work at all on Windows. 


 


\begin{tabular}{cc}
\textbf{Note}
 \\
� &

  This opcode can also be written as \emph{sense}
. 


\end{tabular}

\subsection*{Examples}


  Here is an example of the sensekey opcode. It uses the files \emph{sensekey.orc}
 and \emph{sensekey.sco}
. 


 \textbf{Example 1. Example of the sensekey opcode.}

\begin{lstlisting}
/* sensekey.orc */
; Initialize the global variables.
sr = 44100
kr = 4410
ksmps = 10
nchnls = 1

; Instrument #1.
instr 1
  k1 sensekey
  printk2 k1
endin
/* sensekey.orc */
        
\end{lstlisting}
\begin{lstlisting}
/* sensekey.sco */
; Play Instrument #1 for thirty seconds.
i 1 0 30
e
/* sensekey.sco */
        
\end{lstlisting}
 Here is what the output should look like when the ``q'' button is pressed... \begin{lstlisting}
q i1 357967744.00000
      
\end{lstlisting}
\subsection*{Credits}


 


 


\begin{tabular}{cccc}
Author: John ffitch &University of Bath, Codemist. Ltd. &Bath, UK &October 2000

\end{tabular}



 


 Example written by Kevin Conder.


 New in Csound version 4.09. Renamed in Csound version 4.10.
%\hline 


\begin{comment}
\begin{tabular}{lcr}
Previous &Home &Next \\
sense &Up &seqtime

\end{tabular}


\end{document}
\end{comment}
