\begin{comment}
\documentclass[10pt]{article}
\usepackage{fullpage, graphicx, url}
\setlength{\parskip}{1ex}
\setlength{\parindent}{0ex}
\title{jitter}
\begin{document}


\begin{tabular}{ccc}
The Alternative Csound Reference Manual & & \\
Previous & &Next

\end{tabular}

%\hline 
\end{comment}
\section{jitter}
jitter�--� Generates a segmented line whose segments are randomly generated. \subsection*{Description}


  Generates a segmented line whose segments are randomly generated. 
\subsection*{Syntax}


 kout \textbf{jitter}
 kamp, kcpsMin, kcpsMax
\subsection*{Performance}


 \emph{kamp}
 -- Amplitude of jitter deviation 


 \emph{kcpsMin}
 -- Minimum speed of random frequency variations (expressed in cps) 


 \emph{kcpsMax}
 -- Maximum speed of random frequency variations (expressed in cps) 


 \emph{jitter}
 generates a segmented line whose segments are randomly generated inside the +kamp and -kamp interval. Duration of each segment is a random value generated according to kcpsmin and kcpsmax values. 


 \emph{jitter}
 can be used to make more natural and ``analog-sounding'' some static, dull sound. For best results, it is suggested to keep its amplitude moderate. 
\subsection*{Examples}


  Here is an example of the jitter opcode. It uses the files \emph{jitter.orc}
 and \emph{jitter.sco}
. 


 \textbf{Example 1. Example of the jitter opcode.}

\begin{lstlisting}
/* jitter.orc */
; Initialize the global variables.
sr = 44100
kr = 4410
ksmps = 10
nchnls = 2

; Instrument #1 -- plain instrument.
instr 1
  aplain vco 20000, 220, 2, 0.83

  outs aplain, aplain
endin

; Instrument #2 -- instrument with jitter.
instr 2
  ; Create a signal modulated the jitter opcode.
  kamp init 2
  kcpsmin init 4
  kcpsmax init 6
  kj jitter kamp, kcpsmin, kcpsmax

  aplain vco 20000, 220, 2, 0.83
  ajitter vco 20000, 220+kj, 2, 0.83

  outs aplain, ajitter
endin
/* jitter.orc */
        
\end{lstlisting}
\begin{lstlisting}
/* jitter.sco */
; Table #1, a sine wave.
f 1 0 16384 10 1

; Play Instrument #1 for 3 seconds.
i 1 0 3
; Play Instrument #2 for 3 seconds.
i 2 3 3
e
/* jitter.sco */
        
\end{lstlisting}
\subsection*{See Also}


 \emph{jitter2}
, \emph{vibr}
, \emph{vibrato}

\subsection*{Credits}


 Author: Gabriel Maldonado


 Example written by Kevin Conder.


 New in Version 4.15
%\hline 


\begin{comment}
\begin{tabular}{lcr}
Previous &Home &Next \\
iweibull &Up &jitter2

\end{tabular}


\end{document}
\end{comment}
