\begin{comment}
\documentclass[10pt]{article}
\usepackage{fullpage, graphicx, url}
\setlength{\parskip}{1ex}
\setlength{\parindent}{0ex}
\title{printk}
\begin{document}


\begin{tabular}{ccc}
The Alternative Csound Reference Manual & & \\
Previous & &Next

\end{tabular}

%\hline 
\end{comment}
\section{printk}
printk�--� Prints one k-rate value at specified intervals. \subsection*{Description}


  Prints one k-rate value at specified intervals. 
\subsection*{Syntax}


 \textbf{printk}
 itime, kval [, ispace]
\subsection*{Initialization}


 \emph{itime}
 -- time in seconds between printings. 


 \emph{ispace}
 (optional, default=0) -- number of spaces to insert before printing. (default: 0, max: 130) 
\subsection*{Performance}


 \emph{kval}
 -- The k-rate values to be printed. 


 \emph{printk}
 prints one k-rate value on every k-cycle, every second or at intervals specified. First the instrument number is printed, then the absolute time in seconds, then a specified number of spaces, then the \emph{kval}
 value. The variable number of spaces enables different values to be spaced out across the screen - so they are easier to view. 


  This opcode can be run on every k-cycle it is run in the instrument. To every accomplish this, set \emph{itime}
 to 0. 


  When \emph{itime}
 is not 0, the opcode print on the first k-cycle it is called, and subsequently when every \emph{itime}
 period has elapsed. The time cycles start from the time the opcode is initialized - typically the initialization of the instrument. 
\subsection*{Examples}


  Here is an example of the printk opcode. It uses the files \emph{printk.orc}
 and \emph{printk.sco}
. 


 \textbf{Example 1. Example of the printk opcode.}

\begin{lstlisting}
/* printk.orc */
; Initialize the global variables.
sr = 44100
kr = 44100
ksmps = 1
nchnls = 1

; Instrument #1.
instr 1
  ; Change a value linearly from 0 to 100,
  ; over the period defined by p3.
  kval line 0, p3, 100

  ; Print the value of kval, once per second.
  printk 1, kval
endin
/* printk.orc */
        
\end{lstlisting}
\begin{lstlisting}
/* printk.sco */
; Play Instrument #1 for 5 seconds.
i 1 0 5
e
/* printk.sco */
        
\end{lstlisting}
 Its output should include lines like this: \begin{lstlisting}
 i   1 time     0.00002:     0.00000
 i   1 time     1.00002:    20.01084
 i   1 time     2.00002:    40.02999
 i   1 time     3.00002:    60.04914
 i   1 time     4.00002:    79.93327
      
\end{lstlisting}
\subsection*{See Also}


 \emph{printk2}
 and \emph{printks}

\subsection*{Credits}


 


 


\begin{tabular}{ccc}
Author: Robin Whittle &Australia &May 1997

\end{tabular}



 


 Example written by Kevin Conder.


 Thanks goes to Luis Jure for pointing out a mistake wit the \emph{itime}
 parameter.
%\hline 


\begin{comment}
\begin{tabular}{lcr}
Previous &Home &Next \\
print &Up &printk2

\end{tabular}


\end{document}
\end{comment}
