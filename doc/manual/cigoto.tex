\begin{comment}
\documentclass[10pt]{article}
\usepackage{fullpage, graphicx, url}
\setlength{\parskip}{1ex}
\setlength{\parindent}{0ex}
\title{cigoto}
\begin{document}


\begin{tabular}{ccc}
The Alternative Csound Reference Manual & & \\
Previous & &Next

\end{tabular}

%\hline 
\end{comment}
\section{cigoto}
cigoto�--� Conditionally transfer control during the i-time pass. \subsection*{Description}


  During the i-time pass only, unconditionally transfer control to the statement labeled by \emph{label}
. 
\subsection*{Syntax}


 \textbf{cigoto}
 condition, label


  where \emph{label}
 is in the same instrument block and is not an expression, and where \emph{R}
 is one of the Relational operators (\emph{$<$}
,\emph{ =}
, \emph{$<$=}
, \emph{==}
, \emph{!=}
) (and \emph{=}
 for convenience, see also under \emph{Conditional Values}
). 
\subsection*{Examples}


  Here is an example of the cigoto opcode. It uses the files \emph{cigoto.orc}
 and \emph{cigoto.sco}
. 


 \textbf{Example 1. Example of the cigoto opcode.}

\begin{lstlisting}
/* cigoto.orc */
; Initialize the global variables.
sr = 44100
kr = 4410
ksmps = 10
nchnls = 1

; Instrument #1.
instr 1
  ; Get the value of the 4th p-field from the score.
  iparam = p4

  ; If iparam is 1 then play the high note.
  ; If not then play the low note.
  cigoto (iparam ==1), highnote
    igoto lownote

highnote:
  ifreq = 880
  goto playit

lownote:
  ifreq = 440
  goto playit

playit:
  ; Print the values of iparam and ifreq.
  print iparam
  print ifreq

  a1 oscil 10000, ifreq, 1
  out a1
endin
/* cigoto.orc */
        
\end{lstlisting}
\begin{lstlisting}
/* cigoto.sco */
; Table #1: a simple sine wave.
f 1 0 32768 10 1

; p4: 1 = high note, anything else = low note
; Play Instrument #1 for one second, a low note.
i 1 0 1 0
; Play a Instrument #1 for one second, a high note.
i 1 1 1 1
e
/* cigoto.sco */
        
\end{lstlisting}
 Its output should include lines like: \begin{lstlisting}
instr 1:  iparam = 0.000
instr 1:  ifreq = 440.000
instr 1:  iparam = 1.000
instr 1:  ifreq = 880.000
      
\end{lstlisting}
\subsection*{See Also}


 \emph{cggoto}
, \emph{ckgoto}
, \emph{cngoto}
, \emph{goto}
, \emph{if}
, \emph{kgoto}
, \emph{rigoto}
, \emph{tigoto}
, \emph{timout}

\subsection*{Credits}


 Added a note by Jim Aikin.


 Example written by Kevin Conder.
%\hline 


\begin{comment}
\begin{tabular}{lcr}
Previous &Home &Next \\
checkbox &Up &ckgoto

\end{tabular}


\end{document}
\end{comment}
