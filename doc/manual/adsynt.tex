\begin{comment}
\documentclass[10pt]{article}
\usepackage{fullpage, graphicx, url}
\setlength{\parskip}{1ex}
\setlength{\parindent}{0ex}
\title{adsynt}
\begin{document}


\begin{tabular}{ccc}
The Alternative Csound Reference Manual & & \\
Previous & &Next

\end{tabular}

%\hline 
\end{comment}
\section{adsynt}
adsynt�--� Performs additive synthesis with an arbitrary number of partials, not necessarily harmonic. \subsection*{Description}


  Performs additive synthesis with an arbitrary number of partials, not necessarily harmonic. 
\subsection*{Syntax}


 ar \textbf{adsynt}
 kamp, kcps, iwfn, ifreqfn, iampfn, icnt [, iphs]
\subsection*{Initialization}


 \emph{iwfn}
 -- table containing a waveform, usually a sine. Table values are not interpolated for performance reasons, so larger tables provide better quality. 


 \emph{ifreqfn}
 -- table containing frequency values for each partial. \emph{ifreqfn}
 may contain beginning frequency values for each partial, but is usually used for generating parameters at runtime with \emph{tablew}
. Frequencies must be relative to \emph{kcps}
. Size must be at least \emph{icnt}
. 


 \emph{iampfn}
 -- table containing amplitude values for each partial. \emph{iampfn}
 may contain beginning amplitude values for each partial, but is usually used for generating parameters at runtime with \emph{tablew}
. Amplitudes must be relative to \emph{kamp}
. Size must be at least \emph{icnt}
. 


 \emph{icnt}
 -- number of partials to be generated 


 \emph{iphs}
 -- initial phase of each oscillator, if \emph{iphs}
 = -1, initialization is skipped. If \emph{iphs}
 $>$ 1, all phases will be initialized with a random value. 
\subsection*{Performance}


 \emph{kamp}
 -- amplitude of note 


 \emph{kcps}
 -- base frequency of note. Partial frequencies will be relative to \emph{kcps}
. 


  Frequency and amplitude of each partial is given in the two tables provided. The purpose of this opcode is to have an instrument generate synthesis parameters at k-rate and write them to global parameter tables with the \emph{tablew}
 opcode. 
\subsection*{Examples}


  Here is an example of the adsynt opcode. It uses the files \emph{adsynt.orc}
 and \emph{adsynt.sco}
. These two instruments perform additive synthesis. The output of each sounds like a Tibetan bowl. The first one is static, as parameters are only generated at init-time. In the second one, parameters are continuously changed. 


 \textbf{Example 1. Example of the adsynt opcode.}

\begin{lstlisting}
/* adsynt.orc */
; Initialize the global variables.
sr = 44100
kr = 4410
ksmps = 10
nchnls = 1

; Generate a sinewave table.
giwave ftgen 1, 0, 1024, 10, 1
; Generate two empty tables for adsynt.
gifrqs ftgen 2, 0, 32, 7, 0, 32, 0
; A table for freqency and amp parameters.
giamps ftgen 3, 0, 32, 7, 0, 32, 0
  
; Generates parameters at init time
instr 1
  ; Generate 10 voices.
  icnt = 10 
  ; Init loop index.
  index = 0 

; Loop only executed at init time.
loop: 
  ; Define non-harmonic partials.
  ifreq pow index + 1, 1.5 
  ; Define amplitudes.
  iamp = 1 / (index+1) 
  ; Write to tables.
  tableiw ifreq, index, gifrqs 
  ; Used by adsynt.
  tableiw iamp, index, giamps 
  
  index = index + 1
  ; Do loop/
  if (index < icnt) igoto loop 
  
  asig adsynt 5000, 150, giwave, gifrqs, giamps, icnt
  out asig
endin

; Generates parameters every k-cycle.
instr 2 
  ; Generate 10 voices.
  icnt = 10 
  ; Reset loop index.
  kindex = 0

; Loop executed every k-cycle.
loop:
  ; Generate lfo for frequencies.
  kspeed  pow kindex + 1, 1.6
  ; Individual phase for each voice.
  kphas phasorbnk kspeed * 0.7, kindex, icnt
  klfo table kphas, giwave, 1
  ; Arbitrary parameter twiddling...
  kdepth pow 1.4, kindex
  kfreq pow kindex + 1, 1.5
  kfreq = kfreq + klfo*0.006*kdepth

  ; Write freqs to table for adsynt.
  tablew kfreq, kindex, gifrqs 
  
  ; Generate lfo for amplitudes.
  kspeed  pow kindex + 1, 0.8
  ; Individual phase for each voice.
  kphas phasorbnk kspeed*0.13, kindex, icnt, 2
  klfo table kphas, giwave, 1
  ; Arbitrary parameter twiddling...
  kamp pow 1 / (kindex + 1), 0.4
  kamp = kamp * (0.3+0.35*(klfo+1))

  ; Write amps to table for adsynt.
  tablew kamp, kindex, giamps
  
  kindex = kindex + 1
  ; Do loop.
  if (kindex < icnt) kgoto loop

  asig adsynt 5000, 150, giwave, gifrqs, giamps, icnt
  out asig
endin
/* adsynt.orc */
        
\end{lstlisting}
\begin{lstlisting}
/* adsynt.sco */
; Play Instrument #1 for 2.5 seconds.
i 1 0 2.5
; Play Instrument #2 for 2.5 seconds.
i 2 3 2.5
e
/* adsynt.sco */
        
\end{lstlisting}
\subsection*{Credits}


 


 


\begin{tabular}{ccc}
Author: Peter Neub\"acker &Munich, Germany &August, 1999

\end{tabular}



 


 New in Csound version 3.58
%\hline 


\begin{comment}
\begin{tabular}{lcr}
Previous &Home &Next \\
adsyn &Up &aexprand

\end{tabular}


\end{document}
\end{comment}
