\begin{comment}
\documentclass[10pt]{article}
\usepackage{fullpage, graphicx, url}
\setlength{\parskip}{1ex}
\setlength{\parindent}{0ex}
\title{zkr}
\begin{document}


\begin{tabular}{ccc}
The Alternative Csound Reference Manual & & \\
Previous & &Next

\end{tabular}

%\hline 
\end{comment}
\section{zkr}
zkr�--� Reads from a location in zk space at k-rate. \subsection*{Description}


  Reads from a location in zk space at k-rate. 
\subsection*{Syntax}


 kr \textbf{zkr}
 kndx
\subsection*{Initialization}


 \emph{kndx}
 -- points to the zk location to be read. 
\subsection*{Performance}


 \emph{zkr}
 reads the array of floats at \emph{kndx}
 in zk space. 
\subsection*{Examples}


  Here is an example of the zkr opcode. It uses the files \emph{zkr.orc}
 and \emph{zkr.sco}
. 


 \textbf{Example 1. Example of the zkr opcode.}

\begin{lstlisting}
/* zkr.orc */
; Initialize the global variables.
sr = 44100
kr = 4410
ksmps = 10
nchnls = 1

; Initialize the ZAK space.
; Create 1 a-rate variable and 1 k-rate variable.
zakinit 1, 1

; Instrument #1 -- a simple waveform.
instr 1
  ; Linearly vary a k-rate signal from 440 to 880.
  kline line 440, p3, 880

  ; Add the linear signal to zk variable #1.
  zkw kline, 1
endin

; Instrument #2 -- generates audio output.
instr 2
  ; Read zk variable #1.
  kfreq zkr 1

  ; Use the value of zk variable #1 to vary 
  ; the frequency of a sine waveform.
  a1 oscil 20000, kfreq, 1

  ; Generate the audio output.
  out a1

  ; Clear the zk variables, get them ready for 
  ; another pass.
  zkcl 0, 1
endin
/* zkr.orc */
        
\end{lstlisting}
\begin{lstlisting}
/* zkr.sco */
; Table #1, a sine wave.
f 1 0 16384 10 1

; Play Instrument #1 for one second.
i 1 0 1
; Play Instrument #2 for one second.
i 2 0 1
e
/* zkr.sco */
        
\end{lstlisting}
\subsection*{See Also}


 \emph{zar}
, \emph{zarg}
, \emph{zir}
, \emph{zkcl}
, \emph{zkmod}
, \emph{zkwm}
, \emph{zkw}

\subsection*{Credits}


 


 


\begin{tabular}{ccc}
Author: Robin Whittle &Australia &May 1997

\end{tabular}



 


 Example written by Kevin Conder.
%\hline 


\begin{comment}
\begin{tabular}{lcr}
Previous &Home &Next \\
zkmod &Up &zkw

\end{tabular}


\end{document}
\end{comment}
