\begin{comment}
\documentclass[10pt]{article}
\usepackage{fullpage, graphicx, url}
\setlength{\parskip}{1ex}
\setlength{\parindent}{0ex}
\title{ftchnls}
\begin{document}


\begin{tabular}{ccc}
The Alternative Csound Reference Manual & & \\
Previous & &Next

\end{tabular}

%\hline 
\end{comment}
\section{ftchnls}
ftchnls�--� Returns the number of channels in a stored function table. \subsection*{Description}


  Returns the number of channels in a stored function table. 
\subsection*{Syntax}


 \textbf{ftchnls}
(x) (init-rate args only)
\subsection*{Performance}


  Returns the number of channels of a \emph{GEN01}
 table, determined from the header of the original file. If the original file has no header or the table was not created by these GEN01, \emph{ftchnls}
 returns -1. 
\subsection*{Examples}


  Here is an example of the ftchnls opcode. It uses the files \emph{ftchnls.orc}
, \emph{ftchnls.sco}
, and \emph{mary.wav}
. 


 \textbf{Example 1. Example of the ftchnls opcode.}

\begin{lstlisting}
/* ftchnls.orc */
; Initialize the global variables.
sr = 44100
kr = 4410
ksmps = 10
nchnls = 1

; Instrument #1.
instr 1
  ; Print out the number of channels in Table #1.
  ichnls = ftchnls(1)
  print ichnls
endin
/* ftchnls.orc */
        
\end{lstlisting}
\begin{lstlisting}
/* ftchnls.sco */
; Table #1: Use an audio file, Csound will determine its size.
f 1 0 0 1 "mary.wav" 0 0 0

; Play Instrument #1 for 1 second.
i 1 0 1
e
/* ftchnls.sco */
        
\end{lstlisting}
 Since the audio file ``mary.wav'' is monophonic (1 channel), its output should include a line like this: \begin{lstlisting}
instr 1:  ichnls = 1.000
      
\end{lstlisting}
\subsection*{See Also}


 \emph{ftlen}
, \emph{ftlptim}
, \emph{ftsr}
, \emph{nsamp}

\subsection*{Credits}


 


 


\begin{tabular}{ccc}
Author: Chris McCormick &Perth, Australia &December 2001

\end{tabular}



 


 Example written by Kevin Conder.
%\hline 


\begin{comment}
\begin{tabular}{lcr}
Previous &Home &Next \\
frac &Up &ftgen

\end{tabular}


\end{document}
\end{comment}
