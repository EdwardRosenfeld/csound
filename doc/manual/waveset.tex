\begin{comment}
\documentclass[10pt]{article}
\usepackage{fullpage, graphicx, url}
\setlength{\parskip}{1ex}
\setlength{\parindent}{0ex}
\title{waveset}
\begin{document}


\begin{tabular}{ccc}
The Alternative Csound Reference Manual & & \\
Previous & &Next

\end{tabular}

%\hline 
\end{comment}
\section{waveset}
waveset�--� A simple time stretch by repeating cycles. \subsection*{Description}


  A simple time stretch by repeating cycles. 
\subsection*{Syntax}


 ar \textbf{waveset}
 ain, krep [, ilen]
\subsection*{Initialization}


 \emph{ilen}
 (optional, default=0) -- the length (in samples) of the audio signal. If \emph{ilen}
 is set to 0, it defaults to half the given note length (p3). 
\subsection*{Performance}


 \emph{ain}
 -- the input audio signal. 


 \emph{krep}
 -- the number of times the cycle is repeated. 


  The input is read and each complete cycle (two zero-crossings) is repeated krep times. 


  There is an internal buffer as the output is clearly slower that the input. Some care is taken if the buffer is too short, but there may be strange effects. 
\subsection*{Examples}


  Here is an example of the waveset opcode. It uses the files \emph{waveset.orc}
, \emph{waveset.sco}
, and \emph{beats.wav}
. 


 \textbf{Example 1. Example of the waveset opcode.}

\begin{lstlisting}
/* waveset.orc */
; Initialize the global variables.
sr = 44100
kr = 4410
ksmps = 10
nchnls = 1

; Instrument #1 - play an audio file.
instr 1
  asig soundin "beats.wav"
  out asig
endin


; Instrument #2 - stretch the audio file with waveset.
instr 2
  asig soundin "beats.wav"
  a1 waveset asig, 2

  out a1
endin
/* waveset.orc */
        
\end{lstlisting}
\begin{lstlisting}
/* waveset.sco */
; Play Instrument #1 for two seconds.
i 1 0 2
; Play Instrument #2 for four seconds.
i 2 3 4
e
/* waveset.sco */
        
\end{lstlisting}
\subsection*{Credits}


 


 


\begin{tabular}{cc}
Author: John ffitch &February 2001

\end{tabular}



 


 Example written by Kevin Conder.


 New in version 4.11
%\hline 


\begin{comment}
\begin{tabular}{lcr}
Previous &Home &Next \\
vpvoc &Up &weibull

\end{tabular}


\end{document}
\end{comment}
