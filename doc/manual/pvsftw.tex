\begin{comment}
\documentclass[10pt]{article}
\usepackage{fullpage, graphicx, url}
\setlength{\parskip}{1ex}
\setlength{\parindent}{0ex}
\title{pvsftw}
\begin{document}


\begin{tabular}{ccc}
The Alternative Csound Reference Manual & & \\
Previous & &Next

\end{tabular}

%\hline 
\end{comment}
\section{pvsftw}
pvsftw�--� Writes amplitude and/or frequency data to function tables. \subsection*{Description}


  Writes amplitude and/or frequency data to function tables. 
\subsection*{Syntax}


 kflag \textbf{pvsftw}
 fsrc, ifna [, ifnf]
\subsection*{Initialization}


 \emph{ifna}
 -- A table, at least inbins in size, that stores amplitude data. Ignored if ifna = 0 


 \emph{ifnf}
 -- A table, at least inbins in size, that stores frequency data. Ignored if ifnf = 0 
\subsection*{Performance}


 \emph{kflag}
 -- A flag that has the value of 1 when new data is available, 0 otherwise. 


 \emph{fsrc}
 -- a PVOC-EX formatted source. 


  Enables the contents of \emph{fsrc}
 to be exchanged with function tables, for custom processing. Except when the frame overlap equals \emph{ksmps}
 (which will generally not be the case), the frame data is not updated each control period. The data in \emph{ifna}
, \emph{ifnf}
 should only be processed when \emph{kflag}
 is set to 1. To process only frequency data, set \emph{ifna}
 to zero. 


  As the functions tables are required only to store data from \emph{fsrc}
, there is no advantage in defining then in the score. They should generally be created in the instrument using \emph{ftgen}
. 


  By exporting amplitude data, say, from one fsig and importing it into another, basic cross-synthesis (as in \emph{pvscross}
) can be performed, with the option to modify the data beforehand using the table manipulation opodes. 


  Note that the format data in the source fsig is not written to the tables. This therefore offers a means of transferring amplitude and frequency data between non-identical fsigs. Used this way, these opcodes become potentially pathological, and can be relied upon to produce unexpected results. In such cases, resynthesis using \emph{pvsadsyn}
 would almost certainly be required. 


  To perform a straight copy from one fsig to another one of identical format, the conventional assignment syntax can be used: 


 �\\ 
 fsig1�=�fsig2\\ 
 ������
 It is not necessary to use function tables in this case. \subsection*{Examples}


 


 
\begin{lstlisting}
ifn     ftgen      0,0,inbins,10,1        ; make ftable
kflag   pvsftw     fsrc,ifn          ; export  amps to table,
kamp    init       0
if      kflag==0   kgoto contin   ; only proc when frame is ready
; kill lowest bins, for obvious effect
        tablew      kamp,1,ifn
        tablew      kamp,2,ifn
        tablew      kamp,3,ifn
        tablew      kamp,4,ifn
; read modified data back to fsrc
        pvsftr      fsrc,ifn
contin:
; and resynth
aout    pvsynth    fsrc
        
\end{lstlisting}


 
\subsection*{See Also}


 \emph{pvsftr}

\subsection*{Credits}


 


 


\begin{tabular}{cc}
Author: Richard Dobson &August 2001 

\end{tabular}



 


 New in version 4.13
%\hline 


\begin{comment}
\begin{tabular}{lcr}
Previous &Home &Next \\
pvsftr &Up &pvsinfo

\end{tabular}


\end{document}
\end{comment}
