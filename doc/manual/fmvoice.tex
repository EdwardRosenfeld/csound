\begin{comment}
\documentclass[10pt]{article}
\usepackage{fullpage, graphicx, url}
\setlength{\parskip}{1ex}
\setlength{\parindent}{0ex}
\title{fmvoice}
\begin{document}


\begin{tabular}{ccc}
The Alternative Csound Reference Manual & & \\
Previous & &Next

\end{tabular}

%\hline 
\end{comment}
\section{fmvoice}
fmvoice�--� FM Singing Voice Synthesis \subsection*{Description}


  FM Singing Voice Synthesis 
\subsection*{Syntax}


 ar \textbf{fmvoice}
 kamp, kfreq, kvowel, ktilt, kvibamt, kvibrate, ifn1, ifn2, ifn3, ifn4, ivibfn
\subsection*{Initialization}


 \emph{ifn1, ifn2, ifn3,ifn3}
 -- Tables, usually of sinewaves. 
\subsection*{Performance}


 \emph{kamp}
 -- Amplitude of note. 


 \emph{kfreq}
 -- Frequency of note played. 


 \emph{kvowel}
 -- the vowel being sung, in the range 0-64 


 \emph{ktilt}
 -- the spectral tilt of the sound in the range 0 to 99 


 \emph{kvibamt}
 -- Depth of vibrato 


 \emph{kvibrate}
 -- Rate of vibrato 
\subsection*{Examples}


  Here is an example of the fmvoice opcode. It uses the files \emph{fmvoice.orc}
 and \emph{fmvoice.sco}
. 


 \textbf{Example 1. Example of the fmvoice opcode.}

\begin{lstlisting}
/* fmvoice.orc */
; Initialize the global variables.
sr = 44100
kr = 4410
ksmps = 10
nchnls = 1

; Instrument #1.
instr 1
  kamp = 30000
  kfreq = 110
  ; Use the fourth p-field for the vowel.
  kvowel = p4
  ktilt = 0
  kvibamt = 0.005
  kvibrate = 6
  ifn1 = 1
  ifn2 = 1
  ifn3 = 1
  ifn4 = 1
  ivibfn = 1

  a1 fmvoice kamp, kfreq, kvowel, ktilt, kvibamt, kvibrate, ifn1, ifn2, ifn3, ifn4, ivibfn
  out a1
endin
/* fmvoice.orc */
        
\end{lstlisting}
\begin{lstlisting}
/* fmvoice.sco */
; Table #1, a sine wave.
f 1 0 16384 10 1

; p4 = vowel (a value from 0 to 64)
; Play Instrument #1 for one second, vowel=1.
i 1 0 1 1
; Play Instrument #1 for one second, vowel=2.
i 1 1 1 2
; Play Instrument #1 for one second, vowel=3.
i 1 2 1 3
; Play Instrument #1 for one second, vowel=4.
i 1 3 1 4
; Play Instrument #1 for one second, vowel=5.
i 1 4 1 5
e
/* fmvoice.sco */
        
\end{lstlisting}
\subsection*{Credits}


 


 


\begin{tabular}{ccc}
Author: John ffitch (after Perry Cook) &University of Bath, Codemist Ltd. &Bath, UK

\end{tabular}



 


 New in Csound version 3.47
%\hline 


\begin{comment}
\begin{tabular}{lcr}
Previous &Home &Next \\
fmrhode &Up &fmwurlie

\end{tabular}


\end{document}
\end{comment}
