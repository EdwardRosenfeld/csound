\begin{comment}
\documentclass[10pt]{article}
\usepackage{fullpage, graphicx, url}
\setlength{\parskip}{1ex}
\setlength{\parindent}{0ex}
\title{pvscross}
\begin{document}


\begin{tabular}{ccc}
The Alternative Csound Reference Manual & & \\
Previous & &Next

\end{tabular}

%\hline 
\end{comment}
\section{pvscross}
pvscross�--� Performs cross-synthesis between two source fsigs. \subsection*{Description}


  Performs cross-synthesis between two source fsigs. 
\subsection*{Syntax}


 fsig \textbf{pvscross}
 fsrc, fdest, kamp1, kamp2
\subsection*{Performance}


  The operation of this opcode is identical to that of \emph{pvcross}
 (q.v.), except in using \emph{fsig}
s rather than analysis files, and the absence of spectral envelope preservation. The amplitudes from \emph{fsrc}
 are applied to \emph{fdest}
, using scale factors \emph{kamp1}
 and \emph{kamp2}
 respectively. \emph{kamp1}
 and \emph{kamp2}
 must not exceed the range 0 to 1. 


  With this opcode, cross-synthesis can be performed on real-time audio input, by using \emph{pvsanal}
 to generate \emph{fsrc}
 and \emph{fdest}
. These must have the same format. 
\subsection*{Examples}


 


 
\begin{lstlisting}
kcross  linseg    0,p3/3,0,p3/3,1,p3/3,1 ; progressive cross-synthesis
fcross  pvscross  fsig1,fsig2,1-kcross,kcross
across  pvsynth   fcross
        
\end{lstlisting}


 
\subsection*{Credits}


 


 


\begin{tabular}{cc}
Author: Richard Dobson &August 2001 

\end{tabular}



 


 November 2003. Thanks to Kanata Motohashi, fixed the link to the \emph{pvcross}
 opcode.


 New in version 4.13
%\hline 


\begin{comment}
\begin{tabular}{lcr}
Previous &Home &Next \\
pvsanal &Up &pvsfread

\end{tabular}


\end{document}
\end{comment}
