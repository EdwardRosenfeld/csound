\begin{comment}
\documentclass[10pt]{article}
\usepackage{fullpage, graphicx, url}
\setlength{\parskip}{1ex}
\setlength{\parindent}{0ex}
\title{FLsetAlign}
\begin{document}


\begin{tabular}{ccc}
The Alternative Csound Reference Manual & & \\
Previous & &Next

\end{tabular}

%\hline 
\end{comment}
\section{FLsetAlign}
FLsetAlign�--� Sets the text alignment of a label of a FLTK widget. \subsection*{Description}


 \emph{FLsetAlign}
 sets the text alignment of the label of the target widget. 
\subsection*{Syntax}


 \textbf{FLsetAlign}
 ialign, ihandle
\subsection*{Initialization}


 \emph{ialign}
 -- sets the alignment of the label text of widgets. 


  The legal values for the \emph{ialign}
 argument are: 


 
\begin{itemize}
\item 

 1 - align center

\item 

 2 - align top

\item 

 3 - align bottom

\item 

 4 - align left

\item 

 5 - align right

\item 

 6 - align top-left

\item 

 7 - align top-right

\item 

 8 - align bottom-left

\item 

 9 - align bottom-right


\end{itemize}


 \emph{ihandle}
 -- an integer number (used as unique identifier) taken from the output of a previously located widget opcode (which corresponds to the target widget). It is used to unequivocally identify the widget when modifying its appearance with this class of opcodes. The user must not set the \emph{ihandle}
 value directly, otherwise a Csound crash will occur. 
\subsection*{See Also}


 \emph{FLcolor}
, \emph{FLcolor2}
, \emph{FLhide}
, \emph{FLlabel}
, \emph{FLsetBox}
, \emph{FLsetColor}
, \emph{FLsetColor2}
, \emph{FLsetFont}
, \emph{FLsetPosition}
, \emph{FLsetSize}
, \emph{FLsetText}
, \emph{FLsetTextColor}
, \emph{FLsetTextSize}
, \emph{FLsetTextType}
, \emph{FLsetVal\_i}
, \emph{FLsetVal}
, \emph{FLshow}

\subsection*{Credits}


 Author: Gabriel Maldonado


 New in version 4.22
%\hline 


\begin{comment}
\begin{tabular}{lcr}
Previous &Home &Next \\
FLscrollEnd &Up &FLsetBox

\end{tabular}


\end{document}
\end{comment}
