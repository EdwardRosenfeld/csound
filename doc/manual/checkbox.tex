\begin{comment}
\documentclass[10pt]{article}
\usepackage{fullpage, graphicx, url}
\setlength{\parskip}{1ex}
\setlength{\parindent}{0ex}
\title{checkbox}
\begin{document}


\begin{tabular}{ccc}
The Alternative Csound Reference Manual & & \\
Previous & &Next

\end{tabular}

%\hline 
\end{comment}
\section{checkbox}
checkbox�--� Sense on-screen controls. \subsection*{Description}


  Sense on-screen controls. Requires Winsound or TCL/TK. 
\subsection*{Syntax}


 kr \textbf{checkbox}
 knum
\subsection*{Performance}


 \emph{kr}
 -- value of the checkbox control. If the checkbox is set (pushed) then return 1, if not, return 0. 


 \emph{knum}
 -- the number of the checkbox. If it does not exist, it is made on-screen at initialization. 
\subsection*{Examples}


  Here is a simple example of the checkbox opcode. It uses the files \emph{checkbox.orc}
 and \emph{checkbox.sco}
. 


 \textbf{Example 1. Simple example of the checkbox opcode.}

\begin{lstlisting}
/* checkbox.orc */
; Initialize the global variables.
sr = 44100
kr = 44100
ksmps = 1
nchnls = 1
 
instr 1
  ; Get the value from the checkbox.
  k1 checkbox 1

  ; If the checkbox is selected then k2=440, otherwise k2=880.
  k2 = (k1 == 0 ? 440 : 880)

  a1 oscil 10000, k2, 1
  out a1
endin
/* checkbox.orc */
        
\end{lstlisting}
\begin{lstlisting}
/* checkbox.sco */
; Just generate a nice, ordinary sine wave.
f 1 0 32768 10 1

; Play Instrument #1 for ten seconds.
i 1 0 10 
e
/* checkbox.sco */
        
\end{lstlisting}
\subsection*{See Also}


 \emph{button}

\subsection*{Credits}


 


 


\begin{tabular}{cccc}
Author: John ffitch &University of Bath, Codemist. Ltd. &Bath, UK &September, 2000

\end{tabular}



 


 Example written by Kevin Conder.


 New in Csound version 4.08
%\hline 


\begin{comment}
\begin{tabular}{lcr}
Previous &Home &Next \\
chanctrl &Up &cigoto

\end{tabular}


\end{document}
\end{comment}
