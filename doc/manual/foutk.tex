\begin{comment}
\documentclass[10pt]{article}
\usepackage{fullpage, graphicx, url}
\setlength{\parskip}{1ex}
\setlength{\parindent}{0ex}
\title{foutk}
\begin{document}


\begin{tabular}{ccc}
The Alternative Csound Reference Manual & & \\
Previous & &Next

\end{tabular}

%\hline 
\end{comment}
\section{foutk}
foutk�--� Outputs k-rate signals of an arbitrary number of channels to a specified file. \subsection*{Description}


 \emph{foutk}
 outputs \emph{N}
 a-rate signals to a specified file of \emph{N}
 channels. 
\subsection*{Syntax}


 \textbf{foutk}
 ifilename, iformat, kout1 [, kout2, kout3,....,koutN]
\subsection*{Initialization}


 \emph{ifilename}
 -- the output file's name (in double-quotes). 


 \emph{iformat}
 -- a flag to choose output file format: 


 
\begin{itemize}
\item 

 0 - 32-bit floating point samples without header (binary PCM multichannel file)

\item 

 1 - 16-bit integers without header (binary PCM multichannel file)

\item 

 2 - 16-bit integers with .wav type header (Microsoft WAV mono or stereo file)


\end{itemize}
\subsection*{Performance}


 \emph{kout1,...koutN}
 -- control-rate signals to be written to the file 


 \emph{foutk}
 operates in the same way as \emph{fout}
, but with k-rate signals. \emph{iformat}
 can be set only to 0 or 1. 


  Notice that \emph{fout}
 and \emph{foutk}
 can use either a string containing a file pathname, or a handle-number generated by \emph{fiopen}
. Whereas, with \emph{fouti}
 and \emph{foutir}
, the target file can be only specified by means of a handle-number. 
\subsection*{See Also}


 \emph{fiopen}
, \emph{fout}
, \emph{fouti}
, \emph{foutir}

\subsection*{Credits}


 


 


\begin{tabular}{ccc}
Author: Gabriel Maldonado &Italy &1999

\end{tabular}



 


 New in Csound version 3.56
%\hline 


\begin{comment}
\begin{tabular}{lcr}
Previous &Home &Next \\
foutir &Up &fprintks

\end{tabular}


\end{document}
\end{comment}
