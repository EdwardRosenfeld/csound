\begin{comment}
\documentclass[10pt]{article}
\usepackage{fullpage, graphicx, url}
\setlength{\parskip}{1ex}
\setlength{\parindent}{0ex}
\title{bqrez}
\begin{document}


\begin{tabular}{ccc}
The Alternative Csound Reference Manual & & \\
Previous & &Next

\end{tabular}

%\hline 
\end{comment}
\section{bqrez}
bqrez�--� A second-order multi-mode filter. \subsection*{Description}


  A second-order multi-mode filter. 
\subsection*{Syntax}


 ar \textbf{bqrez}
 asig, xfco, xres [, imode]
\subsection*{Initialization}


 \emph{imode}
 (optional, default=0) -- The mode of the filter. Choose from one of the following: 


 
\begin{itemize}
\item 

 0 = low-pass (default)

\item 

 1 = high-pass

\item 

 2 = band-pass

\item 

 3 = band-reject

\item 

 4 = all-pass


\end{itemize}
\subsection*{Performance}


 \emph{ar}
 -- output audio signal. 


 \emph{asig}
 -- input audio signal. 


 \emph{xfco}
 -- filter cut-off frequency in Hz. May be i-time, k-rate, a-rate. 


 \emph{xres}
 -- amount of resonance. Values of 1 to 100 are typical. Resonance should be one or greater. A value of 100 gives a 20dB gain at the cutoff frequency. May be i-time, k-rate, a-rate. 


  All filter modes can be frequency modulated as well as the resonance can also be frequency modulated. 


 \emph{bqrez}
 is a resonant low-pass filter created using the Laplace s-domain equations for low-pass, high-pass, and band-pass filters normalized to a frequency. The bi-linear transform was used which contains a frequency transform constant from s-domain to z-domain to exactly match the frequencies together. Alot of trigonometric identities where used to simplify the calculation. It is very stable across the working frequency range up to the Nyquist frequency. 
\subsection*{Examples}


  Here is an example of the bqrez opcode. It uses the files \emph{bqrez.orc}
 and \emph{bqrez.sco}
. 


 \textbf{Example 1. Example of the bqrez opcode borrowed from the ``rezzy'' opcode in Kevin Conder's manual.}

\begin{lstlisting}
/* bqrez.orc */
/* Written by Matt Gerassimof from example by Kevin Conder */
; Initialize the global variables.
sr = 44100
kr = 4410
ksmps = 10
nchnls = 1

; Instrument #1.
instr 1
  ; Use a nice sawtooth waveform.
  asig vco 16000, 220, 1

  ; Vary the filter-cutoff frequency from .2 to 2 KHz.
  kfco line 200, p3, 2000

  ; Set the resonance amount.
  kres init 0.99

  a1 bqrez asig, kfco, kres

  out a1
endin
/* bqrez.orc */
        
\end{lstlisting}
\begin{lstlisting}
/* bqrez.sco */
/* Written by Matt Gerassimof from example by Kevin Conder */
; Table #1, a sine wave for the vco opcode.
f 1 0 16384 10 1

; Play Instrument #1 for three seconds.
i 1 0 3
e
/* bqrez.sco */
        
\end{lstlisting}
\subsection*{See Also}


 \emph{biquad}
, \emph{moogvcf}
, \emph{rezzy}

\subsection*{Credits}


 


 


\begin{tabular}{ccc}
Author: Matt Gerassimoff &New in version 4.32 &Written in November 2002.

\end{tabular}



 
%\hline 


\begin{comment}
\begin{tabular}{lcr}
Previous &Home &Next \\
birnd &Up &butbp

\end{tabular}


\end{document}
\end{comment}
