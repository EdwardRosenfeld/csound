\begin{comment}
\documentclass[10pt]{article}
\usepackage{fullpage, graphicx, url}
\setlength{\parskip}{1ex}
\setlength{\parindent}{0ex}
\title{fprints}
\begin{document}


\begin{tabular}{ccc}
The Alternative Csound Reference Manual & & \\
Previous & &Next

\end{tabular}

%\hline 
\end{comment}
\section{fprints}
fprints�--� Similar to prints but prints to a file. \subsection*{Description}


  Similar to \emph{prints}
 but prints to a file. 
\subsection*{Syntax}


 \textbf{fprints}
 ``filename'', ``string'' [, kval1] [, kval2] [...]
\subsection*{Initialization}


 \emph{``filename''}
 -- name of the output file. 


 \emph{``string''}
 -- the text string to be printed. Can be up to 8192 characters and must be in double quotes. 
\subsection*{Performance}


 \emph{kval1, kval2, ...}
 (optional) -- The k-rate values to be printed. These are specified in \emph{``string''}
 with the standard C value specifier (\%f, \%d, etc.) in the order given. 


 \emph{fprints}
 is similar to the \emph{prints}
 opcode except it outputs to a file. For more information about output formatting, please look at \emph{printks's documentation}
. 
\subsection*{Examples}


  Here is an example of the fprints opcode. It uses the files \emph{fprints.orc}
 and \emph{fprints.sco}
. 


 \textbf{Example 1. Example of the fprints opcode.}

\begin{lstlisting}
/* fprints.orc */
/* Written by Matt Ingalls, edited by Kevin Conder. */
; Initialize the global variables.
sr = 44100
kr = 4410
ksmps = 10
nchnls = 1

; Instrument #1 - a score generator example.
instr 1
  ; Print to the file "my.sco".
  fprints "my.sco", "%!Generated score by ma++\\n \\n"
endin
/* fprints.orc */
        
\end{lstlisting}
\begin{lstlisting}
/* fprints.sco */
/* Written by Matt Ingalls, edited by Kevin Conder. */
; Play Instrument #1.
i 1 0 0.001
/* fprints.sco */
        
\end{lstlisting}
 This example will generate a file called ``my.sco''. It should contain a line like this: \begin{lstlisting}
;Generated score by ma++
      
\end{lstlisting}
\subsection*{See Also}


 \emph{prints}

\subsection*{Credits}


 


 


\begin{tabular}{cc}
Author: Matt Ingalls &January 2003

\end{tabular}



 
%\hline 


\begin{comment}
\begin{tabular}{lcr}
Previous &Home &Next \\
fprintks &Up &frac

\end{tabular}


\end{document}
\end{comment}
