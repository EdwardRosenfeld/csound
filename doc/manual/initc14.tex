\begin{comment}
\documentclass[10pt]{article}
\usepackage{fullpage, graphicx, url}
\setlength{\parskip}{1ex}
\setlength{\parindent}{0ex}
\title{initc14}
\begin{document}


\begin{tabular}{ccc}
The Alternative Csound Reference Manual & & \\
Previous & &Next

\end{tabular}

%\hline 
\end{comment}
\section{initc14}
initc14�--� Initializes the controllers used to create a 14-bit MIDI value. \subsection*{Description}


  Initializes the controllers used to create a 14-bit MIDI value. 
\subsection*{Syntax}


 \textbf{initc14}
 ichan, ictlno1, ictlno2, ivalue
\subsection*{Initialization}


 \emph{ichan}
 -- MIDI channel (1-16) 


 \emph{ictlno1}
 -- most significant byte controller number (0-127) 


 \emph{ictlno2}
 -- least significant byte controller number (0-127) 


 \emph{ivalue}
 -- floating point value (must be within 0 to 1) 
\subsection*{Performance}


 \emph{initc14}
 can be used together with both \emph{midic14}
 and \emph{ctrl14}
 opcodes for initializing the first controller's value. \emph{ivalue}
 argument must be set with a number within 0 to 1. An error occurs if it is not. Use the following formula to set \emph{ivalue}
 according with \emph{midic14}
 and \emph{ctrl14}
 min and max range: 


 ivalue�=�(initial\_value�-�min)�/�(max�-�min)\\ 
 �������
\subsection*{See Also}


 \emph{ctrl7}
, \emph{ctrl14}
, \emph{ctrl21}
, \emph{ctrlinit}
, \emph{initc7}
, \emph{initc21}
, \emph{midic7}
, \emph{midic14}
, \emph{midic21}

\subsection*{Credits}


 


 


\begin{tabular}{cc}
Author: Gabriel Maldonado &Italy

\end{tabular}



 


 New in Csound version 3.47


 Thanks goes to Rasmus Ekman for pointing out the correct MIDI channel and controller number ranges.
%\hline 


\begin{comment}
\begin{tabular}{lcr}
Previous &Home &Next \\
init &Up &initc21

\end{tabular}


\end{document}
\end{comment}
