\begin{comment}
\documentclass[10pt]{article}
\usepackage{fullpage, graphicx, url}
\setlength{\parskip}{1ex}
\setlength{\parindent}{0ex}
\title{MIDI Support}
\begin{document}


\begin{tabular}{ccc}
The Alternative Csound Reference Manual & & \\
Previous & &Next

\end{tabular}

%\hline 
\end{comment}
\section{MIDI Support}
\section{Controller Input}


  Opocodes that accept MIDI input are \emph{aftouch}
, \emph{chanctrl}
, \emph{ctrl7}
, \emph{ctrl14}
, \emph{ctrl21}
, \emph{initc7}
, \emph{initc14}
, \emph{initc21}
, \emph{midic7}
, \emph{midic14}
, \emph{midic21}
, \emph{midichannelaftertouch}
, \emph{midichn}
, \emph{midicontrolchange}
, \emph{mididefault}
, \emph{midinoteoff}
, \emph{midinoteoncps}
, \emph{midinoteonkey}
, \emph{midinoteonoct}
, \emph{midinoteonpch}
, \emph{midipitchbend}
, \emph{midipolyaftertouch}
, \emph{midiprogramchange}
, and \emph{polyaft}
. 
%\hline 


\begin{comment}
\begin{tabular}{lcr}
Previous &Home &Next \\
Trigonometric Functions &Up &Converters

\end{tabular}


\end{document}
\end{comment}
