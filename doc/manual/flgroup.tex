\begin{comment}
\documentclass[10pt]{article}
\usepackage{fullpage, graphicx, url}
\setlength{\parskip}{1ex}
\setlength{\parindent}{0ex}
\title{FLgroup}
\begin{document}


\begin{tabular}{ccc}
The Alternative Csound Reference Manual & & \\
Previous & &Next

\end{tabular}

%\hline 
\end{comment}
\section{FLgroup}
FLgroup�--� A FLTK container opcode that groups child widgets. \subsection*{Description}


  A FLTK container opcode that groups child widgets. 
\subsection*{Syntax}


 \textbf{FLgroup}
 ``label'', iwidth, iheight, ix, iy [, iborder] [, image]
\subsection*{Initialization}


 \emph{``label''}
 -- a double-quoted string containing some user-provided text, placed near the corresponding widget. 


 \emph{iwidth}
 -- width of widget. 


 \emph{iheight}
 -- height of widget. 


 \emph{ix}
 -- horizontal position of upper left corner of the valuator, relative to the upper left corner of corresponding window (expressed in pixels). 


 \emph{iy}
 -- vertical position of upper left corner of the valuator, relative to the upper left corner of corresponding window (expressed in pixels). 


 \emph{iborder}
 (optional, default=0) -- border type of the container. It is expressed by means of an integer number chosen from the following: 


 
\begin{itemize}
\item 

 0 - no border

\item 

 1 - down box border

\item 

 2 - up box border

\item 

 3 - engraved border

\item 

 4 - embossed border

\item 

 5 - black line border

\item 

 6 - thin down border

\item 

 7 - thin up border


\end{itemize}
 If the integer number doesn't match any of the previous values, no border is provided as the default. 

 \emph{image}
 (optional) -- a handle referring to an eventual image opened with the \emph{bmopen}
 opcode. If it is set, it allows a skin for that widget. 


 


\begin{tabular}{cc}
\textbf{Note about the bmopen opcode}
 \\
� &

  Although the documentation mentions the \emph{bmopen}
 opcode, it has not been implemented in Csound 4.22. 


\end{tabular}

\subsection*{Performance}


  Containers are useful to format the graphic appearance of the widgets. The most important container is \emph{FLpanel}
, that actually creates a window. It can be filled with other containers and/or valuators or other kinds of widgets. 


  There are no k-rate arguments in containers. 
\subsection*{See Also}


 \emph{FLgroupEnd}
, \emph{FLpack}
, \emph{FLpackEnd}
, \emph{FLpanel}
, \emph{FLpanelEnd}
, \emph{FLscroll}
, \emph{FLscrollEnd}
, \emph{FLtabs}
, \emph{FLtabsEnd}

\subsection*{Credits}


 Author: Gabriel Maldonado


 New in version 4.22
%\hline 


\begin{comment}
\begin{tabular}{lcr}
Previous &Home &Next \\
FLgetsnap &Up &FLgroupEnd

\end{tabular}


\end{document}
\end{comment}
