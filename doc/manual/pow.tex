\begin{comment}
\documentclass[10pt]{article}
\usepackage{fullpage, graphicx, url}
\setlength{\parskip}{1ex}
\setlength{\parindent}{0ex}
\title{pow}
\begin{document}


\begin{tabular}{ccc}
The Alternative Csound Reference Manual & & \\
Previous & &Next

\end{tabular}

%\hline 
\end{comment}
\section{pow}
pow�--� Computes one argument to the power of another argument. \subsection*{Description}


  Computes \emph{xarg}
 to the power of \emph{kpow}
 (or \emph{ipow}
) and scales the result by \emph{inorm}
. 


 
\subsection*{Syntax}


 ar \textbf{pow}
 aarg, kpow [, inorm]


 ir \textbf{pow}
 iarg, ipow [, inorm]


 kr \textbf{pow}
 karg, kpow [, inorm]
\subsection*{Initialization}


 \emph{inorm}
 (optional, default=1) -- The number to divide the result (default to 1). This is especially useful if you are doing powers of a- or k- signals where samples out of range are extremely common! 
\subsection*{Performance}


 \emph{aarg}
, \emph{iarg}
, \emph{karg}
 -- the base. 


 \emph{ipow}
, \emph{kpow}
 -- the exponent. 


 


\begin{tabular}{cc}
Note &

  Use \emph{\^{}}
 with caution in arithmetical statements, as the precedence may not be correct. New in Csound version 3.493. 


\end{tabular}

\subsection*{Examples}


  Here is an example of the pow opcode. It uses the files \emph{pow.orc}
 and \emph{pow.sco}
. 


 \textbf{Example 1. Example of the pow opcode.}

\begin{lstlisting}
/* pow.orc */
; Initialize the global variables.
sr = 44100
kr = 4410
ksmps = 10
nchnls = 1

; Instrument #1.
instr 1
  ; This could also be expressed as: i1 = 2 ^ 12
  i1 pow 2, 12

  print i1
endin
/* pow.orc */
        
\end{lstlisting}
\begin{lstlisting}
/* pow.sco */
; Play Instrument #1 for one second.
i 1 0 1
e
/* pow.sco */
        
\end{lstlisting}
 Its output should include a line like this: \begin{lstlisting}
instr 1:  i1 = 4096.000
      
\end{lstlisting}
\subsection*{Credits}


 


 


\begin{tabular}{ccc}
Author: Paris Smaragdis &MIT, Cambridge &1995

\end{tabular}



 


 Example written by Kevin Conder.
%\hline 


\begin{comment}
\begin{tabular}{lcr}
Previous &Home &Next \\
poscil3 &Up &powoftwo

\end{tabular}


\end{document}
\end{comment}
