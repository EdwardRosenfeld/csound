\begin{comment}
\documentclass[10pt]{article}
\usepackage{fullpage, graphicx, url}
\setlength{\parskip}{1ex}
\setlength{\parindent}{0ex}
\title{out}
\begin{document}


\begin{tabular}{ccc}
The Alternative Csound Reference Manual & & \\
Previous & &Next

\end{tabular}

%\hline 
\end{comment}
\section{out}
out�--� Writes mono audio data to an external device or stream. \subsection*{Description}


  Writes mono audio data to an external device or stream. 
\subsection*{Syntax}


 \textbf{out}
 asig
\subsection*{Performance}


  Sends mono audio samples to an accumulating output buffer (created at the beginning of performance) which serves to collect the output of all active instruments before the sound is written to disk. There can be any number of these output units in an instrument. 


  The type (mono, stereo, quad, hex, or oct) should agree with \emph{nchnls}
. But as of version 3.50, Csound will attempt to change an incorrect opcode to agree with \emph{nchnls}
 statement. 
\subsection*{See Also}


 \emph{outh}
, \emph{outo}
, \emph{outq}
, \emph{outq1}
, \emph{outq2}
, \emph{outq3}
, \emph{outq4}
, \emph{outs}
, \emph{outs1}
, \emph{outs2}
, \emph{soundout}

\subsection*{Credits}


 


 


\begin{tabular}{ccc}
Author: Barry L. Vercoe, Matt Ingalls/Mike Berry &MIT, Mills College &1993-1997

\end{tabular}



 
%\hline 


\begin{comment}
\begin{tabular}{lcr}
Previous &Home &Next \\
oscilx &Up &out32

\end{tabular}


\end{document}
\end{comment}
