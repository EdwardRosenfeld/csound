\begin{comment}
\documentclass[10pt]{article}
\usepackage{fullpage, graphicx, url}
\setlength{\parskip}{1ex}
\setlength{\parindent}{0ex}
\title{GEN32}
\begin{document}


\begin{tabular}{ccc}
The Alternative Csound Reference Manual & & \\
Previous & &Next

\end{tabular}

%\hline 
\end{comment}
\section{GEN32}
GEN32�--� Mixes any waveform, resampled with either FFT or linear interpolation. \subsection*{Description}


  This routine is similar to \emph{GEN31}
, but allows specifying source ftable for each partial. Tables can be resampled either with FFT, or linear interpolation. 
\subsection*{Syntax}


 \textbf{f}
 \# time size 32 srca pna stra phsa srcb pnb strb phsb ...
\subsection*{Performance}


 \emph{srca, srcb}
 -- source table number. A negative value can be used to read the table with linear interpolation (by default, the source waveform is transposed and phase shifted using FFT); this is less accurate, but faster, and allows non-integer and negative partial numbers. 


 \emph{pna, pnb, ...}
 -- partial number, must be a positive integer if source table number is positive (i.e. resample with FFT). 


 \emph{stra, strb, ...}
 -- amplitude scale 


 \emph{phsa, phsb, ...}
 -- start phase (0 to 1) 
\subsection*{Examples}


 


 
\begin{lstlisting}
itmp    ftgen 1, 0, 16384, 7, 1, 16384, -1      ; sawtooth
itmp    ftgen 2, 0, 8192, 10, 1                 ; sine
; mix tables
itmp    ftgen 5, 0, 4096, -32, -2, 1.5, 1.0, 0.25, 1, 2, 0.5, 0,        \
                                1, 3, -0.25, 0.5
; window
itmp    ftgen 6, 0, 16384, 20, 3, 1
; generate band-limited waveforms
inote   =  0
loop0:
icps    =  440 * exp(log(2) * (inote - 69) / 12)        ; one table for
inumh   =  sr / (2 * icps)                              ; each MIDI note number
ift     =  int(inote + 256.5)
itmp    ftgen ift, 0, 4096, -30, 5, 1, inumh
inote   =  inote + 1
        if (inote < 127.5) igoto loop0

        instr 1

kcps    expon 20, p3, 16000
kft     =  int(256.5 + 69 + 12 * log(kcps / 440) / log(2))
kft     =  (kft > 383 ? 383 : kft)

a1      phasor kcps
a1      tableikt a1, kft, 1, 0, 1

        out a1 * 10000

        endin
        instr 2

kcps    expon 20, p3, 16000
kft     =  int(256.5 + 69 + 12 * log(kcps / 440) / log(2))
kft     =  (kft > 383 ? 383 : kft)

kgdur   limit 10 / kcps, 0.1, 1
a1      grain2 kcps, 0.02, kgdur, 30, kft, 6, -0.5

        out a1 * 2000

        endin

----------
score:
----------

t 0 60
i 1 0 10
i 2 12 10
e
        
\end{lstlisting}


 
\subsection*{Credits}


 Author: Rasmus Ekman


 Programmer: Istvan Varga


 New in version 4.17
%\hline 


\begin{comment}
\begin{tabular}{lcr}
Previous &Home &Next \\
GEN31 &Up &GEN33

\end{tabular}


\end{document}
\end{comment}
