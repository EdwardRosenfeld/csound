\begin{comment}
\documentclass[10pt]{article}
\usepackage{fullpage, graphicx, url}
\setlength{\parskip}{1ex}
\setlength{\parindent}{0ex}
\title{specfilt}
\begin{document}


\begin{tabular}{ccc}
The Alternative Csound Reference Manual & & \\
Previous & &Next

\end{tabular}

%\hline 
\end{comment}
\section{specfilt}
specfilt�--� Filters each channel of an input spectrum. \subsection*{Description}


  Filters each channel of an input spectrum. 
\subsection*{Syntax}


 wsig \textbf{specfilt}
 wsigin, ifhtim
\subsection*{Initialization}


 \emph{ifhtim}
 -- half-time constant. 
\subsection*{Performance}


 \emph{wsigin}
 -- the input spectrum. 


  Filters each channel of an input spectrum. At each new frame of \emph{wsigin}
, each magnitude value is injected into a 1st-order lowpass recursive filter, whose half-time constant has been initially set by sampling the ftable \emph{ifhtim}
 across the (logarithmic) frequency space of the input spectrum. This unit effectively applies a \emph{persistence}
 factor to the data occurring in each spectral channel, and is useful for simulating the \emph{energy integration}
 that occurs during auditory perception. It may also be used as a time-attenuated running \emph{histogram}
 of the spectral distribution. 
\subsection*{Examples}


 


 
\begin{lstlisting}
  wsig2    \emph{specdiff}
         wsig1               ; sense onsets 
  wsig3    \emph{specfilt}
         wsig2, 2            ; absorb slowly 
           \emph{specdisp}
         wsig2, .1           ; & display both spectra 
           \emph{specdisp}
         wsig3, .1
        
\end{lstlisting}


 
\subsection*{See Also}


 \emph{specaddm}
, \emph{specdiff}
, \emph{spechist}
, \emph{specscal}

%\hline 


\begin{comment}
\begin{tabular}{lcr}
Previous &Home &Next \\
specdisp &Up &spechist

\end{tabular}


\end{document}
\end{comment}
