\begin{comment}
\documentclass[10pt]{article}
\usepackage{fullpage, graphicx, url}
\setlength{\parskip}{1ex}
\setlength{\parindent}{0ex}
\title{pvread}
\begin{document}


\begin{tabular}{ccc}
The Alternative Csound Reference Manual & & \\
Previous & &Next

\end{tabular}

%\hline 
\end{comment}
\section{pvread}
pvread�--� Reads from a phase vocoder analysis file and returns the frequency and amplitude from a single analysis channel or bin. \subsection*{Description}


 \emph{pvread}
 reads from a \emph{pvoc}
 file and returns the frequency and amplitude from a single analysis channel or bin. The returned values can be used anywhere else in the Csound instrument. For example, one can use them as arguments to an oscillator to synthesize a single component from an analyzed signal or a bank of pvreads can be used to resynthesize the analyzed sound using additive synthesis by passing the frequency and magnitude values to a bank of oscillators. 
\subsection*{Syntax}


 kfreq, kamp \textbf{pvread}
 ktimpnt, ifile, ibin
\subsection*{Initialization}


 \emph{ifile}
 -- the \emph{pvoc}
 number (n in pvoc.n) or the name in quotes of the analysis file made using pvanal. (See \emph{pvoc}
.) 


 \emph{ibin}
 -- the number of the analysis channel from which to return frequency in Hz and magnitude. 
\subsection*{Performance}


 \emph{kfreq, kamp}
 -- outputs of the \emph{pvread}
 unit. These values, retrieved from a phase vocoder analysis file, represent the values of frequency and amplitude from a single analysis channel specified in the ibin argument. Interpolation between analysis frames is performed at k-rate resolution and dependent of course upon the rate and direction of ktimpnt. 


 \emph{ktimpnt}
 -- the passage of time, in seconds, through this file. \emph{ktimpnt}
 must always be positive, but can move forwards or backwards in time, be stationary or discontinuous, as a pointer into the analysis file. 
\subsection*{Examples}


  The example below shows the use \emph{pvread}
 to synthesize a single component from a phase vocoder analysis file. It should be noted that the \emph{kfreq}
 and \emph{kamp}
 outputs can be used for any kind of synthesis, filtering, processing, and so on. 


 


 
\begin{lstlisting}
ktime         \emph{line}
    0, p3, 3 
kfreq, kamp   \emph{pvread}
  ktime, "pvoc.file", 7 ; read
                                      ;data from 7th analysis bin.  
asig          \emph{oscili}
  kamp, kfreq, 1  ; function 1
                                      ;is a stored sine
        
\end{lstlisting}


 
\subsection*{See Also}


 \emph{pvbufread}
, \emph{pvcross}
, \emph{pvinterp}
, \emph{tableseg}
, \emph{tablexseg}

\subsection*{Credits}


 


 


\begin{tabular}{ccc}
Author: Richard Karpen &Seattle, Wash &1997

\end{tabular}



 
%\hline 


\begin{comment}
\begin{tabular}{lcr}
Previous &Home &Next \\
pvoc &Up &pvsadsyn

\end{tabular}


\end{document}
\end{comment}
