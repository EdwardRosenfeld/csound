\begin{comment}
\documentclass[10pt]{article}
\usepackage{fullpage, graphicx, url}
\setlength{\parskip}{1ex}
\setlength{\parindent}{0ex}
\title{reson}
\begin{document}


\begin{tabular}{ccc}
The Alternative Csound Reference Manual & & \\
Previous & &Next

\end{tabular}

%\hline 
\end{comment}
\section{reson}
reson�--� A second-order resonant filter. \subsection*{Description}


  A second-order resonant filter. 
\subsection*{Syntax}


 ar \textbf{reson}
 asig, kcf, kbw [, iscl] [, iskip]
\subsection*{Initialization}


 \emph{iscl}
 (optional, default=0) -- coded scaling factor for resonators. A value of 1 signifies a peak response factor of 1, i.e. all frequencies other than kcf are attenuated in accordance with the (normalized) response curve. A value of 2 raises the response factor so that its overall RMS value equals 1. (This intended equalization of input and output power assumes all frequencies are physically present; hence it is most applicable to white noise.) A zero value signifies no scaling of the signal, leaving that to some later adjustment (see \emph{balance}
). The default value is 0. 


 \emph{iskip}
 (optional, default=0) -- initial disposition of internal data space. Since filtering incorporates a feedback loop of previous output, the initial status of the storage space used is significant. A zero value will clear the space; a non-zero value will allow previous information to remain. The default value is 0. 
\subsection*{Performance}


 \emph{ar}
 -- the output signal at audio rate. 


 \emph{asig}
 -- the input signal at audio rate. 


 \emph{kcf}
 -- the center frequency of the filter, or frequency position of the peak response. 


 \emph{kbw}
 -- bandwidth of the filter (the Hz difference between the upper and lower half-power points). 


 \emph{reson}
 is a second-order filter in which \emph{kcf}
 controls the center frequency, or frequency position of the peak response, and \emph{kbw}
 controls its bandwidth (the frequency difference between the upper and lower half-power points). 
\subsection*{Examples}


  Here is an example of the reson opcode. It uses the files \emph{reson.orc}
 and \emph{reson.sco}
. 


 \textbf{Example 1. Example of the reson opcode.}

\begin{lstlisting}
/* reson.orc */
; Initialize the global variables.
sr = 44100
kr = 4410
ksmps = 10
nchnls = 1

; Instrument #1.
instr 1
  ; Generate a sine waveform.
  asine buzz 15000, 440, 3, 1

  ; Vary the cut-off frequency from 220 to 1280.
  kcf line 220, p3, 1320
  kbw init 20

  ; Run the sine through a resonant filter.
  ares reson asine, kcf, kbw

  ; Give the filtered signal the same amplitude 
  ; as the original signal. 
  a1 balance ares, asine
  out a1
endin
/* reson.orc */
        
\end{lstlisting}
\begin{lstlisting}
/* reson.sco */
; Table #1, an ordinary sine wave.
f 1 0 16384 10 1

; Play Instrument #1 for 4 seconds.
i 1 0 4
e
/* reson.sco */
        
\end{lstlisting}
\subsection*{See Also}


 \emph{areson}
, \emph{aresonk}
, \emph{atone}
, \emph{atonek}
, \emph{port}
, \emph{portk}
, \emph{resonk}
, \emph{tone}
, \emph{tonek}

\subsection*{Credits}


 Example written by Kevin Conder.
%\hline 


\begin{comment}
\begin{tabular}{lcr}
Previous &Home &Next \\
repluck &Up &resonk

\end{tabular}


\end{document}
\end{comment}
