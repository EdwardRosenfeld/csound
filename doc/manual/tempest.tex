\begin{comment}
\documentclass[10pt]{article}
\usepackage{fullpage, graphicx, url}
\setlength{\parskip}{1ex}
\setlength{\parindent}{0ex}
\title{tempest}
\begin{document}


\begin{tabular}{ccc}
The Alternative Csound Reference Manual & & \\
Previous & &Next

\end{tabular}

%\hline 
\end{comment}
\section{tempest}
tempest�--� Estimate the tempo of beat patterns in a control signal. \subsection*{Description}


  Estimate the tempo of beat patterns in a control signal. 
\subsection*{Syntax}


 ktemp \textbf{tempest}
 kin, iprd, imindur, imemdur, ihp, ithresh, ihtim, ixfdbak, istartempo, ifn [, idisprd] [, itweek]
\subsection*{Initialization}


 \emph{iprd}
 -- period between analyses (in seconds). Typically about .02 seconds. 


 \emph{imindur}
 -- minimum duration (in seconds) to serve as a unit of tempo. Typically about .2 seconds. 


 \emph{imemdur}
 -- duration (in seconds) of the \emph{kin}
 short-term memory buffer which will be scanned for periodic patterns. Typically about 3 seconds. 


 \emph{ihp}
 -- half-power point (in Hz) of a low-pass filter used to smooth input \emph{kin}
 prior to other processing. This will tend to suppress activity that moves much faster. Typically 2 Hz. 


 \emph{ithresh}
 -- loudness threshold by which the low-passed \emph{kin}
 is center-clipped before being placed in the short-term buffer as tempo-relevant data. Typically at the noise floor of the incoming data. 


 \emph{ihtim}
 -- half-time (in seconds) of an internal forward-masking filter that masks new \emph{kin}
 data in the presence of recent, louder data. Typically about .005 seconds. 


 \emph{ixfdbak}
 -- proportion of this unit's \emph{anticipated value}
 to be mixed with the incoming \emph{kin}
 prior to all processing. Typically about .3. 


 \emph{istartempo}
 -- initial tempo (in beats per minute). Typically 60. 


 \emph{ifn}
 -- table number of a stored function (drawn left-to-right) by which the short-term memory data is attenuated over time. 


 \emph{idisprd}
 (optional) -- if non-zero, display the short-term past and future buffers every \emph{idisprd}
 seconds (normally a multiple of \emph{iprd}
). The default value is 0 (no display). 


 \emph{itweek}
 (optional) -- fine-tune adjust this unit so that it is stable when analyzing events controlled by its own output. The default value is 1 (no change). 
\subsection*{Performance}


 \emph{tempest}
 examines \emph{kin}
 for amplitude periodicity, and estimates a current tempo. The input is first low-pass filtered, then center-clipped, and the residue placed in a short-term memory buffer (attenuated over time) where it is analyzed for periodicity using a form of autocorrelation. The period, expressed as a \emph{tempo}
 in beats per minute, is output as \emph{ktemp}
. The period is also used internally to make predictions about future amplitude patterns, and these are placed in a buffer adjacent to that of the input. The two adjacent buffers can be periodically displayed, and the predicted values optionally mixed with the incoming signal to simulate expectation. 


 This unit is useful for sensing the metric implications of any k-signal (e.g.- the RMS of an audio signal, or the second derivative of a conducting gesture), before sending to a \emph{tempo}
 statement. 
\subsection*{Examples}


  Here is an example of the tempest opcode. It uses the files \emph{tempest.orc}
, \emph{tempest.sco}
, and \emph{beats.wav}
. 


 \textbf{Example 1. Example of the tempest opcode.}

\begin{lstlisting}
/* tempest.orc */
; Initialize the global variables.
sr = 44100
kr = 4410
ksmps = 10
nchnls = 1

; Instrument #1.
instr 1
  ; Use the "beats.wav" sound file.
  asig soundin "beats.wav"
  ; Extract the pitch and the envelope.
  kcps, krms pitchamdf asig, 150, 500, 200

  iprd = 0.01
  imindur = 0.1
  imemdur = 3
  ihp = 1
  ithresh = 30
  ihtim = 0.005
  ixfdbak = 0.05
  istartempo = 110
  ifn = 1

  ; Estimate its tempo.
  k1 tempest krms, iprd, imindur, imemdur, ihp, ithresh, ihtim, ixfdbak, istartempo, ifn
  printk2 k1

  out asig
endin
/* tempest.orc */
        
\end{lstlisting}
\begin{lstlisting}
/* tempest.sco */
; Table #1, a declining line.
f 1 0 128 16 1 128 1

; Play Instrument #1 for two seconds.
i 1 0 2
e
/* tempest.sco */
        
\end{lstlisting}
 The tempo of the audio file ``beats.wav'' is 120 beats per minute. In this examples, tempest will print out its best guess as the audio file plays. Its output should include lines like this: \begin{lstlisting}
. i1   118.24654
. i1   121.72949
      
\end{lstlisting}
%\hline 


\begin{comment}
\begin{tabular}{lcr}
Previous &Home &Next \\
tbvcf &Up &tempo

\end{tabular}


\end{document}
\end{comment}
