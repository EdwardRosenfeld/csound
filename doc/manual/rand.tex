\begin{comment}
\documentclass[10pt]{article}
\usepackage{fullpage, graphicx, url}
\setlength{\parskip}{1ex}
\setlength{\parindent}{0ex}
\title{rand}
\begin{document}


\begin{tabular}{ccc}
The Alternative Csound Reference Manual & & \\
Previous & &Next

\end{tabular}

%\hline 
\end{comment}
\section{rand}
rand�--� Generates a controlled random number series. \subsection*{Description}


  Output is a controlled random number series between -\emph{amp}
 and +\emph{amp}

\subsection*{Syntax}


 ar \textbf{rand}
 xamp [, iseed] [, isel] [, ibase]


 kr \textbf{rand}
 xamp [, iseed] [, isel] [, ibase]
\subsection*{Initialization}


 \emph{iseed}
 (optional, default=0.5) -- a seed value for the recursive pseudo-random formula. A value between 0 and 1 will produce an initial output of \emph{kamp * iseed}
. A value greater than 1 will be seeded from the system clock. A negative value will cause seed re-initialization to be skipped. The default seed value is .5. 


 \emph{isel}
 (optional, default=0) -- if zero, a 16-bit number is generated. If non-zero, a 31-bit random number is generated. Default is 0. 


 \emph{ioffset}
 (optional, default=0) -- a base value added to the random result. New in Csound version 4.03. 
\subsection*{Performance}


 \emph{kamp, xamp}
 -- range over which random numbers are distributed. 


 \emph{kcps, xcps}
 -- the frequency which new random numbers are generated. 


  The internal pseudo-random formula produces values which are uniformly distributed over the range \emph{kamp}
 to \emph{-kamp}
. \emph{rand}
 will thus generate uniform white noise with an R.M.S value of \emph{kamp / root 2}
. 


  The remaining units produce band-limited noise: the \emph{kcps}
 and \emph{xcps}
 parameters permit the user to specify that new random numbers are to be generated at a rate less than the sampling or control frequencies. 
\subsection*{Examples}


  Here is an example of the rand opcode. It uses the files \emph{rand.orc}
 and \emph{rand.sco}
. 


 \textbf{Example 1. Example of the rand opcode.}

\begin{lstlisting}
/* rand.orc */
; Initialize the global variables.
sr = 44100
kr = 4410
ksmps = 10
nchnls = 1

; Instrument #1.
instr 1
  ; Choose a random frequency between 4,100 and 44,100.
  kfreq rand 20000
  kcps = kfreq + 24100

  a1 oscil 30000, kcps, 1
  out a1
endin
/* rand.orc */
        
\end{lstlisting}
\begin{lstlisting}
/* rand.sco */
; Table #1, a sine wave.
f 1 0 16384 10 1

; Play Instrument #1 for one second.
i 1 0 1
e
/* rand.sco */
        
\end{lstlisting}
\subsection*{See Also}


 \emph{randh}
, \emph{randi}

\subsection*{Credits}


 Example written by Kevin Conder.


 Thanks to a note from John ffitch, I changed the names of the parameters.
%\hline 


\begin{comment}
\begin{tabular}{lcr}
Previous &Home &Next \\
pvsynth &Up &randh

\end{tabular}


\end{document}
\end{comment}
