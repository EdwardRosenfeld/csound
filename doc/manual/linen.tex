\begin{comment}
\documentclass[10pt]{article}
\usepackage{fullpage, graphicx, url}
\setlength{\parskip}{1ex}
\setlength{\parindent}{0ex}
\title{linen}
\begin{document}


\begin{tabular}{ccc}
The Alternative Csound Reference Manual & & \\
Previous & &Next

\end{tabular}

%\hline 
\end{comment}
\section{linen}
linen�--� Applies a straight line rise and decay pattern to an input amp signal. \subsection*{Description}


 \emph{linen}
 -- apply a straight line rise and decay pattern to an input amp signal. 
\subsection*{Syntax}


 ar \textbf{linen}
 xamp, irise, idur, idec


 kr \textbf{linen}
 kamp, irise, idur, idec
\subsection*{Initialization}


 \emph{irise}
 -- rise time in seconds. A zero or negative value signifies no rise modification. 


 \emph{idur}
 -- overall duration in seconds. A zero or negative value will cause initialization to be skipped. 


 \emph{idec}
 -- decay time in seconds. Zero means no decay. An \emph{idec}
 $>$ \emph{idur}
 will cause a truncated decay. 
\subsection*{Performance}


 \emph{kamp, xamp}
 -- input amplitude signal. 


  Rise modifications are applied for the first \emph{irise}
 seconds, and decay from time \emph{idur - idec}
. If these periods are separated in time there will be a steady state during which \emph{amp}
 will be unmodified. If \emph{linen}
 rise and decay periods overlap then both modifications will be in effect for that time. If the overall duration \emph{idur}
 is exceeded in performance, the final decay will continue on in the same direction, going negative. 
\subsection*{See Also}


 \emph{envlpx}
, \emph{envlpxr}
, \emph{linenr}

%\hline 


\begin{comment}
\begin{tabular}{lcr}
Previous &Home &Next \\
line &Up &linenr

\end{tabular}


\end{document}
\end{comment}
