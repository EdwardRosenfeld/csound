\begin{comment}
\documentclass[10pt]{article}
\usepackage{fullpage, graphicx, url}
\setlength{\parskip}{1ex}
\setlength{\parindent}{0ex}
\title{cpstuni}
\begin{document}


\begin{tabular}{ccc}
The Alternative Csound Reference Manual & & \\
Previous & &Next

\end{tabular}

%\hline 
\end{comment}
\section{cpstuni}
cpstuni�--� Returns micro-tuning values at init-rate. \subsection*{Description}


  Returns micro-tuning values at init-rate. 
\subsection*{Syntax}


 icps \textbf{cpstuni}
 index, ifn
\subsection*{Initialization}


 \emph{icps}
 -- Return value in cycles per second. 


 \emph{index}
 -- An integer number denoting an index of scale. 


 \emph{ifn}
 -- Function table containing the parameters (numgrades, interval, basefreq, basekeymidi) and the tuning ratios. 
\subsection*{Performance}


  These opcodes are similar to \emph{cpstmid}
, but work without necessity of MIDI. 


 \emph{cpstuni}
 works at init-rate. It allows fully customized micro-tuning scales. It requires a function table number containing the tuning ratios, and some other parameters stored in the function table itself. 


  The \emph{index}
 argument should be filled with integer numbers expressing the grade of given scale to be converted in cps. The function table ifn should be generated by \emph{GEN02}
 and the first four values stored in this function are parameters that express: 


 
\begin{itemize}
\item 

 numgrades -- The number of grades of the micro-tuning scale.

\item 

 interval -- The frequency range covered before repeating the grade ratios, for example 2 for one octave, 1.5 for a fifth etcetera.

\item 

 basefreq -- The base frequency of the scale in cycles per second.

\item 

 basekey -- The integer index of the scale to which to assign basefreq unmodified.


\end{itemize}


  After these four values, the user can begin to insert the tuning ratios. For example, for a standard 12-grade scale with the base-frequency of 261 cps assigned to the key-number 60, the corresponding f-statement in the score to generate the table should be: 


 
\begin{lstlisting}
;           numgrades    basefreq     tuning-ratios (eq.temp) .......
;                  interval    basekey
f1 0 64 -2  12     2     261   60     1   1.059463 1.12246 1.18920 ..etc...
        
\end{lstlisting}


 


 Another example with a 24-grade scale with a base frequency of 440 assigned to the key-number 48, and a repetition interval of 1.5: 


 
\begin{lstlisting}
                  numgrades       basefreq      tuning-ratios .......
                          interval       basekey
f1 0 64 -2         24      1.5     440    48     1   1.01  1.02  1.03   ..etc...
        
\end{lstlisting}


 
\subsection*{Examples}


  Here is an example of the cpstuni opcode. It uses the files \emph{cpstuni.orc}
 and \emph{cpstuni.sco}
. 


 \textbf{Example 1. Example of the cpstuni opcode.}

\begin{lstlisting}
/* cpstuni.orc */
; Initialize the global variables.
sr = 44100
kr = 4410
ksmps = 10
nchnls = 1

; Table #1, a normal 12-tone equal temperament scale.
; numgrades = 12 (twelve tones)
; interval = 2 (one octave)
; basefreq = 261.659 (Middle C)
; basekeymidi = 60 (Middle C)
gitemp ftgen 1, 0, 64, -2, 12, 2, 261.659, 60, 1.00, \
             1.059, 1.122, 1.189, 1.260, 1.335, 1.414, \
             1.498, 1.588, 1.682, 1.782, 1.888, 2.000

; Instrument #1.
instr 1
  ; Use Table #1.
  ifn = 1

  ; If the base key (note #60) is C, then 9 notes 
  ; above it (note #60 + 9 = note #69) should be A.
  index = 69

  i1 cpstuni index, ifn

  print i1
endin
/* cpstuni.orc */
        
\end{lstlisting}
\begin{lstlisting}
/* cpstuni.sco */
; Play Instrument #1 for one second.
i 1 0 1
e
/* cpstuni.sco */
        
\end{lstlisting}
 Its output should include lines like this: \begin{lstlisting}
instr 1:  i1 = 440.110
      
\end{lstlisting}
\subsection*{See Also}


 \emph{cpstmid}
, \emph{cpstun}
, \emph{GEN02}

\subsection*{Credits}


 Example written by Kevin Conder.
%\hline 


\begin{comment}
\begin{tabular}{lcr}
Previous &Home &Next \\
cpstun &Up &cpsxpch

\end{tabular}


\end{document}
\end{comment}
