\begin{comment}
\documentclass[10pt]{article}
\usepackage{fullpage, graphicx, url}
\setlength{\parskip}{1ex}
\setlength{\parindent}{0ex}
\title{nchnls}
\begin{document}


\begin{tabular}{ccc}
The Alternative Csound Reference Manual & & \\
Previous & &Next

\end{tabular}

%\hline 
\end{comment}
\section{nchnls}
nchnls�--� Sets the number of channels of audio output. \subsection*{Description}


  These statements are global value \emph{assignments}
, made at the beginning of an orchestra, before any instrument block is defined. Their function is to set certain \emph{reserved symbol variables}
 that are required for performance. Once set, these reserved symbols can be used in expressions anywhere in the orchestra. 
\subsection*{Syntax}


 \textbf{nchnls}
 = iarg
\subsection*{Initialization}


 \emph{nchnls}
 = (optional) -- set number of channels of audio output to \emph{iarg}
. (1 = mono, 2 = stereo, 4 = quadraphonic.) The default value is 1 (mono). 


  In addition, any \emph{global variable}
 can be initialized by an \emph{init-time assignment}
 anywhere before the first \emph{instr statement}
. All of the above assignments are run as instrument 0 (i-pass only) at the start of real performance. 
\subsection*{See Also}


 \emph{kr}
, \emph{ksmps}
, \emph{sr}

%\hline 


\begin{comment}
\begin{tabular}{lcr}
Previous &Home &Next \\
mxadsr &Up &nestedap

\end{tabular}


\end{document}
\end{comment}
