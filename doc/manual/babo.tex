\begin{comment}
\documentclass[10pt]{article}
\usepackage{fullpage, graphicx, url}
\setlength{\parskip}{1ex}
\setlength{\parindent}{0ex}
\title{babo}
\begin{document}


\begin{tabular}{ccc}
The Alternative Csound Reference Manual & & \\
Previous & &Next

\end{tabular}

%\hline 
\end{comment}
\section{babo}
babo�--� A physical model reverberator. \subsection*{Description}


 \emph{babo}
 stands for \emph{ba}
ll-within-the-\emph{bo}
x. It is a physical model reverberator based on the paper by Davide Rocchesso ``The Ball within the Box: a sound-processing metaphor'', Computer Music Journal, Vol 19, N.4, pp.45-47, Winter 1995. 


  The resonator geometry can be defined, along with some response characteristics, the position of the listener within the resonator, and the position of the sound source. 
\subsection*{Syntax}


 a1, a2 \textbf{babo}
 asig, ksrcx, ksrcy, ksrcz, irx, iry, irz [, idiff] [, ifno]
\subsection*{Initialization}


 \emph{irx, iry, irz}
 -- the coordinates of the geometry of the resonator (length of the edges in meters) 


 \emph{idiff}
 -- is the coefficient of diffusion at the walls, which regulates the amount of diffusion (0-1, where 0 = no diffusion, 1 = maximum diffusion - default: 1) 


 \emph{ifno}
 -- expert values function: a function number that holds all the additional parameters of the resonator. This is typically a GEN2--type function used in non-rescaling mode. They are as follows: 


 
\begin{itemize}
\item 

 \emph{decay}
 -- main decay of the resonator (default: 0.99)

\item 

 \emph{hydecay}
 -- high frequency decay of the resonator (default: 0.1)

\item 

 \emph{rcvx, rcvy, rcvz}
 -- the coordinates of the position of the receiver (the listener) (in meters; 0,0,0 is the resonator center)

\item 

 \emph{rdistance}
 -- the distance in meters between the two pickups (your ears, for example - default: 0.3)

\item 

 \emph{direct}
 -- the attenuation of the direct signal (0-1, default: 0.5)

\item 

 \emph{early\_diff}
 -- the attenuation coefficient of the early reflections (0-1, default: 0.8)


\end{itemize}
\subsection*{Performance}


 \emph{asig}
 -- the input signal 


 \emph{ksrcx, ksrcy, ksrcz}
 -- the virtual coordinates of the source of sound (the input signal). These are allowed to move at k-rate and provide all the necessary variations in terms of response of the resonator. 
\subsection*{Examples}


  Here is a simple example of the babo opcode. It uses the files \emph{babo.orc}
, \emph{babo.sco}
, and \emph{beats.wav}
. 


 \textbf{Example 1. A simple example of the babo opcode.}

\begin{lstlisting}
/* babo.orc */
/* Written by Nicola Bernardini */
; Initialize the global variables.
sr = 44100
kr = 4410
ksmps = 10
nchnls = 2

; minimal babo instrument
;
instr 1
       ix     = p4  ; x position of source
       iy     = p5  ; y position of source
       iz     = p6  ; z position of source
       ixsize = p7  ; width  of the resonator
       iysize = p8  ; depth  of the resonator
       izsize = p9  ; height of the resonator

ainput soundin "beats.wav"

al,ar  babo    ainput*0.7, ix, iy, iz, ixsize, iysize, izsize

       outs    al,ar
endin
/* babo.orc */
        
\end{lstlisting}
\begin{lstlisting}
/* babo.sco */
/* Written by Nicola Bernardini */
; simple babo usage:
;
;p4     : x position of source
;p5     : y position of source
;p6     : z position of source
;p7     : width  of the resonator
;p8     : depth  of the resonator
;p9     : height of the resonator
;
i  1  0  10 6  4  3    14.39 11.86 10
;           ^^^^^^^    ^^^^^^^^^^^^^^
;           |||||||    ++++++++++++++: optimal room dims according to
;           |||||||                    Milner and Bernard JASA 85(2), 1989
;           +++++++++: source position
e
/* babo.sco */
        
\end{lstlisting}


  Here is an advanced example of the babo opcode. It uses the files \emph{babo\_expert.orc}
, \emph{babo\_expert.sco}
, and \emph{beats.wav}
. 


 \textbf{Example 2. An advanced example of the babo opcode.}

\begin{lstlisting}
/* babo_expert.orc */
/* Written by Nicola Bernardini */
; Initialize the global variables.
sr = 44100
kr = 4410
ksmps = 10
nchnls = 2

; full blown babo instrument with movement
;
instr 2
  ixstart = p4   ; start x position of source (left-right)
  ixend   = p7   ; end   x position of source
  iystart = p5   ; start y position of source (front-back)
  iyend   = p8   ; end   y position of source
  izstart = p6   ; start z position of source (up-down)
  izend   = p9  ; end   z position of source
  ixsize  = p10  ; width  of the resonator
  iysize  = p11  ; depth  of the resonator
  izsize  = p12  ; height of the resonator
  idiff   = p13  ; diffusion coefficient
  iexpert = p14  ; power user values stored in this function

ainput    soundin "beats.wav"
ksource_x line    ixstart, p3, ixend
ksource_y line    iystart, p3, iyend
ksource_z line    izstart, p3, izend

al,ar     babo    ainput*0.7, ksource_x, ksource_y, ksource_z, ixsize, iysize, izsize, idiff, iexpert

          outs    al,ar
endin
/* babo_expert.orc */
        
\end{lstlisting}
\begin{lstlisting}
/* babo_expert.sco */
/* Written by Nicola Bernardini */
; full blown instrument
;p4         : start x position of source (left-right)
;p5         : end   x position of source
;p6         : start y position of source (front-back)
;p7         : end   y position of source
;p8         : start z position of source (up-down)
;p9         : end   z position of source
;p10        : width  of the resonator
;p11        : depth  of the resonator
;p12        : height of the resonator
;p13        : diffusion coefficient
;p14        : power user values stored in this function

;         decay  hidecay rx ry rz rdistance direct early_diff
f1  0 8 -2  0.95   0.95   0  0  0    0.3     0.5      0.8  ; brighter
f2  0 8 -2  0.95   0.5    0  0  0    0.3     0.5      0.8  ; default (to be set as)
f3  0 8 -2  0.95   0.01   0  0  0    0.3     0.5      0.8  ; darker
f4  0 8 -2  0.95   0.7    0  0  0    0.3     0.1      0.4  ; to hear the effect of diffusion
f5  0 8 -2  0.9    0.5    0  0  0    0.3     2.0      0.98 ; to hear the movement
f6  0 8 -2  0.99   0.1    0  0  0    0.3     0.5      0.8  ; default vals
;        ^
;         ----- gen. number: negative to avoid rescaling


i2 0 10  6  4  3   6   4  3  14.39  11.86  10    1  6 ; defaults
i2 +  4  6  4  3   6   4  3  14.39  11.86  10    1  1 ; hear brightness 1
i2 +  4  6  4  3  -6  -4  3  14.39  11.86  10    1  2 ; hear brightness 2
i2 +  4  6  4  3  -6  -4  3  14.39  11.86  10    1  3 ; hear brightness 3
i2 +  3 .6 .4 .3 -.6 -.4 .3  1.439  1.186  1.0 0.0  4 ; hear diffusion 1
i2 +  3 .6 .4 .3 -.6 -.4 .3  1.439  1.186  1.0 1.0  4 ; hear diffusion 2
i2 +  4 12  4  3 -12  -4 -3  24.39  21.86  20    1  5 ; hear movement
;
i2 +  4  6  4  3   6   4  3  14.39  11.86   10   1  1 ; hear brightness 1
i2 +  4  6  4  3  -6  -4  3  14.39  11.86   10   1  2 ; hear brightness 2
i2 +  4  6  4  3  -6  -4  3  14.39  11.86   10   1  3 ; hear brightness 3
i2 +  3 .6 .4 .3 -.6 -.4 .3  1.439  1.186  1.0 0.0  4 ; hear diffusion 1
i2 +  3 .6 .4 .3 -.6 -.4 .3  1.439  1.186  1.0 1.0  4 ; hear diffusion 2
i2 +  4 12  4  3 -12  -4 -3  24.39  21.86   20   1  5 ; hear movement
;       ^^^^^^^^^^^^^^^^^^^  ^^^^^^^^^^^^^^^^^   ^  ^
;       |||||||||||||||||||  |||||||||||||||||   |   --: expert values function
;       |||||||||||||||||||  |||||||||||||||||   +--: diffusion
;       |||||||||||||||||||  ----------------: optimal room dims according to Milner and Bernard JASA 85(2), 1989
;       |||||||||||||||||||
;       --------------------: source position start and end
e
/* babo_expert.sco */
        
\end{lstlisting}
\subsection*{Credits}


 


 


\begin{tabular}{ccc}
Author: Paolo Filippi &Padova, Italy &1999

\end{tabular}



 


 


 


\begin{tabular}{ccc}
Nicola Bernardini &Rome, Italy &2000

\end{tabular}



 


 New in Csound version 4.09
%\hline 


\begin{comment}
\begin{tabular}{lcr}
Previous &Home &Next \\
aweibull &Up &balance

\end{tabular}


\end{document}
\end{comment}
