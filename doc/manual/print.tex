\begin{comment}
\documentclass[10pt]{article}
\usepackage{fullpage, graphicx, url}
\setlength{\parskip}{1ex}
\setlength{\parindent}{0ex}
\title{print}
\begin{document}


\begin{tabular}{ccc}
The Alternative Csound Reference Manual & & \\
Previous & &Next

\end{tabular}

%\hline 
\end{comment}
\section{print}
print�--� Displays the values init, control, or audio signals. \subsection*{Description}


  These units will print orchestra init-values, or produce graphic display of orchestra control signals and audio signals. Uses X11 windows if enabled, else (or if \emph{-g}
 flag is set) displays are approximated in ASCII characters. 
\subsection*{Syntax}


 \textbf{print}
 iarg [, iarg1] [, iarg2] [...]
\subsection*{Initialization}


 \emph{iarg, iarg2, ... }
 -- i-rate arguments. 
\subsection*{Performance}


 \emph{print}
 -- print the current value of the i-time arguments (or expressions) \emph{iarg}
 at every i-pass through the instrument. 
\subsection*{Examples}


  Here is an example of the print opcode. It uses the files \emph{print.orc}
 and \emph{print.sco}
. 


 \textbf{Example 1. Example of the print opcode.}

\begin{lstlisting}
/* print.orc */
; Initialize the global variables.
sr = 44100
kr = 4410
ksmps = 10
nchnls = 1

; Instrument #1.
instr 1
  ; Print the fourth p-field.
  print p4
endin
/* print.orc */
        
\end{lstlisting}
\begin{lstlisting}
/* print.sco */
; p4 = value to be printed.
; Play Instrument #1 for one second, p4 = 50.375.
i 1 0 1 50.375
; Play Instrument #1 for one second, p4 = 300.
i 1 1 1 300
; Play Instrument #1 for one second, p4 = -999.
i 1 2 1 -999
e
/* print.sco */
        
\end{lstlisting}
 Its output should include lines like this: \begin{lstlisting}
instr 1:  p4 = 50.375
instr 1:  p4 = 300.000
instr 1:  p4 = -999.000
      
\end{lstlisting}
\subsection*{See Also}


 \emph{dispfft}
, \emph{display}
, \emph{printk}
, \emph{printk2}
, \emph{printks}
 and \emph{prints}

\subsection*{Credits}


 Example written by Kevin Conder.


 Comments about the \emph{inprds}
 parameter contributed by Rasmus Ekman.
%\hline 


\begin{comment}
\begin{tabular}{lcr}
Previous &Home &Next \\
prealloc &Up &printk

\end{tabular}


\end{document}
\end{comment}
