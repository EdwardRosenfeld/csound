\begin{comment}
\documentclass[10pt]{article}
\usepackage{fullpage, graphicx, url}
\setlength{\parskip}{1ex}
\setlength{\parindent}{0ex}
\title{readclock}
\begin{document}


\begin{tabular}{ccc}
The Alternative Csound Reference Manual & & \\
Previous & &Next

\end{tabular}

%\hline 
\end{comment}
\section{readclock}
readclock�--� Reads the value of an internal clock. \subsection*{Description}


  Reads the value of an internal clock. 
\subsection*{Syntax}


 ir \textbf{readclock}
 inum
\subsection*{Initialization}


 \emph{inum}
 -- the number of a clock. There are 32 clocks numbered 0 through 31. All other values are mapped to clock number 32. 


 \emph{ir}
 -- value at i-time, of the clock specified by \emph{inum}

\subsection*{Performance}


  Between a \emph{clockon}
 and a \emph{clockoff}
 opcode, the CPU time used is accumulated in the clock. The precision is machine dependent but is the millisecond range on UNIX and Windows systems. The \emph{readclock}
 opcde reads the current value of a clock at initialization time. 
\subsection*{Examples}


  Here is an example of the readclock opcode. It uses the files \emph{readclock.orc}
 and \emph{readclock.sco}
. 


 \textbf{Example 1. Example of the readclock opcode.}

\begin{lstlisting}
/* readclock.orc */
; Initialize the global variables.
sr = 44100
kr = 44100
ksmps = 1
nchnls = 1

; Instrument #1.
instr 1
  ; Start clock #1.
  clockon 1
  ; Do something that keeps Csound busy.
  a1 oscili 10000, 440, 1
  out a1
  ; Stop clock #1.
  clockoff 1
  ; Print the time accumulated in clock #1.
  i1 readclock 1
  print i1
endin
/* readclock.orc */
        
\end{lstlisting}
\begin{lstlisting}
/* readclock.sco */
; Initialize the function tables.
; Table 1: an ordinary sine wave.
f 1 0 32768 10 1

; Play Instrument #1 for one second starting at 0:00.
i 1 0 1
; Play Instrument #1 for one second starting at 0:01.
i 1 1 1
; Play Instrument #1 for one second starting at 0:02.
i 1 2 1
e
/* readclock.sco */
        
\end{lstlisting}
 Its output should include lines like this: \begin{lstlisting}
instr 1:  i1 = 0.000
instr 1:  i1 = 90.000
instr 1:  i1 = 180.000
      
\end{lstlisting}
\subsection*{See Also}


 \emph{clockoff}
, \emph{clockon}

\subsection*{Credits}


 


 


\begin{tabular}{cccc}
Author: John ffitch &University of Bath/Codemist Ltd. &Bath, UK &July, 1999

\end{tabular}



 


 Example written by Kevin Conder.


 New in Csound version 3.56
%\hline 


\begin{comment}
\begin{tabular}{lcr}
Previous &Home &Next \\
randomi &Up &readk

\end{tabular}


\end{document}
\end{comment}
