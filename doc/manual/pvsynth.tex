\begin{comment}
\documentclass[10pt]{article}
\usepackage{fullpage, graphicx, url}
\setlength{\parskip}{1ex}
\setlength{\parindent}{0ex}
\title{pvsynth}
\begin{document}


\begin{tabular}{ccc}
The Alternative Csound Reference Manual & & \\
Previous & &Next

\end{tabular}

%\hline 
\end{comment}
\section{pvsynth}
pvsynth�--� Resynthesise using a FFT overlap-add. \subsection*{Description}


  Resynthesise using a FFT overlap-add. 
\subsection*{Syntax}


 ar \textbf{pvsynth}
 fsrc, [iinit]
\subsection*{Performance}


 \emph{ar}
 -- output audio signal 


 \emph{fsrc}
 -- input signal 


 \emph{iinit}
 -- not yet implemented. 
\subsection*{Examples}


 


 \textbf{Example 1. Example (using score-supplied f-table, assuming fsig fftsize = 1024)}

\begin{lstlisting}
; score f-table using cubic spline to define shaped peaks
f1 0 513 8 0 2 1 3 0 4 1 6 0 10 1 12 0 16 1 32 0 1 0 436 0
 
asig  buzz     20000,199,50,3        ; pulsewave source
fsig  pvsanal  asig,1024,256,1024,0  ; create fsig
kmod  linseg   0,p3/2,1,p3/2,0       ; simple control sig
 
fsig  pvsmaska fsig,2,kmod           ; apply weird eq to fsig
aout  pvsynth  fsig                  ; resynthesize,
      dispfft  aout,0.1,1024         ; and view the effect
        
\end{lstlisting}
 This also illustrates that the usual Csound behaviour applies to fsigs; the same name can be used for both input and output. \subsection*{See Also}


 \emph{pvsadsyn}

\subsection*{Credits}


 


 


\begin{tabular}{cc}
Author: Richard Dobson &August 2001 

\end{tabular}



 


 New in version 4.13


 February 2004. Thanks to a note from Francisco Vila, updated the example.
%\hline 


\begin{comment}
\begin{tabular}{lcr}
Previous &Home &Next \\
pvsmaska &Up &rand

\end{tabular}


\end{document}
\end{comment}
