\begin{comment}
\documentclass[10pt]{article}
\usepackage{fullpage, graphicx, url}
\setlength{\parskip}{1ex}
\setlength{\parindent}{0ex}
\title{peak}
\begin{document}


\begin{tabular}{ccc}
The Alternative Csound Reference Manual & & \\
Previous & &Next

\end{tabular}

%\hline 
\end{comment}
\section{peak}
peak�--� Maintains the output equal to the highest absolute value received. \subsection*{Description}


  These opcodes maintain the output k-rate variable as the peak absolute level so far received. 
\subsection*{Syntax}


 kr \textbf{peak}
 asig


 kr \textbf{peak}
 ksig
\subsection*{Performance}


 \emph{kr}
 -- Output equal to the highest absolute value received so far. This is effectively an input to the opcode as well, since it reads \emph{kr}
 in order to decide whether to write something higher into it. 


 \emph{ksig }
 -- k-rate input signal. 


 \emph{asig }
 -- a-rate input signal. 
\subsection*{Examples}


  Here is an example of the peak opcode. It uses the files \emph{peak.orc}
, \emph{peak.sco}
, and \emph{beats.wav}
. 


 \textbf{Example 1. Example of the peak opcode.}

\begin{lstlisting}
/* peak.orc */
; Initialize the global variables.
sr = 44100
kr = 44100
ksmps = 1
nchnls = 1

; Instrument #1 - play an audio file.
instr 1
  ; Capture the highest amplitude in the "beats.wav" file.
  asig soundin "beats.wav"
  kp peak asig

  ; Print out the peak value once per second.
  printk 1, kp
  
  out asig
endin
/* peak.orc */
        
\end{lstlisting}
\begin{lstlisting}
/* peak.sco */
; Play Instrument #1, the audio file, for three seconds.
i 1 0 3
e
/* peak.sco */
        
\end{lstlisting}
 Its output should include lines like this: \begin{lstlisting}
 i   1 time     0.00002:  4835.00000
 i   1 time     1.00002: 29312.00000
 i   1 time     2.00002: 32767.00000
      
\end{lstlisting}
\subsection*{Credits}


 


 


\begin{tabular}{ccc}
Author: Robin Whittle &Australia &May 1997

\end{tabular}



 


 Example written by Kevin Conder.
%\hline 


\begin{comment}
\begin{tabular}{lcr}
Previous &Home &Next \\
pchoct &Up &peakk

\end{tabular}


\end{document}
\end{comment}
