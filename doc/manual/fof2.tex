\begin{comment}
\documentclass[10pt]{article}
\usepackage{fullpage, graphicx, url}
\setlength{\parskip}{1ex}
\setlength{\parindent}{0ex}
\title{fof2}
\begin{document}


\begin{tabular}{ccc}
The Alternative Csound Reference Manual & & \\
Previous & &Next

\end{tabular}

%\hline 
\end{comment}
\section{fof2}
fof2�--� Produces sinusoid bursts including k-rate incremental indexing with each successive burst. \subsection*{Description}


  Audio output is a succession of sinusoid bursts initiated at frequency \emph{xfund}
 with a spectral peak at \emph{xform}
. For \emph{xfund}
 above 25 Hz these bursts produce a speech-like formant with spectral characteristics determined by the k-input parameters. For lower fundamentals this generator provides a special form of granular synthesis. 


 \emph{fof2}
 implements k-rate incremental indexing into \emph{ifna}
 function with each successive burst. 
\subsection*{Syntax}


 ar \textbf{fof2}
 xamp, xfund, xform, koct, kband, kris, kdur, kdec, iolaps, ifna, ifnb, itotdur, kphs, kgliss [, iskip]
\subsection*{Initialization}


 \emph{iolaps}
 -- number of preallocated spaces needed to hold overlapping burst data. Overlaps are frequency dependent, and the space required depends on the maximum value of \emph{xfund * kdur}
. Can be over-estimated at no computation cost. Uses less than 50 bytes of memory per \emph{iolap}
. 


 \emph{ifna, ifnb}
 -- table numbers of two stored functions. The first is a sine table for sineburst synthesis (size of at least 4096 recommended). The second is a rise shape, used forwards and backwards to shape the sineburst rise and decay; this may be linear (\emph{GEN07}
) or perhaps a sigmoid (\emph{GEN19}
). 


 \emph{itotdur}
 -- total time during which this \emph{fof}
 will be active. Normally set to p3. No new sineburst is created if it cannot complete its \emph{kdur}
 within the remaining \emph{itotdur}
. 


 \emph{iskip}
 (optional, default=0) -- If non-zero, skip initialization (allows legato use). 
\subsection*{Performance}


 \emph{xamp}
 -- peak amplitude of each sineburst, observed at the true end of its rise pattern. The rise may exceed this value given a large bandwidth (say, Q $<$ 10) and/or when the bursts are overlapping. 


 \emph{xfund}
 -- the fundamental frequency (in Hertz) of the impulses that create new sinebursts. 


 \emph{xform}
 -- the formant frequency, i.e. freq of the sinusoid burst induced by each \emph{xfund}
 impulse. This frequency can be fixed for each burst or can vary continuously (see \emph{ifmode}
). 


 \emph{koct}
 -- octaviation index, normally zero. If greater than zero, lowers the effective \emph{xfund}
 frequency by attenuating odd-numbered sinebursts. Whole numbers are full octaves, fractions transitional. 


 \emph{kband}
 -- the formant bandwidth (at -6dB), expressed in Hz. The bandwidth determines the rate of exponential decay throughout the sineburst, before the enveloping described below is applied. 


 \emph{kris, kdur, kdec}
 -- rise, overall duration, and decay times (in seconds) of the sinusoid burst. These values apply an enveloped duration to each burst, in similar fashion to a Csound \emph{linen}
 generator but with rise and decay shapes derived from the \emph{ifnb}
 input. \emph{kris}
 inversely determines the skirtwidth (at -40 dB) of the induced formant region. \emph{kdur}
 affects the density of sineburst overlaps, and thus the speed of computation. Typical values for vocal imitation are .003,.02,.007. 


 \emph{kphs}
 -- allows k-rate indexing of function table \emph{ifna}
 with each successive burst, making it suitable for time-warping applications. Values of for \emph{kphs}
 are normalized from 0 to 1, 1 being the end of the function table \emph{ifna}
. 


 \emph{kgliss}
 -- sets the end pitch of each grain relative to the initial pitch, in octaves. Thus \emph{kgliss}
 = 2 means that the grain ends two octaves above its initial pitch, while \emph{kgliss}
 = -5/3 has the grain ending a perfect major sixth below. \emph{Note}
: There are no optional parameters in \emph{fof2}



  Csound's \emph{fof}
 generator is loosely based on Michael Clarke's C-coding of IRCAM's \emph{CHANT}
 program (Xavier Rodet et al.). Each fof produces a single formant, and the output of four or more of these can be summed to produce a rich vocal imitation. \emph{fof}
 synthesis is a special form of granular synthesis, and this implementation aids transformation between vocal imitation and granular textures. Computation speed depends on \emph{kdur, xfund}
, and the density of any overlaps. 
\subsection*{See Also}


 \emph{fof}

\subsection*{Credits}


 


 


\begin{tabular}{cc}
Author: Rasmus Ekman &\emph{fof2}
 is a modification of \emph{fof}
 by Rasmus Ekman

\end{tabular}



 


 New in Csound 3.45
%\hline 


\begin{comment}
\begin{tabular}{lcr}
Previous &Home &Next \\
fof &Up &fog

\end{tabular}


\end{document}
\end{comment}
