\begin{comment}
\documentclass[10pt]{article}
\usepackage{fullpage, graphicx, url}
\setlength{\parskip}{1ex}
\setlength{\parindent}{0ex}
\title{vco2}
\begin{document}


\begin{tabular}{ccc}
The Alternative Csound Reference Manual & & \\
Previous & &Next

\end{tabular}

%\hline 
\end{comment}
\section{vco2}
vco2�--� Implementation of a band-limited oscillator using pre-calculated tables. \subsection*{Description}


 \emph{vco2}
 is similar to \emph{vco}
. But the implementation uses pre-calculated tables of band-limited waveforms (see also \emph{GEN30}
) rather than integrating impulses. This opcode can be faster than \emph{vco}
 (especially if a low control-rate is used) and also allows better sound quality. Additionally, there are more waveforms and oscillator phase can be modulated at k-rate. The disadvantage is increased memory usage. For more details about vco2 tables, see also \emph{vco2init}
 and \emph{vco2ft}
. 
\subsection*{Syntax}


 ar \textbf{vco2}
 kamp, kcps [, imode] [, kpw] [, kphs] [, inyx]
\subsection*{Initialization}


 \emph{imode}
 (optional, default=0) -- a sum of values representing the waveform and its control values. 


  One may use of any of the following values for \emph{imode}
: 


 
\begin{itemize}
\item 

 16: enable k-rate phase control (if set, kphs is a required k-rate parameter that allows phase modulation)

\item 

 1: skip initialization


\end{itemize}


  One may use exactly one of these \emph{imode}
 values to select the waveform to be generated: 


 
\begin{itemize}
\item 

 14: user defined waveform -1 (requires using the \emph{vco2init}
 opcode)

\item 

 12: triangle (no ramp, faster)

\item 

 10: square wave (no PWM, faster)

\item 

 8: 4 * x * (1 - x) (i.e. integrated sawtooth)

\item 

 6: pulse (not normalized)

\item 

 4: sawtooth / triangle / ramp

\item 

 2: square / PWM

\item 

 0: sawtooth


\end{itemize}


  The default value for \emph{imode}
 is zero, which means a sawtooth wave with no k-rate phase control. 


 \emph{inyx}
 (optional, default=0.5) -- bandwidth of the generated waveform, as percentage (0 to 1) of the sample rate. The expected range is 0 to 0.5 (i.e. up to \emph{sr}
/2), other values are limited to the allowed range. 


  Setting inyx to 0.25 (sr/4), or 0.3333 (sr/3) can produce a ``fatter'' sound in some cases, although it is more likely to reduce quality. 
\subsection*{Performance}


 \emph{ar }
 -- the output audio signal. 


 \emph{kamp}
 -- amplitude scale. In the case of a \emph{imode}
 waveform value of 6 (a pulse waveform), the actual output level can be a lot higher than this value. 


 \emph{kcps}
 -- frequency in Hz (should be in the range -sr/2 to sr/2). 


 \emph{kpw}
 (optional) -- the pulse width of the square wave (\emph{imode}
 waveform=2) or the ramp characteristics of the triangle wave (\emph{imode}
 waveform=4). It is required only by these waveforms and ignored in all other cases. The expected range is 0 to 1, any other value is wrapped to the allowed range. 


 


\begin{tabular}{cc}
Warning &\textbf{Warning}
 \\
� &

 \emph{kpw}
 must not be an exact integer value (e.g. 0 or 1) if a sawtooth / triangle ramp (\emph{imode}
 waveform=4) is generated. In this case, the recommended range is about 0.01 to 0.99). There is no such limitation for a square/PWM waveform. 


\end{tabular}



 \emph{kphs}
 (optional) -- oscillator phase (depending on \emph{imode}
, this can be either an optional i-rate parameter that defaults to zero or required k-rate). Similarly to \emph{kpw}
, the expected range is 0 to 1. 


 


\begin{tabular}{cc}
\textbf{Note}
 \\
� &

  When a low control-rate is used, pulse width (\emph{kpw}
) and phase (\emph{kphs}
) modulation is internally converted to frequency modulation. This allows for faster processing and reduced artifacts. But in the case of very long notes and continuous fast changes in \emph{kpw}
 or \emph{kphs}
, the phase may drift away from the requested value. In most cases, the phase error is at most 0.037 per hour (assuming a sample rate of 44100 Hz). 


  This is a problem mainly in the case of pulse width (\emph{kpw}
), where it may result in various artifacts. While future releases of \emph{vco2}
 may fix such errors, the following work-arounds may also be of some help: 


 
\begin{itemize}
\item 

 Use \emph{kpw}
 values only in the range 0.05 to 0.95. (There are more artifacts around integer values)

\item 

 Try to avoid modulating \emph{kpw}
 by asymmetrical waveforms like a sawtooth wave. Relatively slow ($<$= 20 Hz) symmetrical modulation (e.g. sine or triangle), random splines (also slow), or a fixed pulse width is a lot less likely to cause synchronization problems.

\item 

 In some cases, adding random jitter (for example: random splines with an amplitude of about 0.01) to \emph{kpw}
 may also fix the problem.


\end{itemize}


\end{tabular}

\subsection*{Examples}


  Here is an example of the vco2 opcode. It uses the files \emph{vco2.orc}
 and \emph{vco2.sco}
. 


 \textbf{Example 1. Example of the vco2 opcode.}

\begin{lstlisting}
/* vco2.orc */
sr      =  44100
ksmps   =  10
nchnls  =  1

; user defined waveform -1: trapezoid wave with default parameters (can be
; accessed at ftables starting from 10000)
itmp    ftgen 1, 0, 16384, 7, 0, 2048, 1, 4096, 1, 4096, -1, 4096, -1, 2048, 0
ift     vco2init -1, 10000, 0, 0, 0, 1
; user defined waveform -2: fixed table size (4096), number of partials
; multiplier is 1.02 (~238 tables)
itmp    ftgen 2, 0, 16384, 7, 1, 4095, 1, 1, -1, 4095, -1, 1, 0, 8192, 0
ift     vco2init -2, ift, 1.02, 4096, 4096, 2

        instr 1
kcps    expon p4, p3, p5                ; instr 1: basic vco2 example
a1      vco2 12000, kcps                ; (sawtooth wave with default
        out a1                          ; parameters)
        endin

        instr 2
kcps    expon p4, p3, p5                        ; instr 2:
kpw     linseg 0.1, p3/2, 0.9, p3/2, 0.1        ; PWM example
a1      vco2 10000, kcps, 2, kpw
        out a1
        endin

        instr 3
kcps    expon p4, p3, p5                ; instr 3: vco2 with user
a1      vco2 14000, kcps, 14            ; defined waveform (-1)
aenv    linseg 1, p3 - 0.1, 1, 0.1, 0   ; de-click envelope
        out a1 * aenv
        endin

        instr 4
kcps    expon p4, p3, p5                ; instr 4: vco2ft example,
kfn     vco2ft kcps, -2, 0.25           ; with user defined waveform
a1      oscilikt 12000, kcps, kfn       ; (-2), and sr/4 bandwidth
        out a1
        endin
/* vco2.orc */
        
\end{lstlisting}
\begin{lstlisting}
/* vco2.sco */
i 1  0 3 20 2000
i 2  4 2 200 400
i 3  7 3 400 20
i 4 11 2 100 200

f 0 14

e
/* vco2.sco */
        
\end{lstlisting}
\subsection*{See Also}


 \emph{vco}
, \emph{vco2ft}
, \emph{vco2ift}
, and \emph{vco2init}
. 
\subsection*{Credits}


 Author: Istvan Varga


 New in version 4.22
%\hline 


\begin{comment}
\begin{tabular}{lcr}
Previous &Home &Next \\
vco &Up &vco2ft

\end{tabular}


\end{document}
\end{comment}
