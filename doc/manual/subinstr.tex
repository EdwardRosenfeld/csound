\begin{comment}
\documentclass[10pt]{article}
\usepackage{fullpage, graphicx, url}
\setlength{\parskip}{1ex}
\setlength{\parindent}{0ex}
\title{subinstr}
\begin{document}


\begin{tabular}{ccc}
The Alternative Csound Reference Manual & & \\
Previous & &Next

\end{tabular}

%\hline 
\end{comment}
\section{subinstr}
subinstr�--� Creates and runs a numbered instrument instance. \subsection*{Description}


  Creates an instance of another instrument and is used as if it were an opcode. 
\subsection*{Syntax}


 a1, [...] [, a8] \textbf{subinstr}
 instrnum [, p4] [, p5] [...]


 a1, [...] [, a8] \textbf{subinstr}
 ``insname'' [, p4] [, p5] [...]
\subsection*{Initialization}


 \emph{instrnum}
 -- Number of the instrument to be called. 


 \emph{``insname''}
 -- A string (in double-quotes) representing a named instrument. 


  For more information about specifying input and output interfaces, see \emph{Calling an Instrument within an Instrument}
. 
\subsection*{Performance}


 \emph{a1, ..., a8}
 -- The audio output from the called instrument. This is generated using the \emph{signal output}
 opcodes. 


 \emph{p4, p5, ...}
 -- Additional input values the are mapped to the called instrument p-fields, starting with p4. 


  The called instrument's p2 and p3 values will be identical to the host instrument's values. While the host instrument can \emph{control its own duration}
, any such attempts inside the called instrument will most likely have no effect. 
\subsection*{See Also}


 \emph{Calling an Instrument within an Instrument}
, \emph{event}
, \emph{schedule}
, \emph{subinstrinit}

\subsection*{Examples}


  Here is an example of the subinstr opcode. It uses the files \emph{subinstr.orc}
 and \emph{subinstr.sco}
. 


 \textbf{Example 1. Example of the subinstr opcode.}

\begin{lstlisting}
/* subinstr.orc */
; Initialize the global variables.
sr = 44100
kr = 4410
ksmps = 10
nchnls = 1

; Instrument #1 - Creates a basic tone.
instr 1
  ; Print the value of p4, should be equal to
  ; Instrument #2's iamp field.
  print p4

  ; Print the value of p5, should be equal to
  ; Instrument #2's ipitch field.
  print p5

  ; Create a tone.
  asig oscils p4, p5, 0

  out asig
endin


; Instrument #2 - Demonstrates the subinstr opcode.
instr 2
  iamp = 20000
  ipitch = 440

  ; Use Instrument #1 to create a basic sine-wave tone.
  ; Its p4 parameter will be set using the iamp variable.
  ; Its p5 parameter will be set using the ipitch variable.
  abasic subinstr 1, iamp, ipitch

  ; Output the basic tone that we have created.
  out abasic
endin
/* subinstr.orc */
        
\end{lstlisting}
\begin{lstlisting}
/* subinstr.sco */
; Table #1, a sine wave.
f 1 0 16384 10 1

; Play Instrument #2 for one second.
i 2 0 1
e
/* subinstr.sco */
        
\end{lstlisting}


  Here is an example of the subinstr opcode using a named instrument. It uses the files \emph{subinstr\_named.orc}
 and \emph{subinstr\_named.sco}
. 


 \textbf{Example 2. Example of the subinstr opcode using a named instrument.}

\begin{lstlisting}
/* subinstr_named.orc */
; Initialize the global variables.
sr = 44100
kr = 4410
ksmps = 10
nchnls = 1

; Instrument "basic_tone" - Creates a basic tone.
instr basic_tone
  ; Print the value of p4, should be equal to
  ; Instrument #2's iamp field.
  print p4

  ; Print the value of p5, should be equal to
  ; Instrument #2's ipitch field.
  print p5

  ; Create a tone.
  asig oscils p4, p5, 0

  out asig
endin


; Instrument #1 - Demonstrates the subinstr opcode.
instr 1
  iamp = 20000
  ipitch = 440

  ; Use the "basic_tone" named instrument to create a 
  ; basic sine-wave tone.
  ; Its p4 parameter will be set using the iamp variable.
  ; Its p5 parameter will be set using the ipitch variable.
  abasic subinstr "basic_tone", iamp, ipitch

  ; Output the basic tone that we have created.
  out abasic
endin
/* subinstr_named.orc */
        
\end{lstlisting}
\begin{lstlisting}
/* subinstr_named.sco */
; Table #1, a sine wave.
f 1 0 16384 10 1

; Play Instrument #1 for one second.
i 1 0 1
e
/* subinstr_named.sco */
        
\end{lstlisting}
\subsection*{Credits}


 New in version 4.21
%\hline 


\begin{comment}
\begin{tabular}{lcr}
Previous &Home &Next \\
strset &Up &subinstrinit

\end{tabular}


\end{document}
\end{comment}
