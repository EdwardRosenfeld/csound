\begin{comment}
\documentclass[10pt]{article}
\usepackage{fullpage, graphicx, url}
\setlength{\parskip}{1ex}
\setlength{\parindent}{0ex}
\title{log10}
\begin{document}


\begin{tabular}{ccc}
The Alternative Csound Reference Manual & & \\
Previous & &Next

\end{tabular}

%\hline 
\end{comment}
\section{log10}
log10�--� Returns a base 10 log. \subsection*{Description}


  Returns the base 10 log of \emph{x}
 (\emph{x}
 positive only). 


  The argument value is restricted for \emph{log}
, \emph{log10}
, and \emph{sqrt}
. 
\subsection*{Syntax}


 \textbf{log10}
(x) (no rate restriction)


  where the argument within the parentheses may be an expression. Value converters perform arithmetic translation from units of one kind to units of another. The result can then be a term in a further expression. 
\subsection*{Examples}


  Here is an example of the log10 opcode. It uses the files \emph{log10.orc}
 and \emph{log10.sco}
. 


 \textbf{Example 1. Example of the log10 opcode.}

\begin{lstlisting}
/* log10.orc */
; Initialize the global variables.
sr = 44100
kr = 4410
ksmps = 10
nchnls = 1

; Instrument #1.
instr 1
  i1 = log10(8)
  print i1
endin
/* log10.orc */
        
\end{lstlisting}
\begin{lstlisting}
/* log10.sco */
; Play Instrument #1 for one second.
i 1 0 1
e
/* log10.sco */
        
\end{lstlisting}
 Its output should include a line like this: \begin{lstlisting}
instr 1:  i1 = 0.903
      
\end{lstlisting}
\subsection*{See Also}


 \emph{abs}
, \emph{exp}
, \emph{frac}
, \emph{int}
, \emph{log}
, \emph{i}
, \emph{sqrt}

\subsection*{Credits}


 Example written by Kevin Conder.
%\hline 


\begin{comment}
\begin{tabular}{lcr}
Previous &Home &Next \\
log &Up &logbtwo

\end{tabular}


\end{document}
\end{comment}
