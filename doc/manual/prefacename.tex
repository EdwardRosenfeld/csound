\begin{comment}
\documentclass[10pt]{article}
\usepackage{fullpage, graphicx, url}
\setlength{\parskip}{1ex}
\setlength{\parindent}{0ex}
\title{Why is this called the Alternative Csound Reference Manual?}
\begin{document}


\begin{tabular}{ccc}
The Alternative Csound Reference Manual & & \\
Previous &Preface &Next

\end{tabular}

%\hline 
\end{comment}
\section{Why is this called the \emph{Alternative}
 Csound Reference Manual?}


  When I originally started my manual project, there was already an Official Csound Reference Manual (last known address: \emph{\url{http://www.lakewoodsound.com/csound/hypertext/manual.htm}}
). The Official manual was maintained by David M. Boothe. I found its layout confusing and I wanted to change it. But since it was maintained with a commercial word processing program, I couldn't. I could neither afford this program nor was it available for my main computing platform. 


  So I created an alternative to the Official Csound Reference Manual. I distributed my manual using the \emph{DocBook/SGML}
 format so that anyone on any platform could edit it with a text editor. This manual can also be produced with freely available programs. 


  David M. Boothe wasn't interested in maintaining my DocBook/SGML version of the manual. He was also concerned that people would confuse his project (the ``Official'' one) with mine. So out of respect for his wishes, I named my project the Alternative Csound Reference Manual. I made this decision so that nobody would confuse my project (the ``Alternative'' one) with his. 


  Written by Kevin Conder, October 2002. 
%\hline 


\begin{comment}
\begin{tabular}{lcr}
Previous &Home &Next \\
Acknowledgements &Up &What is the scope of the Alternative Csound Reference Manual?

\end{tabular}


\end{document}
\end{comment}
