\begin{comment}
\documentclass[10pt]{article}
\usepackage{fullpage, graphicx, url}
\setlength{\parskip}{1ex}
\setlength{\parindent}{0ex}
\title{db}
\begin{document}


\begin{tabular}{ccc}
The Alternative Csound Reference Manual & & \\
Previous & &Next

\end{tabular}

%\hline 
\end{comment}
\section{db}
db�--� Returns the amplitude equivalent for a given decibel amount. \subsection*{Description}


  Returns the amplitude equivalent for a given decibel amount. This opcode is the same as \emph{db}
. 
\subsection*{Syntax}


 \textbf{db}
(x)


  This function works at a-rate, i-rate, and k-rate. 
\subsection*{Initialization}


 \emph{x}
 -- a value expressed in decibels. 
\subsection*{Performance}


  Returns the amplitude for a given decibel amount. 
\subsection*{Examples}


  Here is an example of the db opcode. It uses the files \emph{db.orc}
 and \emph{db.sco}
. 


 \textbf{Example 1. Example of the db opcode.}

\begin{lstlisting}
/* db.orc */
; Initialize the global variables.
sr = 44100
kr = 4410
ksmps = 10
nchnls = 1

; Instrument #1.
instr 1
  ; Calculate the amplitude of 40 decibels.
  idecibels = 40
  iamp = db(idecibels)

  print iamp
endin
/* db.orc */
        
\end{lstlisting}
\begin{lstlisting}
/* db.sco */
; Play Instrument #1 for one second.
i 1 0 1
e
/* db.sco */
        
\end{lstlisting}
 Its output should include lines like: \begin{lstlisting}
instr 1:  iamp = 100.000
      
\end{lstlisting}
\subsection*{See Also}


 \emph{ampdb}
, \emph{cent}
, \emph{octave}
, \emph{semitone}

\subsection*{Credits}


 Example written by Kevin Conder.


 New in version 4.16
%\hline 


\begin{comment}
\begin{tabular}{lcr}
Previous &Home &Next \\
dam &Up &dbamp

\end{tabular}


\end{document}
\end{comment}
