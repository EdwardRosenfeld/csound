\begin{comment}
\documentclass[10pt]{article}
\usepackage{fullpage, graphicx, url}
\setlength{\parskip}{1ex}
\setlength{\parindent}{0ex}
\title{streson}
\begin{document}


\begin{tabular}{ccc}
The Alternative Csound Reference Manual & & \\
Previous & &Next

\end{tabular}

%\hline 
\end{comment}
\section{streson}
streson�--� A string resonator with variable fundamental frequency. \subsection*{Description}


  An audio signal is modified by a string resonator with variable fundamental frequency. 
\subsection*{Syntax}


 ar \textbf{streson}
 asig, kfr, ifdbgain
\subsection*{Initialization}


 \emph{ifdbgain}
 -- feedback gain, between 0 and 1, of the internal delay line. A value close to 1 creates a slower decay and a more pronounced resonance. Small values may leave the input signal unaffected. Depending on the filter frequency, typical values are $>$ .9. 
\subsection*{Performance}


 \emph{asig}
 -- the input audio signal. 


 \emph{kfr}
 -- the fundamental frequency of the string. 


 \emph{streson}
 passes the input \emph{asig}
 through a network composed of comb, low-pass and all-pass filters, similar to the one used in some versions of the Karplus-Strong algorithm, creating a string resonator effect. The fundamental frequency of the ``string'' is controlled by the k-rate variable \emph{kfr}
.This opcode can be used to simulate sympathetic resonances to an input signal. 


 \emph{streson}
 is an adaptation of the StringFlt object of the SndObj Sound Object Library developed by the author. 
\subsection*{Examples}


  Here is an example of the streson opcode. It uses the files \emph{streson.orc}
 and \emph{streson.sco}
. 


 \textbf{Example 1. Example of the streson opcode.}

\begin{lstlisting}
/* streson.orc */
; Initialize the global variables.
sr = 44100
kr = 4410
ksmps = 10
nchnls = 1

; Instrument #1.
instr 1
  ; Generate a normal sine wave.
  asig oscils 8000, 440, 1

  ; Vary the fundamental frequency of the string 
  ; resonator linearly from 220 to 880 Hertz. 
  kfr line 220, p3, 880
  ifdbgain = 0.95

  ; Run our sine wave through the string resonator.
  astres streson asig, kfr, ifdbgain

  ; The resonance can get quite loud.
  ; So we'll clip the signal at 30,000.
  a1 clip astres, 1, 30000
  out a1
endin
/* streson.orc */
        
\end{lstlisting}
\begin{lstlisting}
/* streson.sco */
; Play Instrument #1 for five seconds.
i 1 0 5
e
/* streson.sco */
        
\end{lstlisting}
\subsection*{Credits}


 


 


\begin{tabular}{ccccc}
Author: Victor Lazzarini &Music Department &National University of Ireland, Maynooth &Maynooth, Co. Kildare &1998

\end{tabular}



 


 Example written by Kevin Conder.


 New in Csound version 3.494
%\hline 


\begin{comment}
\begin{tabular}{lcr}
Previous &Home &Next \\
stix &Up &strset

\end{tabular}


\end{document}
\end{comment}
