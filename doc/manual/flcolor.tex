\begin{comment}
\documentclass[10pt]{article}
\usepackage{fullpage, graphicx, url}
\setlength{\parskip}{1ex}
\setlength{\parindent}{0ex}
\title{FLcolor}
\begin{document}


\begin{tabular}{ccc}
The Alternative Csound Reference Manual & & \\
Previous & &Next

\end{tabular}

%\hline 
\end{comment}
\section{FLcolor}
FLcolor�--� A FLTK opcode that sets the primary colors. \subsection*{Description}


  Sets the primary colors to RGB values given by the user. 
\subsection*{Syntax}


 \textbf{FLcolor}
 ired, igreen, iblue
\subsection*{Initialization}


 \emph{ired}
 -- The red color of the target widget. The range for each RGB component is 0-255 


 \emph{igreen}
 -- The green color of the target widget. The range for each RGB component is 0-255 


 \emph{iblue}
 -- The blue color of the target widget. The range for each RGB component is 0-255 
\subsection*{Performance}


  These opcodes modify the appearance of other widgets. There are two types of such opcodes, those that don't contain the \emph{ihandle}
 argument which affect all subsequently declared widgets, and those without \emph{ihandle}
 which affect only a target widget previously defined. 


 \emph{FLcolor}
 sets the primary colors to RGB values given by the user. This opcode affects the primary color of (almost) all widgets defined next its location. User can put several instances of \emph{FLcolor}
 in front of each widget he intend to modify. However, to modify a single widget, it would be better to use the opcode belonging to the second type (i.e. those containing \emph{ihandle}
 argument). 


 \emph{FLcolor}
 is designed to modify the colors of a group of related widgets that assume the same color. The influence of \emph{FLcolor}
 on subsequent widgets can be turned off by using -1 as the only argument of the opcode. Also, using -2 (or -3) as the only value of \emph{FLcolor}
 makes all next widget colors randomly selected. The difference is that -2 selects a light random color, while -3 selects a dark random color. 
\subsection*{See Also}


 \emph{FLcolor2}
, \emph{FLhide}
, \emph{FLlabel}
, \emph{FLsetAlign}
, \emph{FLsetBox}
, \emph{FLsetColor}
, \emph{FLsetColor2}
, \emph{FLsetFont}
, \emph{FLsetPosition}
, \emph{FLsetSize}
, \emph{FLsetText}
, \emph{FLsetTextColor}
, \emph{FLsetTextSize}
, \emph{FLsetTextType}
, \emph{FLsetVal\_i}
, \emph{FLsetVal}
, \emph{FLshow}

\subsection*{Credits}


 Author: Gabriel Maldonado


 New in version 4.22
%\hline 


\begin{comment}
\begin{tabular}{lcr}
Previous &Home &Next \\
FLbutton &Up &FLcolor2

\end{tabular}


\end{document}
\end{comment}
