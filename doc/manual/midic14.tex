\begin{comment}
\documentclass[10pt]{article}
\usepackage{fullpage, graphicx, url}
\setlength{\parskip}{1ex}
\setlength{\parindent}{0ex}
\title{midic14}
\begin{document}


\begin{tabular}{ccc}
The Alternative Csound Reference Manual & & \\
Previous & &Next

\end{tabular}

%\hline 
\end{comment}
\section{midic14}
midic14�--� Allows a floating-point 14-bit MIDI signal scaled with a minimum and a maximum range. \subsection*{Description}


  Allows a floating-point 14-bit MIDI signal scaled with a minimum and a maximum range. 
\subsection*{Syntax}


 idest \textbf{midic14}
 ictlno1, ictlno2, imin, imax [, ifn]


 kdest \textbf{midic14}
 ictlno1, ictlno2, kmin, kmax [, ifn]
\subsection*{Initialization}


 \emph{idest}
 -- output signal 


 \emph{ictln1o}
 -- most-significant byte controller number (0-127) 


 \emph{ictlno2}
 -- least-significant byte controller number (0-127) 


 \emph{imin}
 -- user-defined minimum floating-point value of output 


 \emph{imax}
 -- user-defined maximum floating-point value of output 


 \emph{ifn}
 (optional) -- table to be read when indexing is required. Table must be normalized. Output is scaled according to \emph{imin}
 and \emph{imax}
 values. 
\subsection*{Performance}


 \emph{kdest}
 -- output signal 


 \emph{kmin}
 -- user-defined minimum floating-point value of output 


 \emph{kmax}
 -- user-defined maximum floating-point value of output 


 \emph{midic14}
 (i- and k-rate 14 bit MIDI control) allows a floating-point 14-bit MIDI signal scaled with a minimum and a maximum range. The minimum and maximum values can be varied at k-rate. It can use optional interpolated table indexing. It requires two MIDI controllers as input. 
\subsection*{See Also}


 \emph{ctrl7}
, \emph{ctrl14}
, \emph{ctrl21}
, \emph{initc7}
, \emph{initc14}
, \emph{initc21}
, \emph{midic7}
, \emph{midic21}

\subsection*{Credits}


 


 


\begin{tabular}{cc}
Author: Gabriel Maldonado &Italy

\end{tabular}



 


 New in Csound version 3.47


 Thanks goes to Rasmus Ekman for pointing out the correct MIDI channel and controller number ranges.
%\hline 


\begin{comment}
\begin{tabular}{lcr}
Previous &Home &Next \\
mdelay &Up &midic21

\end{tabular}


\end{document}
\end{comment}
