\begin{comment}
\documentclass[10pt]{article}
\usepackage{fullpage, graphicx, url}
\setlength{\parskip}{1ex}
\setlength{\parindent}{0ex}
\title{control}
\begin{document}


\begin{tabular}{ccc}
The Alternative Csound Reference Manual & & \\
Previous & &Next

\end{tabular}

%\hline 
\end{comment}
\section{control}
control�--� Configurable slider controls for realtime user input. \subsection*{Description}


  Configurable slider controls for realtime user input. Requires Winsound or TCL/TK. \emph{control}
 reads a slider's value. 
\subsection*{Syntax}


 kr \textbf{control}
 knum
\subsection*{Performance}


 \emph{knum}
 -- number of the slider to be read. 


  Calling \emph{control}
 will create a new slider on the screen. There is no theoretical limit to the number of sliders. Windows and TCL/TK use only integers for slider values, so the values may need rescaling. GUIs usually pass values at a fairly slow rate, so it may be advisable to pass the output of control through \emph{port}
. 
\subsection*{Examples}


  See the \emph{setctrl}
 opcode for an example. 
\subsection*{See Also}


 \emph{setctrl}

\subsection*{Credits}


 


 


\begin{tabular}{cccc}
Author: John ffitch &University of Bath, Codemist. Ltd. &Bath, UK &May, 2000

\end{tabular}



 


 New in Csound version 4.06
%\hline 


\begin{comment}
\begin{tabular}{lcr}
Previous &Home &Next \\
comb &Up &convle

\end{tabular}


\end{document}
\end{comment}
