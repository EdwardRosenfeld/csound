\begin{comment}
\documentclass[10pt]{article}
\usepackage{fullpage, graphicx, url}
\setlength{\parskip}{1ex}
\setlength{\parindent}{0ex}
\title{taninv}
\begin{document}


\begin{tabular}{ccc}
The Alternative Csound Reference Manual & & \\
Previous & &Next

\end{tabular}

%\hline 
\end{comment}
\section{taninv}
taninv�--� Performs an arctangent function. \subsection*{Description}


  Returns the arctangent of \emph{x}
 (\emph{x}
 in radians). 
\subsection*{Syntax}


 \textbf{taninv}
(x) (no rate restriction)
\subsection*{Examples}


  Here is an example of the taninv opcode. It uses the files \emph{taninv.orc}
 and \emph{taninv.sco}
. 


 \textbf{Example 1. Example of the taninv opcode.}

\begin{lstlisting}
/* taninv.orc */
; Initialize the global variables.
sr = 44100
kr = 4410
ksmps = 10
nchnls = 1

; Instrument #1.
instr 1
  irad = 0.5
  i1 = taninv(irad)

  print i1
endin
/* taninv.orc */
        
\end{lstlisting}
\begin{lstlisting}
/* taninv.sco */
; Play Instrument #1 for one second.
i 1 0 1
e
/* taninv.sco */
        
\end{lstlisting}
 Its output should include a line like this: \begin{lstlisting}
instr 1:  i1 = 0.464
      
\end{lstlisting}
\subsection*{See Also}


 \emph{cos}
, \emph{cosh}
, \emph{cosinv}
, \emph{sin}
, \emph{sinh}
, \emph{sininv}
, \emph{tan}
, \emph{tanh}
, \emph{taninv2}

\subsection*{Credits}


 Example written by Kevin Conder.
%\hline 


\begin{comment}
\begin{tabular}{lcr}
Previous &Home &Next \\
tanh &Up &taninv2

\end{tabular}


\end{document}
\end{comment}
