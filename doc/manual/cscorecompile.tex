\begin{comment}
\documentclass[10pt]{article}
\usepackage{fullpage, graphicx, url}
\setlength{\parskip}{1ex}
\setlength{\parindent}{0ex}
\title{Compiling a Cscore Program}
\begin{document}


\begin{tabular}{ccc}
The Alternative Csound Reference Manual & & \\
Previous &Cscore &Next

\end{tabular}

%\hline 
\end{comment}
\section{Compiling a Cscore Program}


  A \emph{Cscore}
 program can be invoked either as a Standalone program or as part of Csound: 


 
\begin{lstlisting}
\emph{cscore -U pvanal}
 scorename outfilename
      
\end{lstlisting}


 
 or 

 
\begin{lstlisting}
\emph{csound}
 -C [otherflags] orchname scorename
      
\end{lstlisting}


 


  To create a standalone program, write a \emph{cscore.c}
 program as shown above and test compile it with '\emph{cc cscore.c'}
. If the compiler cannot find ``\emph{cscore.h}
``, try using \emph{-I/usr/local/include}
, or just copy the \emph{cscore.h }
module from the Csound source directory into your own. There will still be unresolved references, so you must now link your program with certain Csound\emph{ }
I/O modules. If your\emph{ }
Csound installation has created a \emph{libcscore.a}
, you can type 


 
\begin{lstlisting}
cc -o cscore.c -lcscore
      
\end{lstlisting}


 
 Else set an environment variable to a Csound directory containing the already compiled modules, and invoke them explicitly: 

 
\begin{lstlisting}
setenv CSOUND /ti/u/bv/Csound
  cc -o cscore cscore.c $CSOUND/cscoremain.o $CSOUND/cscorefns.o \
    $CSOUND/rdscore.o $CSOUND/memalloc.o
      
\end{lstlisting}


 


  The resulting executable can be applied to an input scorefilein by typing: 


 
\begin{lstlisting}
cscore scorefilein scorefileout
      
\end{lstlisting}


 


  To operate from CSound, first proceed as above then link your program to a complete set of Csound\emph{ }
modules. If your Csound installation has created a \emph{libcsound.a}
, you can do this by typing 


 
\begin{lstlisting}
cc -o mycsound cscore.o -lcsounc -lX11 -lm (X11 if your installation included it)
      
\end{lstlisting}


 
 Else copy \emph{*.c, *.h}
 and \emph{Makefile}
 from the Csound source directory, replace \emph{cscore.c}
 by your own, then run ``\textbf{make CSound}
``. The resulting executable is your own special Csound, usable as above. The \emph{-C flag}
 will invoke your \emph{Cscore}
 program after the input score is sorted into ``\emph{score.srt}
``. With no \emph{lplay}
, the subsequent stages of processing can be seen in the files ``\emph{cscore.out}
`` and ``\emph{cscore.srt}
``. %\hline 


\begin{comment}
\begin{tabular}{lcr}
Previous &Home &Next \\
More Advanced Examples &Up &Adding your own Cmodules to Csound

\end{tabular}


\end{document}
\end{comment}
