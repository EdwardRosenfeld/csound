\begin{comment}
\documentclass[10pt]{article}
\usepackage{fullpage, graphicx, url}
\setlength{\parskip}{1ex}
\setlength{\parindent}{0ex}
\title{deltapn}
\begin{document}


\begin{tabular}{ccc}
The Alternative Csound Reference Manual & & \\
Previous & &Next

\end{tabular}

%\hline 
\end{comment}
\section{deltapn}
deltapn�--� Taps a delay line at variable offset times. \subsection*{Description}


  Tap a delay line at variable offset times. 
\subsection*{Syntax}


 ar \textbf{deltapn}
 xnumsamps
\subsection*{Performance}


 \emph{xnumsamps}
 -- specifies the tapped delay time in number of samples. Each can range from 1 control period to the full delay time of the read/write pair; however, since there is no internal check for adherence to this range, the user is wholly responsible. Each argument can be a constant, a variable, or a time-varying signal. 


 \emph{deltapn}
 is identical to \emph{deltapi}
, except delay time is specified in number of samples, instead of seconds (Hans Mikelson). 


  This opcode can tap into a \emph{delayr}
/\emph{delayw}
 pair, extracting delayed audio from the \emph{idlt}
 seconds of stored sound. There can be any number of \emph{deltap}
 and/or \emph{deltapi}
 units between a read/write pair. Each receives an audio tap with no change of original amplitude. 


  This opcode can provide multiple delay taps for arbitrary delay path and feedback networks. They can deliver either constant-time or time-varying taps, and are useful for building chorus effects, harmonizers, and Doppler shifts. Constant-time delay taps (and some slowly changing ones) do not need interpolated readout; they are well served by \emph{deltap}
. Medium-paced or fast varying dlt's, however, will need the extra services of \emph{deltapi}
. 


 \emph{delayr}
/\emph{delayw}
 pairs may be interleaved. To associate a delay tap unit with a specific \emph{delayr}
 unit, it not only has to be located between that \emph{delayr}
 and the appropriate \emph{delayw}
 unit, but must also precede any following \emph{delayr}
 units. See Example 2. (This feature added in Csound version 3.57 by Jens Groh and John ffitch). 


 \emph{N.B.}
 k-rate delay times are not internally interpolated, but rather lay down stepped time-shifts of audio samples; this will be found quite adequate for slowly changing tap times. For medium to fast-paced changes, however, one should provide a higher resolution audio-rate timeshift as input. 
\subsection*{Examples}


 


 \textbf{Example 1. deltap example \#1}

\begin{lstlisting}
  asource  \emph{buzz}
      1, 440, 20, 1
  atime    \emph{linseg}
    1, p3/2,.01, p3/2,1   ; trace a distance in secs
  ampfac   \emph{=}
         1/atime/atime         ; and calc an amp factor
  adump    \emph{delayr}
    1                     ; set maximum distance
  amove    \emph{deltapi}
   atime                 ; move sound source past
           \emph{delayw}
    asource               ; the listener
           \emph{out}
       amove * ampfac
        
\end{lstlisting}


 


 \textbf{Example 2. deltap example \#2}

\begin{lstlisting}
  ainput1 =	..... 
  ainput2 =	..... 
  kdlyt1  =	..... 
  kdlyt2  =	..... 

;Read delayed signal, first delayr instance:
  adump   \emph{delayr}
  4.0 
  adly1   \emph{deltap}
  kdlyt1       ;associated with first delayr instance 

;Read delayed signal, second delayr instance:
  adump   \emph{delayr}
  4.0 
  adly2   \emph{deltap}
  kdlyt2       ; associated with second delayr instance 

;Do some cross-coupled manipulation: 
  afdbk1  =       0.7 * adly1 + 0.7 * adly2 + ainput1 
  afdbk2  =       -0.7 * adly1 + 0.7 * adly2 + ainput2 

;Feed back signal, associated with first delayr instance: 
          \emph{delayw}
  afdbk1 

;Feed back signal, associated with second delayr instance: 
          \emph{delayw}
  afdbk2
          \emph{outs}
    adly1, adly2
        
\end{lstlisting}
\subsection*{See Also}


 \emph{deltap}
, \emph{deltap3}
, \emph{deltapi}

%\hline 


\begin{comment}
\begin{tabular}{lcr}
Previous &Home &Next \\
deltapi &Up &deltapx

\end{tabular}


\end{document}
\end{comment}
