\begin{comment}
\documentclass[10pt]{article}
\usepackage{fullpage, graphicx, url}
\setlength{\parskip}{1ex}
\setlength{\parindent}{0ex}
\title{moscil}
\begin{document}


\begin{tabular}{ccc}
The Alternative Csound Reference Manual & & \\
Previous & &Next

\end{tabular}

%\hline 
\end{comment}
\section{moscil}
moscil�--� Sends a stream of the MIDI notes. \subsection*{Description}


  Sends a stream of the MIDI notes. 
\subsection*{Syntax}


 \textbf{moscil}
 kchn, knum, kvel, kdur, kpause
\subsection*{Performance}


 \emph{kchn}
 -- MIDI channel number (1-16) 


 \emph{knum}
 -- note number (0-127) 


 \emph{kvel}
 -- velocity (0-127) 


 \emph{kdur}
 -- note duration in seconds 


 \emph{kpause}
 -- pause duration after each noteoff and before new note in seconds 


 \emph{moscil}
 and \emph{midion}
 are the most powerful MIDI OUT opcodes. \emph{moscil}
 (MIDI oscil) plays a stream of notes of \emph{kdur}
 duration. Channel, pitch, velocity, duration and pause can be controlled at k-rate, allowing very complex algorithmically generated melodic lines. When current instrument is deactivated, the note played by current instance of \emph{moscil}
 is forcedly truncated. 


  Any number of \emph{moscil}
 opcodes can appear in the same Csound instrument, allowing a counterpoint-style polyphony within a single instrument. 
\subsection*{See Also}


 \emph{midion}

\subsection*{Credits}


 


 


\begin{tabular}{ccc}
Author: Gabriel Maldonado &Italy &May 1997

\end{tabular}



 


 New in Csound version 3.47


 Thanks goes to Rasmus Ekman for pointing out the correct MIDI channel and controller number ranges.
%\hline 


\begin{comment}
\begin{tabular}{lcr}
Previous &Home &Next \\
moogvcf &Up &mpulse

\end{tabular}


\end{document}
\end{comment}
