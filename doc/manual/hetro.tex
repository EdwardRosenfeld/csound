\begin{comment}
\documentclass[10pt]{article}
\usepackage{fullpage, graphicx, url}
\setlength{\parskip}{1ex}
\setlength{\parindent}{0ex}
\title{hetro}
\begin{document}


\begin{tabular}{ccc}
The Alternative Csound Reference Manual & & \\
Previous & &Next

\end{tabular}

%\hline 
\end{comment}
\section{hetro}
hetro�--� Decomposes an input soundfile into component sinusoids. \subsection*{Description}


  Hetrodyne filter analysis for the Csound \emph{adsyn}
 generator. 
\subsection*{Syntax}


 \textbf{csound -U hetro}
 [flags] infilename outfilename


 \textbf{hetro}
 [flags] infilename outfilename
\subsection*{Initialization}


 \emph{hetro}
 takes an input soundfile, decomposes it into component sinusoids, and outputs a description of the components in the form of breakpoint amplitude and frequency tracks. Analysis is conditioned by the control flags below. A space is optional between flag and value. 


 \emph{-s srate}
 -- sampling rate of the audio input file. This will over-ride the srate of the soundfile header, which otherwise applies. If neither is present, the default is 10000. Note that for \emph{adsyn}
 synthesis the srate of the source file and the generating orchestra need not be the same. 


 \emph{-c channel}
 -- channel number sought. The default is 1. 


 \emph{-b begin}
 -- beginning time (in seconds) of the audio segment to be analyzed. The default is 0.0 


 \emph{-d duration}
 -- duration (in seconds) of the audio segment to be analyzed. The default of 0.0 means to the end of the file. Maximum length is 32.766 seconds. 


 \emph{-f begfreq}
 -- estimated starting frequency of the fundamental, necessary to initialize the filter analysis. The default is 100 (cps). 


 \emph{-h partials}
 -- number of harmonic partials sought in the audio file. Default is 10, maximum is a function of memory available. 


 \emph{-M maxamp}
 -- maximum amplitude summed across all concurrent tracks. The default is 32767. 


 \emph{-m minamp}
 -- amplitude threshold below which a single pair of amplitude/frequency tracks is considered dormant and will not contribute to output summation. Typical values: 128 (48 db down from full scale), 64 (54 db down), 32 (60 db down), 0 (no thresholding). The default threshold is 64 (54 db down). 


 \emph{-n brkpts}
 -- initial number of analysis breakpoints in each amplitude and frequency track, prior to thresholding (\emph{-m}
) and linear breakpoint consolidation. The initial points are spread evenly over the duration. The default is 256. 


 \emph{-l cutfreq}
 -- substitute a 3rd order Butterworth low-pass filter with cutoff frequency \emph{cutfreq}
 (in Hz), in place of the default averaging comb filter. The default is 0 (don't use). 
\subsection*{Performance}


  As of Csound 4.08, \emph{hetro}
 can write SDIF ouput files if the output file name ends with ``.sdif''. See the \emph{sdif2ad utility}
 for more information about the Csound's SDIF support. 
\subsection*{Examples}


 


 
\begin{lstlisting}
\emph{hetro}
 -s44100 -b.5 -d2.5 -h16 -M24000 audiofile.test adsynfile7
        
\end{lstlisting}


 
 This will analyze 2.5 seconds of channel 1 of a file ``audiofile.test'', recorded at 44.1 kHz, beginning .5 seconds from the start, and place the result in a file ``adsynfile7''. We request just the first 16 harmonics of the sound, with 256 initial breakpoint values per amplitude or frequency track, and a peak summation amplitude of 24000. The fundamental is estimated to begin at 100 Hz. Amplitude thresholding is at 54 db down. 

  The Butterworth LPF is not enabled. 
\subsubsection*{File Format}


  The output file contains time-sequenced amplitude and frequency values for each partial of an additive complex audio source. The information is in the form of breakpoints (time, value, time, value, ....) using 16-bit integers in the range 0 - 32767. Time is given in milliseconds, and frequency in Hertz (cps). The breakpoint data is exclusively non-negative, and the values -1 and -2 uniquely signify the start of new amplitude and frequency tracks. A track is terminated by the value 32767. Before being written out, each track is data-reduced by amplitude thresholding and linear breakpoint consolidation. 


  A component partial is defined by two breakpoint sets: an amplitude set, and a frequency set. Within a composite file these sets may appear in any order (amplitude, frequency, amplitude ....; or amplitude, amplitude..., then frequency, frequency,...). During \emph{adsyn}
 resynthesis the sets are automatically paired (amplitude, frequency) from the order in which they were found. There should be an equal number of each. 


  A legal \emph{adsyn}
 control file could have following format: 


 
\begin{lstlisting}
-1 time1 value1 ... timeK valueK 32767 ; amplitude breakpoints for partial 1
-2 time1 value1 ... timeL valueL 32767 ; frequency breakpoints for partial 1
-1 time1 value1 ... timeM valueM 32767 ; amplitude breakpoints for partial 2
-2 time1 value1 ... timeN valueN 32767 ; frequency breakpoints for partial 2
-2 time1 value1 ..........
-2 time1 value1 ..........             ; pairable tracks for partials 3 and 4
-1 time1 value1 ..........
-1 time2 value1 ..........
          
\end{lstlisting}


 
\subsection*{Credits}


  October 2002. Thanks to Rasmus Ekman, added a note about the SDIF format. 
%\hline 


\begin{comment}
\begin{tabular}{lcr}
Previous &Home &Next \\
Analysis File Generation &Up &lpanal

\end{tabular}


\end{document}
\end{comment}
