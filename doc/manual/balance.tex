\begin{comment}
\documentclass[10pt]{article}
\usepackage{fullpage, graphicx, url}
\setlength{\parskip}{1ex}
\setlength{\parindent}{0ex}
\title{balance}
\begin{document}


\begin{tabular}{ccc}
The Alternative Csound Reference Manual & & \\
Previous & &Next

\end{tabular}

%\hline 
\end{comment}
\section{balance}
balance�--� Adjust one audio signal according to the values of another. \subsection*{Description}


  The rms power of asig can be interrogated, set, or adjusted to match that of a comparator signal. 
\subsection*{Syntax}


 ar \textbf{balance}
 asig, acomp [, ihp] [, iskip]
\subsection*{Initialization}


 \emph{ihp}
 (optional) -- half-power point (in Hz) of a special internal low-pass filter. The default value is 10. 


 \emph{iskip}
 (optional, default=0) -- initial disposition of internal data space (see \emph{reson}
). The default value is 0. 
\subsection*{Performance}


 \emph{asig}
 -- input audio signal 


 \emph{acomp}
 -- the comparator signal 


 \emph{balance}
 outputs a version of \emph{asig}
, amplitude-modified so that its rms power is equal to that of a comparator signal \emph{acomp}
. Thus a signal that has suffered loss of power (eg., in passing through a filter bank) can be restored by matching it with, for instance, its own source. It should be noted that \emph{gain}
 and \emph{balance}
 provide amplitude modification only - output signals are not altered in any other respect. 
\subsection*{Examples}


  Here is an example of the balance opcode. It uses the files \emph{balance.orc}
 and \emph{balance.sco}
. 


 \textbf{Example 1. Example of the balance opcode.}

\begin{lstlisting}
/* balance.orc */
; Initialize the global variables.
sr = 44100
kr = 4410
ksmps = 10
nchnls = 1

; Instrument #1.
instr 1
  ; Generate a band-limited pulse train.
  asrc buzz 30000, 440, sr/440, 1

  ; Send the source signal through 2 filters.
  a1 reson asrc, 1000, 100       
  a2 reson a1, 3000, 500

  ; Balance the filtered signal with the source.
  afin balance a2, asrc

  out afin
endin
/* balance.orc */
        
\end{lstlisting}
\begin{lstlisting}
/* balance.sco */
; Table #1, a sine wave.
f 1 0 16384 10 1

; Play Instrument #1 for two seconds.
i 1 0 2
e
/* balance.sco */
        
\end{lstlisting}
\subsection*{See Also}


 \emph{gain}
, \emph{rms}

%\hline 


\begin{comment}
\begin{tabular}{lcr}
Previous &Home &Next \\
babo &Up &bamboo

\end{tabular}


\end{document}
\end{comment}
