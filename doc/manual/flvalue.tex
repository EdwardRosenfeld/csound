\begin{comment}
\documentclass[10pt]{article}
\usepackage{fullpage, graphicx, url}
\setlength{\parskip}{1ex}
\setlength{\parindent}{0ex}
\title{FLvalue}
\begin{document}


\begin{tabular}{ccc}
The Alternative Csound Reference Manual & & \\
Previous & &Next

\end{tabular}

%\hline 
\end{comment}
\section{FLvalue}
FLvalue�--� Shows the current value of a FLTK valuator. \subsection*{Description}


 \emph{FLvalue}
 shows current the value of a valuator in a text field. 
\subsection*{Syntax}


 ihandle \textbf{FLvalue}
 ``label'', iwidth, iheight, ix, iy
\subsection*{Initialization}


 \emph{ihandle}
 -- handle value (an integer number) that unequivocally references the corresponding valuator. It can be used for the \emph{idisp}
 argument of a valuator. 


 \emph{``label''}
 -- a double-quoted string containing some user-provided text, placed near the corresponding widget. 


 \emph{iwidth}
 -- width of widget. 


 \emph{iheight}
 -- height of widget. 


 \emph{ix}
 -- horizontal position of upper left corner of the valuator, relative to the upper left corner of corresponding window (expressed in pixels). 


 \emph{iy}
 -- vertical position of upper left corner of the valuator, relative to the upper left corner of corresponding window (expressed in pixels). 
\subsection*{Performance}


  Note that \emph{FLvalue}
 is not a valuator and its value is fixed. Its value cannot be modified. 


 \emph{FLvalue}
 shows the current values of a valuator in a text field. It outputs \emph{ihandle}
 that can then be used for the \emph{idisp}
 argument of a valuator (see the \emph{FLTK Valuators section}
). In this way, the values of that valuator will be dynamically be shown in a text field. 
\subsection*{Examples}


  Here is an example of the flvalue opcode. It uses the files \emph{flvalue.orc}
 and \emph{flvalue.sco}
. 


 \textbf{Example 1. Example of the flvalue opcode.}

\begin{lstlisting}
/* flvalue.orc */
; Using the opcode flvalue to display the output of a slider 
sr = 44100
kr = 441
ksmps = 100
nchnls = 1

FLpanel "Value Display Box", 900, 200, 50, 50
    ; Width of the value display box in pixels
    iwidth = 50
    ; Height of the value display box in pixels
    iheight = 20
    ; Distance of the left edge of the value display
    ; box from the left edge of the panel
    ix = 65
    ; Distance of the top edge of the value display
    ; box from the top edge of the panel
    iy = 55

    idisp FLvalue "Hertz", iwidth, iheight, ix, iy
    gkfreq, ihandle FLslider "Frequency", 200, 5000, -1, 5, idisp, 750, 30, 125, 50
    FLsetVal_i 500, ihandle
; End of panel contents
FLpanelEnd
; Run the widget thread!
FLrun

instr 1
    iamp = 15000
    ifn = 1
    asig oscili iamp, gkfreq, ifn
    out asig
endin
/* flvalue.orc */
        
\end{lstlisting}
\begin{lstlisting}
/* flvalue.sco */
; Function table that defines a single cycle
; of a sine wave.
f 1 0 1024 10 1

; Instrument 1 will play a note for 1 hour.
i 1 0 3600
e
/* flvalue.sco */
        
\end{lstlisting}
\subsection*{See Also}


 \emph{FLbox}
, \emph{FLbutBank}
, \emph{FLbutton}
, \emph{FLprintk}
, \emph{FLprintk2}

\subsection*{Credits}


 Author: Gabriel Maldonado


 New in version 4.22


 Example written by Iain McCurdy, edited by Kevin Conder.
%\hline 


\begin{comment}
\begin{tabular}{lcr}
Previous &Home &Next \\
FLupdate &Up &fmb3

\end{tabular}


\end{document}
\end{comment}
