\begin{comment}
\documentclass[10pt]{article}
\usepackage{fullpage, graphicx, url}
\setlength{\parskip}{1ex}
\setlength{\parindent}{0ex}
\title{flashtxt}
\begin{document}


\begin{tabular}{ccc}
The Alternative Csound Reference Manual & & \\
Previous & &Next

\end{tabular}

%\hline 
\end{comment}
\section{flashtxt}
flashtxt�--� Allows text to be displayed from instruments like sliders \subsection*{Description}


  Allows text to be displayed from instruments like sliders etc. (only on Unix and Windows at present) 
\subsection*{Syntax}


 \textbf{flashtxt}
 iwhich, String
\subsection*{Initialization}


 \emph{iwhich}
 -- the number of the window. 


 \emph{String}
 -- the string to be displayed. 
\subsection*{Performance}


  A window is created, identified by the iwhich argument, with the text string displayed. If the text is replaced by a number then the window id deleted. Note that the text windows are globally numbered so different instruments can change the text, and the window survives the instance of the instrument. 
\subsection*{Examples}


  Here is an example of the flashtxt opcode. It uses the files \emph{flashtxt.orc}
 and \emph{flashtxt.sco}
. 


 \textbf{Example 1. Example of the flashtxt opcode.}

\begin{lstlisting}
/* flashtxt.orc */
; Initialize the global variables.
sr = 44100
kr = 44100
ksmps = 1
nchnls = 1

instr 1
  flashtxt 1, "Instr 1 live"
  ao oscil 4000, 440, 1
  out ao
endin
/* flashtxt.orc */
        
\end{lstlisting}
\begin{lstlisting}
/* flashtxt.sco */
; Table 1: an ordinary sine wave.
f 1 0 32768 10 1 

; Play Instrument #1 for three seconds.
i 1 0 3
e
/* flashtxt.sco */
        
\end{lstlisting}
%\hline 


\begin{comment}
\begin{tabular}{lcr}
Previous &Home &Next \\
flanger &Up &FLbox

\end{tabular}


\end{document}
\end{comment}
