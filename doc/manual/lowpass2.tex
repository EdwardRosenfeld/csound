\begin{comment}
\documentclass[10pt]{article}
\usepackage{fullpage, graphicx, url}
\setlength{\parskip}{1ex}
\setlength{\parindent}{0ex}
\title{lowpass2}
\begin{document}


\begin{tabular}{ccc}
The Alternative Csound Reference Manual & & \\
Previous & &Next

\end{tabular}

%\hline 
\end{comment}
\section{lowpass2}
lowpass2�--� A resonant lowpass filter. \subsection*{Description}


  Implementation of a resonant second-order lowpass filter. 
\subsection*{Syntax}


 ar \textbf{lowpass2}
 asig, kcf, kq [, iskip]
\subsection*{Initialization}


 \emph{iskip}
 -- initial disposition of internal data space. A zero value will clear the space; a non-zero value will allow previous information to remain. The default value is 0. 
\subsection*{Performance}


 \emph{asig}
 -- input signal to be filtered 


 \emph{kcf}
 -- cutoff or resonant frequency of the filter, measured in Hz 


 \emph{kq}
 -- Q of the filter, defined, for bandpass filters, as bandwidth/cutoff. \emph{kq}
 should be between 1 and 500 


 \emph{lowpass2}
 is a second order IIR lowpass filter, with k-rate controls for cutoff frequency (\emph{kcf}
) and Q (\emph{kq}
). As \emph{kq}
 is increased, a resonant peak forms around the cutoff frequency, transforming the lowpass filter response into a response that is similar to a bandpass filter, but with more low frequency energy. This corresponds to an increase in the magnitude and ``sharpness'' of the resonant peak. For high values of \emph{kq}
, a scaling function such as \emph{balance}
 may be required. In practice, this allows for the simulation of the voltage-controlled filters of analog synthesizers, or for the creation of a pitch of constant amplitude while filtering white noise. 
\subsection*{Examples}


  Here is an example of the lowpass2 opcode. It uses the files \emph{lowpass2.orc}
 and \emph{lowpass2.sco}
. 


 \textbf{Example 1. Example of the lowpass2 opcode.}

\begin{lstlisting}
/* lowpass.orc */
/* Written by Sean Costello */
; Orchestra file for resonant filter sweep of a sawtooth-like waveform.
  sr = 44100
  kr = 2205
  ksmps = 20
  nchnls = 1

          instr 1

  idur    =          p3
  ifreq   =          p4
  iamp    =          p5 * .5
  iharms  =          (sr*.4) / ifreq

; Sawtooth-like waveform
  asig    gbuzz 1, ifreq, iharms, 1, .9, 1

; Envelope to control filter cutoff 
  kfreq   linseg 1, idur * 0.5, 5000, idur * 0.5, 1

  afilt   lowpass2 asig, kfreq, 30

; Simple amplitude envelope
  kenv    linseg 0, .1, iamp, idur -.2, iamp, .1, 0 
          out asig * kenv

          endin
/* lowpass.orc */
        
\end{lstlisting}
\begin{lstlisting}
/* lowpass2.sco */
/* Written by Sean Costello */
f1 0 8192 9 1 1 .25

i1 0 5 100 1000
i1 5 5 200 1000
e
/* lowpass2.sco */
        
\end{lstlisting}
\subsection*{Credits}


 


 


\begin{tabular}{ccc}
Author: Sean Costello &Seattle, Washington &August 1999

\end{tabular}



 


 New in Csound version 4.0
%\hline 


\begin{comment}
\begin{tabular}{lcr}
Previous &Home &Next \\
loscil3 &Up &lowres

\end{tabular}


\end{document}
\end{comment}
