\begin{comment}
\documentclass[10pt]{article}
\usepackage{fullpage, graphicx, url}
\setlength{\parskip}{1ex}
\setlength{\parindent}{0ex}
\title{lpf18}
\begin{document}


\begin{tabular}{ccc}
The Alternative Csound Reference Manual & & \\
Previous & &Next

\end{tabular}

%\hline 
\end{comment}
\section{lpf18}
lpf18�--� A 3-pole sweepable resonant lowpass filter. \subsection*{Description}


  Implementation of a 3 pole sweepable resonant lowpass filter. 
\subsection*{Syntax}


 ar \textbf{lpf18}
 asig, kfco, kres, kdist
\subsection*{Performance}


 \emph{kfco}
 -- the filter cutoff frequency in Hz. Should be in the range 0 to sr/2. 


 \emph{kres}
 -- the amount of resonance. Self-oscillation occurs when \emph{kres}
 is approximately 1. Shoujld usually be in the range 0 to 1, however, values slightly greater than 1 are possible for more sustained oscillation and an ``overdrive'' effect. 


 \emph{kdist}
 -- amount of distortion. \emph{kdis}
t = 0 gives a clean output. \emph{kdist}
 $>$ 0 adds \emph{tanh}
() distortion controlled by the filter parameters, in such a way that both low cutoff and high resonance increase the distortion amount. Some experimentation is encouraged. 


 \emph{lpf18}
 is a digital emulation of a 3 pole (18 dB/oct.) lowpass filter capable of self-oscillation with a built-in distortion unit. It is really a 3-pole version of \emph{moogvcf}
, retuned, recalibrated and with some performance improvements. The tuning and feedback tables use no more than 6 adds and 6 multiplies per control rate. The distortion unit, itself, is based on a modified \emph{tanh}
 function driven by the filter controls. 


 


\begin{tabular}{cc}
\textbf{Note}
 \\
� &

  This filter requires that the input signal be normalized to one. 


\end{tabular}

\subsection*{Examples}


  Here is an example of the lpf18 opcode. It uses the files \emph{lpf18.orc}
 and \emph{lpf18.sco}
. 


 \textbf{Example 1. Example of the lpf18 opcode.}

\begin{lstlisting}
/* lpf18.orc */
; Initialize the global variables.
sr = 44100
kr = 4410
ksmps = 10
nchnls = 1

; Instrument #1.
instr 1
  ; Generate a sine waveform.
  ; Note that its amplitude (kamp) ranges from 0 to 1.
  kamp init 1
  kcps init 440
  knh init 3
  ifn = 1
  asine buzz kamp, kcps, knh, ifn

  ; Filter the sine waveform.
  ; Vary the cutoff frequency (kfco) from 300 to 3,000 Hz.
  kfco line 300, p3, 3000
  kres init 0.8
  kdist init 0.3
  aout lpf18 asine, kfco, kres, kdist

  out aout * 30000
endin
/* lpf18.orc */
        
\end{lstlisting}
\begin{lstlisting}
/* lpf18.sco */
; Table #1, a sine wave.
f 1 0 16384 10 1

; Play Instrument #1 for four seconds.
i 1 0 4
e
/* lpf18.sco */
        
\end{lstlisting}
\subsection*{Credits}


 


 


\begin{tabular}{ccc}
Author: Josep M Comajuncosas &Spain &December 2000

\end{tabular}



 


 Example written by Kevin Conder with help from Iain Duncan. Thanks goes to Iain for helping with the example.


 New in Csound version 4.10
%\hline 


\begin{comment}
\begin{tabular}{lcr}
Previous &Home &Next \\
lowresx &Up &lpfreson

\end{tabular}


\end{document}
\end{comment}
