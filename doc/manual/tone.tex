\begin{comment}
\documentclass[10pt]{article}
\usepackage{fullpage, graphicx, url}
\setlength{\parskip}{1ex}
\setlength{\parindent}{0ex}
\title{tone}
\begin{document}


\begin{tabular}{ccc}
The Alternative Csound Reference Manual & & \\
Previous & &Next

\end{tabular}

%\hline 
\end{comment}
\section{tone}
tone�--� A first-order recursive low-pass with variable frequency response. \subsection*{Description}


  A first-order recursive low-pass with variable frequency response. 


  Tone is a 1 term IIR filter. Its formula is: 


 


  yn = c1 * xn + c2 * yn-1


 
 where 

 


 


 
\begin{itemize}
\item 

 b = 2 - cos(2 \&\#928; hp/sr);

\item 

 c2 = b - sqrt(b2 - 1.0)

\item 

 c1 = 1 - c2


\end{itemize}


 
\subsection*{Syntax}


 ar \textbf{tone}
 asig, khp [, iskip]
\subsection*{Initialization}


 \emph{iskip}
 (optional, default=0) -- initial disposition of internal data space. Since filtering incorporates a feedback loop of previous output, the initial status of the storage space used is significant. A zero value will clear the space; a non-zero value will allow previous information to remain. The default value is 0. 
\subsection*{Performance}


 \emph{ar}
 -- the output audio signal. 


 \emph{asig}
 -- the input audio signal. 


 \emph{khp}
 -- the response curve's half-power point, in Hertz. Half power is defined as peak power / root 2. 


 \emph{tone}
 implements a first-order recursive low-pass filter in which the variable \emph{khp}
 (in Hz) determines the response curve's half-power point. Half power is defined as peak power / root 2. 
\subsection*{See Also}


 \emph{areson}
, \emph{aresonk}
, \emph{atone}
, \emph{atonek}
, \emph{port}
, \emph{portk}
, \emph{reson}
, \emph{resonk}
, \emph{tonek}

%\hline 


\begin{comment}
\begin{tabular}{lcr}
Previous &Home &Next \\
tlineto &Up &tonek

\end{tabular}


\end{document}
\end{comment}
