\begin{comment}
\documentclass[10pt]{article}
\usepackage{fullpage, graphicx, url}
\setlength{\parskip}{1ex}
\setlength{\parindent}{0ex}
\title{FLcolor2}
\begin{document}


\begin{tabular}{ccc}
The Alternative Csound Reference Manual & & \\
Previous & &Next

\end{tabular}

%\hline 
\end{comment}
\section{FLcolor2}
FLcolor2�--� A FLTK opcode that sets the secondary (selection) color. \subsection*{Description}


 \emph{FLcolor2}
 is the same of \emph{FLcolor}
 except it affects the secondary (selection) color. 
\subsection*{Syntax}


 \textbf{FLcolor2}
 ired, igreen, iblue
\subsection*{Initialization}


 \emph{ired}
 -- The red color of the target widget. The range for each RGB component is 0-255 


 \emph{igreen}
 -- The green color of the target widget. The range for each RGB component is 0-255 


 \emph{iblue}
 -- The blue color of the target widget. The range for each RGB component is 0-255 
\subsection*{Performance}


  These opcodes modify the appearance of other widgets. There are two types of such opcodes: those that don't contain the \emph{ihandle}
 argument which affect all subsequently declared widgets, and those without \emph{ihandle}
 which affect only a target widget previously defined. 


 \emph{FLcolor2}
 is the same of \emph{FLcolor}
 except it affects the secondary (selection) color. Setting it to -1 turns off the influence of \emph{FLcolor2}
 on subsequent widgets. A value of -2 (or -3) makes all next widget secondary colors randomly selected. The difference is that -2 selects a light random color, while -3 selects a dark random color. 
\subsection*{See Also}


 \emph{FLcolor}
, \emph{FLhide}
, \emph{FLlabel}
, \emph{FLsetAlign}
, \emph{FLsetBox}
, \emph{FLsetColor}
, \emph{FLsetColor2}
, \emph{FLsetFont}
, \emph{FLsetPosition}
, \emph{FLsetSize}
, \emph{FLsetText}
, \emph{FLsetTextColor}
, \emph{FLsetTextSize}
, \emph{FLsetTextType}
, \emph{FLsetVal\_i}
, \emph{FLsetVal}
, \emph{FLshow}

\subsection*{Credits}


 Author: Gabriel Maldonado


 New in version 4.22
%\hline 


\begin{comment}
\begin{tabular}{lcr}
Previous &Home &Next \\
FLcolor &Up &FLcount

\end{tabular}


\end{document}
\end{comment}
