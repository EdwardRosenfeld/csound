\begin{comment}
\documentclass[10pt]{article}
\usepackage{fullpage, graphicx, url}
\setlength{\parskip}{1ex}
\setlength{\parindent}{0ex}
\title{aresonk}
\begin{document}


\begin{tabular}{ccc}
The Alternative Csound Reference Manual & & \\
Previous & &Next

\end{tabular}

%\hline 
\end{comment}
\section{aresonk}
aresonk�--� A notch filter whose transfer functions are the complements of the reson opcode. \subsection*{Description}


  A notch filter whose transfer functions are the complements of the reson opcode. 
\subsection*{Syntax}


 kr \textbf{aresonk}
 ksig, kcf, kbw [, iscl] [, iskip]
\subsection*{Initialization}


 \emph{iscl}
 (optional, default=0) -- coded scaling factor for resonators. A value of 1 signifies a peak response factor of 1, i.e. all frequencies other than kcf are attenuated in accordance with the (normalized) response curve. A value of 2 raises the response factor so that its overall RMS value equals 1. (This intended equalization of input and output power assumes all frequencies are physically present; hence it is most applicable to white noise.) A zero value signifies no scaling of the signal, leaving that to some later adjustment (see \emph{balance}
). The default value is 0. 


 \emph{iskip}
 (optional, default=0) -- initial disposition of internal data space. Since filtering incorporates a feedback loop of previous output, the initial status of the storage space used is significant. A zero value will clear the space; a non-zero value will allow previous information to remain. The default value is 0. 
\subsection*{Performance}


 \emph{kr}
 -- the output signal at control-rate. 


 \emph{ksig}
 -- the input signal at control-rate. 


 \emph{kcf}
 -- the center frequency of the filter, or frequency position of the peak response. 


 \emph{kbw}
 -- bandwidth of the filter (the Hz difference between the upper and lower half-power points). 


 \emph{aresonk}
 is a filter whose transfer functions is the complement of \emph{reson}
. Thus \emph{aresonk}
 is a notch filter whose transfer functions represents the ``filtered out'' aspects of their complements. However, power scaling is not normalized in \emph{aresonk}
 but remains the true complement of the corresponding unit. 
\subsection*{See Also}


 \emph{areson}
, \emph{atone}
, \emph{atonek}
, \emph{port}
, \emph{portk}
, \emph{reson}
, \emph{resonk}
, \emph{tone}
, \emph{tonek}

%\hline 


\begin{comment}
\begin{tabular}{lcr}
Previous &Home &Next \\
areson &Up &atone

\end{tabular}


\end{document}
\end{comment}
