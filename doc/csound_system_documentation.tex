\section{Csound System Documentation}

\begin{center}\rule{3in}{0.4pt}\end{center}

\subsection{Introduction}

...

\subsection{Orientation}

\subsubsection{CMake Build System}

For Csound 6, we use the \href{http://www.cmake.org}{CMake} build
system. CMake is a meta-build system in that builds project files that
are then used with other build systems. This can produce files for
command-line build systems such as Make or
\href{http://martine.github.io/ninja/}{Ninja}, as well as build projects
for IDE's such as XCode, Eclipse, or KDevelop.

CMake organizes the build files into ones called CMakeLists.txt. Csound
has a top-level CMakeLists.txt that defines some useful functions, as
well as defines the build for libcsound. From there, other
CMakeLists.txt files are included into the top-level one that has build
information for other artifacts, such as command-line executables,
plugin libraries, and GUI applications.

\paragraph{Dependencies}

\paragraph{Targets: libcsound, Executables, Plugin Libraries, and other
Libraries}

\subsubsection{Source Tree Walkthrough}

cs6

root of Csound 6

cs6/include

public headers that will get distributed and used by host-applications
and plugin libraries

cs6/H

private headers that are only used within libcsound itself

cs6/Top

Contains source files for the parts that wrap the engine, including
argument parsing, module loading, CSD reading, CScore, and others

cs6/Engine

Contains source files for the audio engine and compiler of Csound

cs6/InOut

Contains source files for plugin libraries dealing with input and
output, particularly audio, midi, and graphs

cs6/OOps

Contains source files for opcodes. Originally named for ``Original
Opcodes'', now contains mostly opcodes built-in to libcsound

cs6/Opcodes

Contains source files for opcodes. Those that do not have external
library dependencies have been folded back into libcsound. Those with
external dependencies are built as separate opcode plugins.

cs6/interfaces

Contains the source files for building the various interface wrapper
libraries (i.e.~Java, Python), as well as the C++ interface library.

cs6/frontends

Contains the source files for building various frontends (i.e.~CsoundAC,
csoundapi\textasciitilde{}, beats).

cs6/util

Contains the source files for the main csound-related utilities
(i.e.~pvanal, lpanal, dnoise, etc.). The files here are used to build
individual commandline programs, as well as altogether for the stdutil
plugin library.

cs6/util1

Contains the source files for building various frontends (i.e.~CsoundAC,
csoundapi\textasciitilde{}, beats).

cs6/installer

Contains the source code and build scripts for creating various
platform-specific installers.

cs6/android

Contains example projects and build scripts for building Csound for
Android.

cs6/iOS

Contains example projects and build scripts for building Csound for iOS.

\subsection{High-Level Architecture}

The architecture of Csound follows in the tradition of Music-N. It uses
the abstraction of \emph{instruments} for time-schedulable blocks of
code made up of unit generators, called \emph{opcodes}. \ldots{}

\begin{center}\rule{3in}{0.4pt}\end{center}

\section{Part 2: Prelude to Processing}

\subsection{Csound Configuration (Commandline Args/Options)}

\subsection{ORC/SCO/CSD Input}

Text is read through one of two primary paths:

\begin{itemize}
\itemsep1pt\parskip0pt\parsep0pt
\item
  from files
\item
  from memory
\end{itemize}

Both of the paths actually filter through the CORFILE system\ldots{}

\section{Part 2: Orchestra Compiler}

\subsection{Introduction}

\subsection{Compiler Phases}

\subsubsection{Pre-Processing}

\subsubsection{Lexer}

The lexing phase reads in a stream of characters and breaks them up into
tokens. (Lexers are also known as\emph{scanners} or \emph{tokenizers}.)
Csound uses the \href{http://flex.sourceforge.net/}{Flex} tool to
generate its lexing code. The source for this is found in
Engine/csound\_orc.l.

\subsubsection{Parser}

The parsing phase uses the tokens generated from the lexing phase and
uses rules defined in a \emph{grammar} to structure the tokens into a
TREE. The parsing code was originally done with hand-written code, but
is now generated using the
\href{http://www.gnu.org/software/bison/}{Bison} parser-generator tool.
The source for this is found in Engine/csound\_orc.y.

\subsubsection{Semantic Analysis}

\paragraph{Type System}

\paragraph{Lookup}

\paragraph{Expressions}

\paragraph{Blocks}

\subsubsection{Compiler}

\paragraph{Runtime Data Structures}

Instruments, Opcodes, Instrument Instances, Opcode Instances

\paragraph{Labels, Block Expressions}

\paragraph{Expression Expansion}

\paragraph{Transactional Compilation}

\begin{center}\rule{3in}{0.4pt}\end{center}

\section{Part 2: Score Compiler}

\subsection{Event Parsing}

\subsection{Score Sorting}

\begin{center}\rule{3in}{0.4pt}\end{center}

\section{Part 3: Runtime}

\subsection{Introduction}

\subsection{Instrument Instance Lifecycle}

\subsection{Performance}

\subsection{Scheduler and senseEvents}

\subsubsection{Realtime Events}

\paragraph{SCO}

\paragraph{MIDI}

\paragraph{Channels}

\begin{center}\rule{3in}{0.4pt}\end{center}

\section{Part 4: API and Wrappers}

\begin{center}\rule{3in}{0.4pt}\end{center}

\section{Part 5: Developer Information}

\subsection{Csound Testing}

The csound6/tests folder contains various forms of tests for Csound. The
tests are generally written to aid development, testing that new code
functions as expected and that they handle errors correctly. The
following describes the various forms of tests.

\subsubsection{tests/commandline}

This folder contains CSD's that get run by a python test
runner(test.py). These are generally used for testing the compiler and
can be considered integration tests. These can be run from the CMake
generated build file, i.e. ``make csdtests''.

Note: Some tests in the tests/commandline folder are not added to the
test suite. These are generally ones that people have contributed to
illustrate a bug, and were used during debugging. It is useful to have
these around and ideally we will extend the test suite to do runtime
testing as well as compiler testing.

\subsubsection{tests/c}

This folder contains unit tests written in C, using the CUnit library.
Currently there are tests for various parts of the compiler and some API
methods. These tests serve to help ensure we didn't break something
moving forward, and also act as a documentation on how functions are
used. These can be run from the CMake generated tests using ``make
test'' or calling ``ctest''.

\subsubsection{tests/python}

This folder is intended for tests of the Csound API using Python. The
idea is that it would be useful to test from a host language to make
sure our assumptions about the C API still work from a host language.
